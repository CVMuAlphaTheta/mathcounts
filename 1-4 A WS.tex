\documentclass{article}
\usepackage{amsthm}
\theoremstyle{definition}
\newtheorem*{solution}{Solution}
\title{Fractions, Ratios, and Proportional Reasoning Set A}
\author{}
\date{}
\begin{document}
    \maketitle
    \noindent Problems should be solved without calculators unless otherwise
    specified. Remember to explain how you solved a problem.
    \begin{enumerate}
        \item A shop owner increased the price of a jacket by $17\%$. What
        percent of the new price is the original price of the jacket? Express
        your answer to the nearest tenth.
        \begin{solution}
            If the original price of the jacket is $P$, an increase of $17\%$
            percent makes the new price $1.17P$. The ratio of the original price
            to the new price is $\frac{P}{1.17P} = \frac{1}{1.17} \approx 0.855
            = 85.5\%$.
        \end{solution}
        \item A single portion of a special concoction is a mixture of $3$ units
        of ingredient A, $5$ units of ingredient B, and $11$ units of ingredient
        C. If Isabella wants to make $90$ portions, how many more units of
        ingredient C does she need than ingredient B?
        \begin{solution}
            The ratio $A : B : C$ is $3 : 5 : 11$. A single portion of this
            special concoction uses $11 - 5 = 6$ more units of ingredient C than
            ingredient B. If Isabella wants to make $90$ portions, then she
            needs $90 \cdot 6 = 540$ units more of C than B.
        \end{solution}
        \item Kathleen has a pattern to knit a scarf that requires $441$ yards
        of yarn. For this particular pattern, each stitch uses $0.9$ inches of
        yarn, each row contains $41$ stitches and each full row of knitting
        contributes $\frac{1}{5}$ inches of length to the scarf. How many inches
        long will her scarf be when completed? Express your answer to the
        nearest whole number.
        \begin{solution}
            There are $12 \cdot 3 = 36$ inches in a yard, so $441$ yards of yarn
            is $441 \cdot 36 = 15876$ inches of yarn. Each row of the scarf will
            use $0.9 \cdot 41 = 36.9$ inches of yarn, but that’s only
            $\frac{1}{5}$ inch of scarf. After $5$ rows, Kathleen will have $1$
            inch of scarf and she will have used $5 \cdot 36.9 = 184.5$ inches
            of yarn. Now, if we divide her total length of yarn by the inches of
            yarn used per inch of scarf, we see that with the amount of yarn
            Kathleen has, she can make a scarf that has length
            $\frac{15876}{184.5} \approx 86$ inches long.
        \end{solution}
        \item If $x$ and $y$ are inversely proportional and $x = 10$ when $y =
        6$, what is $x$ when $y = 4$?
        \begin{solution}
            $xy$ is constant because $x$ and $y$ are inversely proportional. $xy
            = 6 \cdot 10 = 60$, so when $y = 4$, we have $xy = 4x = 60$ and $x =
            15$.
        \end{solution}
        \item If $2$ liters of a $20\%$ acid solution are mixed with $8$ liters
        of a $50\%$ acid solution, what is the concentration of the resulting
        solution?
        \begin{solution}
            The concentration of an acid solution is the amount of acid in the
            solution divided by the total volume of the solution. Two liters of
            $20\%$ acid solution contains $2 \cdot 0.2 = 0.4$ liters of acid,
            while the other solution contains $8 \cdot 0.5 = 4$ liters of acid.
            The concentration of the solution is thus $\frac{0.4 + 4}{2 + 8} =
            \frac{4.4}{10} = 44\%$.
        \end{solution}
    \end{enumerate}
\end{document}