\documentclass{article}
\title{Fractions, Ratios, and Proportional Reasoning Set B}
\author{}
\date{}
\begin{document}
    \maketitle
    \noindent Problems should be solved without calculators unless otherwise
    specified. Remember to explain how you solved a problem.
    \begin{enumerate}
        \item A MATHCOUNTS club has a total of $27$ sixth- and seventh-grade
        members. If there are $25\%$ more seventh graders than sixth graders,
        how many sixth graders are in the club?
        \vspace{3cm}
        \item Ms. A owns a house worth $\$10000$. She sells it to Mr. B at
        $10\%$ profit. Mr. B sells the house back to Ms. A at a $10\%$ loss. How
        much money does Ms. A make?
        \vspace{3cm}
        \item Before a plum is dried to become a prune, it is $92\%$ water. A
        prune is just $20\%$ water. If only water is evaporated in the drying
        process, how many pounds of prunes can be made with $100$ pounds of
        plums?
        \vspace{3cm}
        \item Given $\frac{x}{y} = \frac{2}{3}$ and $\frac{y}{z} = \frac{3}{2}$,
        find $\frac{x}{z}$.
        \vspace{3cm}
        \item New packaging for fruit snacks contains $10\%$ less weight than
        the original packaging. If the new package costs $15\%$ more than the
        original package, by what fraction did the unit price increase? Express
        your answer as a common fraction.
        \vspace{3cm}
    \end{enumerate}
\end{document}