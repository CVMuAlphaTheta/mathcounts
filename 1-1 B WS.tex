% cSpell:words Miley, Julias, Chloes, Aiden, Daniela, Barrys
\documentclass{article}
\usepackage{siunitx}
\usepackage{amsthm}
\usepackage{amsmath}
\DeclareSIUnit{\mile}{mi}
\sisetup{per-mode=symbol}
\theoremstyle{definition}
\newtheorem*{solution}{Solution}
\title{1.1 Set B Solutions}
\author{}
\date{}
\begin{document}
    \maketitle
    \begin{enumerate}
        \item What is the value of $127^2 - 14 \cdot 127 + 49$? Bonus: Solve
        using only simple mental math without writing anything.
        \begin{solution}
            One way to solve this is to directly evaluate the expression:
            \[\begin{split} 127^2 - 14 \cdot 127 + 49 & = 16129 - 1778 + 49 \\
                & = 14351 + 49 \\
                & = 14400 \end{split}\] However, there's an easier way. Notice
            that the expression resembles a quadratic. We can rewrite it as
            $127^2 - 2 \cdot 7 \cdot 127 + 7^2$, which is the same as $(127 -
            7)^2$. Therefore, we simply have to compute $120^2 = 14400$.
        \end{solution}
        \item Twins Miley and Riley leave home at the same time, riding their
        bicycles in opposite directions. Miley goes northbound traveling
        \SI{18}{\mile\per\hour}. After $1$ hour, the twins are $32$ miles apart.
        How fast was Riley traveling in the southbound direction?
        \begin{solution}
            Miley was going \SI{18}{\mile\per\hour}, so after one hour she is
            $18$ miles from home. Since Miley and Riley are now 32 miles apart,
            Riley must be $32 - 18 = 14$ miles from home. Therefore, Riley's
            speed was \SI{14}{\mile\per\hour}.
        \end{solution}
        \item On a certain standardized test with $50$ problems, $5$ points were
        awarded for each correct answer, and $1$ point was deducted for each
        incorrect answer. Chloe answered all the questions on the test and
        scored a total of $184$ points. How many questions did she answer
        correctly?
        \begin{solution}
            A perfect score on the standardized test would be $50 \cdot 5 = 250$
            points. Chloe earned $250 - 184 = 66$ points less than perfect. Each
            problem that Chloe got wrong cost her $6$ points, since she both
            failed to earn the $5$ points and had $1$ point deducted. Therefore,
            she must have answered $\frac{66}{6} = 11$ questions incorrectly,
            which means she answered $50 - 11 = 39$ questions correctly. This
            checks out since $39 \cdot 5 - 11 = 195 - 11 = 184$.

            To solve this using a system of equations, let $a$ be the number of
            correctly answered questions and let $b$ be the number of
            incorrectly answered questions. $a + b = 50$ and $5a - b = 184$. If
            we add the two equations, we get $a + b + 5a - b = 50 + 184$, so $6a
            = 234$ and $a = 39$.
        \end{solution}
        \item Two Julias and one Chloe can finish a cup of bubble tea together
        in $10$ seconds. One Julia and three Chloes can finish a cup of bubble
        tea together in $8$ seconds. How long will it take for one Julia and one
        Chloe to finish a cup of bubble tea together? Assume that each person
        drinks bubble tea at a constant rate.
        \begin{solution}
            Since two Julias and one Chloe can finish a cup of bubble tea in
            $10$ seconds, they can drink $\frac{1}{10}$ cups of bubble tea every
            second. Similarly, one Julia and three Chloes can drink
            $\frac{1}{8}$ cups of bubble tea every second. If $a$ is the number
            of cups of bubble tea which one Julia can drink in a second and $b$
            is the number of cups of bubble tea which one Chloe can drink in a
            second, $2a + b = \frac{1}{10}$ and $a + 3b = \frac{1}{8}$. If we
            multiply the second equation by $-2$, we get $-2a - 6b =
            -\frac{1}{4}$. Adding this to the first equation results in $2a + b
            - 2a - 6b = \frac{1}{10} - \frac{1}{4}$, so $-5b = -\frac{3}{20}$
            and $b = \frac{3}{100}$. We can substitute this into the second
            equation to find $a$. We have $a + 3 \cdot \frac{3}{100} =
            \frac{1}{8}$, so $a = \frac{7}{200}$. Therefore, one Julia and one
            Chloe can drink $a + b = \frac{3}{100} + \frac{7}{200} =
            \frac{13}{200}$ cups of bubble tea per second, and it will take them
            $\frac{200}{13}$ seconds to finish a cup of bubble tea.
        \end{solution}
        \item Working together, Aiden, Barry, and Cynthia can carry
        \SI{48}{\kilogram}. Barry, Cynthia, and Daniela can carry
        \SI{50}{\kilogram}. Cynthia, Daniela, and Aiden can carry
        \SI{54}{\kilogram}. How many kilograms can two Barrys and one Cynthia
        carry together?
        \begin{solution}
            If $a$, $b$, $c$, and $d$ are the number of kilograms which Aiden,
            Barry, Cynthia, and Daniela can carry respectively, then we know
            that $a + b + c = 48$, $b + c + d = 50$, and $a + c + d = 54$. We
            wish to find $2b + c$. We can try adding or subtracting the
            equations to try to obtain $2b + c$ on the left side. After some
            experimentation, you might find that if we add the first two
            equations and subtract the third one, we get $a + b + c + b + c + d
            - a - c - d = 48 + 50 - 54$, so $2b + c = 44$. Therefore, two Barrys
            and one Cynthia can carry \SI{44}{\kilogram}.

            If you tried to find the values of each of the variables, you might
            have noticed that it this system of equations actually has
            infinitely many solutions. This is because in a system of linear
            equations with $n$ variables, you need at least $n$ equations in
            order to uniquely determine the value of all of the variables. In
            this case, even though there are infinitely many solutions, the
            value of $2b + c$ is the same for every possible solution.
        \end{solution}
    \end{enumerate}
\end{document}
