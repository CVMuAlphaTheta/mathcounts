\documentclass{article}

\usepackage[margin=0.5in]{geometry}
\usepackage{multicol}
\usepackage{amsthm}

\theoremstyle{definition}
\newtheorem*{solution}{Solution}
\title{Modular Arithmetic Set A}
\author{}
\date{}

\begin{document}
\maketitle
\begin{multicols}{2}
    \begin{enumerate}
        \item What is the largest integer less than $100$ which is congruent to $3 \bmod 5$?
            \begin{solution}
                Clearly $103$ is congruent to $3 \bmod 5$, but $103$ is too big.
                If we subtract $5$, however, we get $98$, which is both congruent to $3 \bmod 5$ and less than $100$.
            \end{solution}
        \item How many positive integers less than $100$ are congruent to $3\bmod 5$?
            \begin{solution}
                The smallest is clearly $3$.
                In the previous problem, you should have found that the largest is $98$.
                How many are there in between? 
                We have $3 = 0 \cdot 5 + 3$ and $98 = 19 \cdot 5 + 3$, and the other numbers congruent to $3 \bmod 5 $ will be $1 \cdot 5 + 3$, $2 \cdot 5 + 3$, and so on.
                The number by which $5$ is multiplied can be $0$, $1$, $2$, ..., $19$, so there are $20$ possibilities.
            \end{solution}
        \item How many integers are there between $50$ and $250$ inclusive which are congruent to $1\bmod 7$?
            \begin{solution}
                We first find the smallest and largest numbers congruent to $1 \bmod 7$.
                To find the smallest, we just take a multiple of $7$ which is close and well-known, like $49$.
                The corresponding number which is congruent to $1 \bmod 7 $ is $49 + 1 = 50$.
                (Or we could have started with $71$ and subtracted: $71$, $64$, $57$, $50$.) 
                We can find the largest in a similar way.
                Since $280$ is $4$ times $70$, it is a multiple of $7$, and $281$ is thus congruent to $1 \bmod 7$.
                Subtracting sevens, we get $274$, $267$, $260$, $253$, $246$.
                The smallest is $50 = 7 \cdot 7 + 1 = 29$ and the largest is $246 = 7 \cdot 75 + 1$.
                Thus the total number is $35 - 7 + 1 = 29$.
            \end{solution}
        \item Find the smallest positive integer which $123$ is congruent to $\bmod \; 4$.
            \begin{solution}
                We could try to subtract out all the $4$'s: $123$, $119$, $115$...
                However, the easiest way is to get rid of all the $4$'s at once, dividing by $4$ and keeping only the remainder.
                Long dividing $123$ by $4$ gives $30$, with remainder $3$.
            \end{solution}
        \item Find the smallest positive integer that $321$ is congruent to $\bmod \; 7$.
            \begin{solution}
                Similarly to problem 4, long dividing by $321$ by $7$ gives $45$, with remainder $6$.
            \end{solution}
    \end{enumerate}
\end{multicols}
\end{document}