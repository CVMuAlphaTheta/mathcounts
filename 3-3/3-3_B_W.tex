\documentclass{article}

\usepackage[margin=0.5in]{geometry}
\usepackage{multicol}
\usepackage{amsmath}

\title{Modular Arithmetic Set B}
\date{}
\author{}

\begin{document}
\maketitle
\noindent Problems should be solved without a calculator unless otherwise specified.
Remember to explain how you solved a problem.
\begin{multicols}{2}
    \raggedcolumns
    \begin{enumerate}
        \item Given that $5x \equiv 6 \pmod 8$, find the smallest value of $x$.
            \vspace{3cm}
        \item There are seven teacups lined up in a row.
            One of the cups is made of pure gold.
            The cups are labeled $A$, $B$, $C$, $D$, $E$, $F$, and $G$.
            If you start counting at $A$ and wind back and forth while counting ($A$, $B$, $C$, $D$, $E$, $F$, $G$, $F$, $E$, $D$, \dots), then the golden cup would be the $1000$th one you count.
            Which cup is made of pure gold?
            \vspace{3cm}
        \item Find the last digit of $7^{100}$.
            \vspace{3cm}
        \item What is the last digit of $(\dots(((7)^7)^7)\dots)^7$ if there are $1000$ $7$'s as exponents and only one $7$ in the middle?
            \vspace{3cm}
        \item What is the remainder when $(1! + 2! + 3! + 4! + 5! + 6! + \dots)$ is divided by $9$?
            \vspace{3cm}
    \end{enumerate}
\end{multicols}
\end{document}