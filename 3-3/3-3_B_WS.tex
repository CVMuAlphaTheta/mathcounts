\documentclass{article}

\usepackage[margin=0.5in]{geometry}
\usepackage{multicol}
\usepackage{amsmath}
\usepackage{amsthm}

\theoremstyle{definition}
\newtheorem*{solution}{Solution}

\title{Modular Arithmetic Set B}
\date{}
\author{}

\begin{document}
\maketitle
\begin{multicols}{2}
    \raggedcolumns
    \begin{enumerate}
        \item Given that $5x \equiv 6 \pmod{8}$, find the smallest nonnegative value of $x$.
            \begin{solution}
                The least common multiple of $5$ and $6$ is $30$, so $x = 6$.
            \end{solution}
        \item There are seven teacups lined up in a row.
            One of the cups is made of pure gold.
            The cups are labeled $A$, $B$, $C$, $D$, $E$, $F$, and $G$.
            If you start counting at $A$ and wind back and forth while counting ($A$, $B$, $C$, $D$, $E$, $F$, $G$, $F$, $E$, $D$, \dots), then the golden cup would be the $1000$th one you count.
            Which cup is made of pure gold?
            \begin{solution}
                The first $12$ steps are $A$, $B$, $C$, $D$, $E$, $F$, $G$, $F$, $E$, $D$, $C$, $B$ and then it repeats itself again.
                So, we just need to see how many sets of $12$ steps there are, and then find the remainder number of steps.
                Since $1000 \equiv 4 \pmod{12}$, you should pick cup $D$.
            \end{solution}
        \item Find the last digit of $7^{100}$.
            \begin{solution}
                $7^{100} \equiv (7^2)^{50} \equiv 49^{50} \equiv (-1)^{50} \equiv 1 \bmod 10$
            \end{solution}
        \item What is the last digit of $(\cdots(((7)^7)^7)\cdots)^7$ if there are $1000$ $7$'s as exponents and only one $7$ in the middle?
            \begin{solution}
                We can solve this problem using mods.
                This can also be stated as ${7^7}^{1000}$.
                After that, we see that $7$ is congruent to $-1$ in $\bmod \; 4$, so we can use this fact to replace the $7$'s with $-1$'s, because $7$ has a pattern of repetitive period $4$ for the units digit.
                $(-1)^{1000}$ is simply $1$, so therefore $7^1 = 7$, which really is the last digit.
            \end{solution}
        \item What is the remainder when $(1! + 2! + 3! + 4! + 5! + 6! + \cdots)$ is divided by $9$?
            \begin{solution}
                First of all, we know that $k! \equiv 0 \pmod{9}$ for all $k \geq 6$.
                Thus, we only need to find $(1! + 2! + 3! + 4! + 5!) \pmod{9}$.
				\begin{align*}
					1! &\equiv 1 &&\pmod{9} \\
					2! &\equiv 2 &&\pmod{9} \\
					3! &\equiv 6 &&\pmod{9} \\
					4! &\equiv 24 \equiv 6 &&\pmod{9} \\
					5! &\equiv 5 \cdot 6 \equiv 30 \equiv 3 &&\pmod{9}
				\end{align*}
                Thus, $(1! + 2! + 3! + 4! + 5!) \equiv 1 + 2 + 6 + 6 + 3 \equiv 18 \equiv 0 \pmod{9}$ so the remainder is 0.
            \end{solution}
    \end{enumerate}
\end{multicols}
\end{document}
