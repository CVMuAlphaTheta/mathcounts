\documentclass{article}

\usepackage[group-separator={,}]{siunitx}
\usepackage{amsthm}

\theoremstyle{definition}
\newtheorem*{solution}{Solution}

\title{Counting Set B}
\author{}
\date{}

\begin{document}
\maketitle
\begin{enumerate}
    \item A 24-hour digital clock displays times from 00:00 to 23:59. How 
        many of the times displayed on this clock contain the digit 2?
        \begin{solution}
            It's easier to count the number of times displayed which don't have 
            the digit 2, because this way the possible values of one digit 
            doesn't depend on whether the other digits contain a 2. By listing 
            all the $24$ hours, we see that $18$ of them don't contain the digit 
            $2$. There are always $5$ possible values for the tens digit of the 
            minute (0, 1, 3, 4, 5) and $9$ possible values for the units digit 
            of the minute (0, 1, 3, 4, 5, 6, 7, 8, 9), so if we multiply these 
            values together we get $18 \cdot 5 \cdot 9 = 810$ as the number of 
            times which don't have the digit 2. There are $24 \cdot 60 = 1440$ 
            minutes in a day, so $1440 - 810 = 630$ times displayed by the clock 
            contains the digit 2.
        \end{solution}
    \item Jeremy owns a rectangular plot of land that is 60 yards long and 
        30 yards wide.
        \begin{enumerate}
            \item If he places a fence post at each corner and posts are placed 
                every three yards, how many fence posts will there be on each 
                side?
                \begin{solution}
                    The fence posts divide the sides into three yard sections, 
                    so there are $\frac{60}{3} = 20$ sections on the $60$-yard 
                    sides. There's a fence post at the beginning of each section 
                    plus one more at the very end, so there are $21$ fence posts 
                    on the $60$-yard sides. Similarly, there are $\frac{30}{3} + 
                    1 = 11$ fence posts on the $30$-yard sides.
                \end{solution}
            \item If he places a fence post at each corner and posts are placed 
                every three yards, how many fence posts will he need to enclose 
                his land?
                \begin{solution}
                    If two of the sides have $11$ posts, and the other two sides 
                    have $21$, it would appear that there are $64$ posts, but we 
                    counted all four corners twice. We must subtract $4$ to get 
                    a total of $60$ posts.

                    Alternatively, we can compute the perimeter of the fence, 
                    which is $2(60 + 30) = 180$. The fence posts divide this 
                    perimeter into three yard sections, and since the fence is a 
                    cycle, each section corresponds to one fence post.  
                    Therefore, there are $\frac{180}{3}$ fence posts.
                \end{solution}
        \end{enumerate}
    \item How many distinct pairs of consecutive integers have a product less 
        than \num{10000}?
        \begin{solution}
            Remember to include consecutive negative integers. We can start with 
            $-100 \cdot -99$ and get all the way up to $100 \cdot 99$. The first 
            integer in each pair can be anywhere from $-100$ to $99$, inclusive.  
            Therefore, there are $99 - (-100) + 1 = 200$ pairs of consecutive 
            integers whose product is less than \num{10000}.
        \end{solution}
    \item Tom randomly selects three different numbers from the set $\{1, 2, 3, 
        4\}$. Jerry randomly selects one number from the set $\{2, 4, 6, 8 ,10, 
        12\}$. What is the probability that Jerry's number is greater than the 
        sum of Tom's numbers? Express your answer as a common fraction.
        \begin{solution}
            Since Tom is selecting three different numbers from a set of four 
            numbers, there are $4$ ways to choose which number will no be 
            selected. The possible sums of the three numbers Tom selects are $1 
            + 2 + 3 = 6$, $1 + 2 + 4 = 7$, $1 + 3 + 4 = 8$, and $2 + 3 + 4 = 9$.  
            Tom has a $\frac{2}{4} = \frac{1}{2}$ chance of randomly selecting 
            three numbers with a sum of $6$ or $7$. If Tom's randomly selected 
            numbers do have a sum of $6$ or $7$, then Jerry has a $\frac{3}{6} = 
            \frac{1}{2}$ chance of selecting one of the three numbers $(8, 10, 
            12)$ in his set that are greater than $6$ or $7$.  Similarly, Tom 
            has a $\frac{2}{4} = \frac{1}{2}$ chance of randomly selecting three 
            numbers with a sum of $8$ or $9$. If tom's randomly selected numbers 
            do have a sum of $8$ or $9$, then Jerry has a $\frac{2}{6} = 
            \frac{1}{3}$ chance of selecting one of the two numbers $(10, 12)$ 
            in his set that are greater than $8$ or $9$. The probability, then, 
            that Jerry's randomly selected number is greater than the sum of 
            Tom's three randomly selected numbers is $\frac{1}{2} \cdot 
            \frac{1}{2} + \frac{1}{2} \cdot \frac{1}{3} = \frac{1}{4} + 
            \frac{1}{6} = \frac{6}{24} + \frac{4}{24} = \frac{10}{24} = 
            \frac{5}{12}$.
        \end{solution}
    \item All squares are both rectangles and rhombuses. All rhombuses and 
        rectangles are parallelograms. On a sheet of paper, Josh draws $19$ 
        rectangles, $15$ rhombuses, and $7$ squares. How many parallelograms did 
        Josh draw?
        \begin{solution}
            Draw a Venn diagram. The area in a diagram where rectangles and 
            rhombuses overlap represents squares. There are $7$ squares, $8$ 
            non-rectangle rhombuses, and $12$ non-rhombus rectangles. That sums 
            up to $27$ parallelograms.

            We can also just use the inclusion-exclusion principle. There are 
            $19$ rectangles, $15$ rhombuses, and $7$ shapes which are both 
            rectangles and rhombuses. Therefore there are $19 + 15 - 7 = 27$ 
            shapes in total.
        \end{solution}
\end{enumerate}
\end{document}
