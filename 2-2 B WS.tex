\documentclass{article}
\usepackage{amsthm}

\theoremstyle{definition}
\newtheorem*{solution}{Solution}

\title{Counting Set B}
\author{}
\date{}

\begin{document}
    \maketitle
    \noindent Problems should be solved without calculators unless otherwise specified.
    Remember to explain how you solved a problem.
    \begin{enumerate}
        \item A 24-hour digital clock displays times from 00:00 to 23:59. How many of the times
        displayed on this clock contain the digit 2?
        \begin{solution}
            The fraction of displays that do not contain the digit 2 in either units or tens
            place of the hour is $\frac{18}{24} = \frac{3}{4}$. The fraction of displays that do
            not contain the digit 2 in the tens of place of the minute is $\frac{5}{6}$, and the
            fraction of displays that do not contain the digit 2 in the units place of the minute
            is  $\frac{9}{10}$. The fraction of displays that do not contain the digit 2, then, is
            $\frac{3}{4} \cdot \frac{5}{6} \cdot \frac{9}{10} = \frac{9}{16}$. Therefore, the
            fraction of displays that do not contain the digit 2 is $1 - \frac{9}{16} = \frac{7}{16}$.
            The total number of displays at 60 displays per hour for 24 hours is $60 \cdot 24 = 1440$ displays.
            So, the number of displays that contain the digit 2 is $\frac{7}{16} \cdot 1440 = 630$.
        \end{solution}
        \item Jeremy owns a rectangular plot of land that is 60 yards long and 30 yards wide.
        \begin{enumerate}
            \item If he places a fence post at each corner and posts are placed every three yards,
            how many fence posts will there be on each side?
            \begin{solution}
                Consider the gaps between the posts. To go 60 yards requires twenty gaps. The first
                post is placed, and then a post is placed every 3 yards for a total of 21 posts.
                30 yards requires 10 gaps, or 11 posts.
            \end{solution}
            \item If he places a fence post at each corner and posts are placed every three yards,
            how many fence posts will he need to enclose his land?
            \begin{solution}
                If two of the sides have 11 posts, and the other two sides have 21, it would appear
                that there are 64 posts, but we counted all four corners twice. We must subtract four
                to get a total of 60 posts.
            \end{solution}
        \end{enumerate}
        \item How many distinct pairs of consecutive integers have a product less than 10,000?
        \begin{solution}
            Remember to include consecutive negative integers. We can start with $-100 \cdot -99$ and get
            all the way up to $100 \cdot 99$. The first integer in each pair can be anywhere from $-100$ to $99$,
            inclusive. $99 - (-100) + 1 = 200$ pairs of consecutive integers whose product is less than 10,000.
        \end{solution}
        \item Tom randomly selects three different numbers from the set $\{1, 2, 3, 4\}$.
        Jerry randomly selects one number from the set $\{2, 4, 6, 8 ,10, 12\}$. What is the
        probability that Jerry's number is greater than the sum of Tom's numbers? Express your
        answer as a common fraction.
        \begin{solution}
            Since Tom is selecting three different numbers from a set of four numbers, there are 4 ways
            to choose which number will no be selected. The possible sums of the three numbers Tom selects
            are $1 + 2 + 3 = 6$, $1 + 2 + 4 = 7$, $1 + 3 + 4 = 8$, and $2 + 3 + 4 = 9$. Tom has a
            $\frac{2}{4} = \frac{1}{2}$ chance of randomly selecting three numbers with a sum of 6 or 7.
            If Tom's randomly selected numbers do have a sum of 6 or 7, then Jerry has a $\frac{3}{6} = \frac{1}{2}$
            chance of selecting one of the three numbers $(8, 10, 12)$ in his set that are greater than 6 or 7.
            Similarly, Tom has a $\frac{2}{4} = \frac{1}{2}$ chance of randomly selecting three numbers with a sum
            of 8 or 9. If tom's randomly selected numbers do have a sum of 8 or 9, then Jerry has a $\frac{2}{6} = \frac{1}{3}$
            chance of selecting one of the two numbers $(10, 12)$ in his set that are greater than 8 or 9. The probability, then,
            that Jerry's randomly selected number is greater than the sum of Tom's three randomly selected numbers
            is $\frac{1}{2} \cdot \frac{1}{2} + \frac{1}{2} \cdot \frac{1}{3} = \frac{1}{4} + \frac{1}{6} = \frac{6}{24} + \frac{4}{24}
            = \frac{10}{24} = \frac{5}{12}$.
        \end{solution}
        \item All squares are both rectangles and rhombuses. All rhombuses and rectangles are
        parallelograms. On a sheet of paper, Josh draws 19 rectangles, 15 rhombuses, and 7 squares.
        How many parallelograms did Josh draw?
        \begin{solution}
                Draw a venn diagram. The area in a diagram where rectangles and rhombuses overlap represents squares.
                There are 7 squares, 8 non-rectangle rhombuses, and 12 non-rhombus rectangles. That sums up to 27 parallelograms.
        \end{solution}
    \end{enumerate}
\end{document}