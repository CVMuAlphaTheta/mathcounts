\documentclass{article}

\usepackage[margin=0.5in]{geometry}
\usepackage{multicol}
\usepackage{siunitx}
\usepackage{tkz-euclide}
\usepackage{amsthm}
\usepackage{csquotes}

\theoremstyle{definition}
\newtheorem*{solution}{Solution}

\title{Problem-Solving Set A}
\author{}
\date{}

\begin{document}
\maketitle
\noindent Problems should be solved without a calculator unless otherwise specified.
Remember to explain how you solved a problem.
\begin{multicols}{2}
    \begin{enumerate}
        \item A rectangular garden $60$ feet long and $20$ feet wide is enclosed by a fence.
            To make the garden larger, while using the same fence, its shape is changed to a square.
            By how many square feet does this enlarge garden?
            \begin{solution}
                We need the same perimeter as a $60$ by $20$ rectangle, but the greatest area we can get.
                Right now the perimeter is $160$.
                To get the greatest area while keeping a perimeter of $160$, the sides should all by $40$.
                That means an area of $1600$.
                Right now, the area is $20$ times $60$ which is $1200$.
                $1600 - 1200 = 400$.
            \end{solution}
        \item Bo, Coe, Flo, Joe, and Moe have different amounts of money.
            Neither Joe nor Bo has as much money as Flo.
            Both Bo and Coe have more than Moe.
            Joe has more than Moe, but less than Bo.
            Who has the least amount of money?
            \begin{solution}
                Use logic to solve this problem.
                You don't actually need to use any equations.
                Neither Jo nor Bo has as much money as Flo, so Flo clearly does not have the least amount of money.
                Both Bo and Coe having more money than Moe rules out Bo and Coe.
                Jo has more than Mo, so he's ruled out.
                The only person who has not been ruled out is Moe.
            \end{solution}
        \item The third exit on a highway is located at milepost $40$ and the tenth exit is at milepost $160$.
            There is service center on the highway located three-fourths of the way from the third exit to the tenth exit.
            At what milepost would you expect to find this service center?
            \begin{solution}
                There are $160 - 40 = 120$ miles between the third and tenth exits, so the service center is at milepost $40 + \frac{3}{4} (120) = 40 + 90 = 130$.
            \end{solution}
        \item Each of the five numbers \numlist{1;4;7;10;13} is placed in one of the five squares so that the sum of the three numbers in the horizontal row equals the sum of the three numbers in the vertical column.
            What is the largest possible value for the horizontal of vertical sum?
            \begin{center}
                \begin{tikzpicture}
                    \tkzDefPoint(0,1){A}
                    \tkzDefPoint(3,1){B}
                    \tkzDefPoint(0,2){C}
                    \tkzDefPoint(3,2){D}

                    \tkzDefPoint(1,0){E}
                    \tkzDefPoint(1,3){F}
                    \tkzDefPoint(2,0){G}
                    \tkzDefPoint(2,3){H}
                    
                    \tkzDrawSegments(A,B C,D E,F G,H A,C B,D E,G F,H)
                \end{tikzpicture}
            \end{center}
            \begin{solution}
                The largest sum occurs when $13$ is placed in the center.
                The sum is $13 + 10 + 1 = 13 + 7 + 4 = 24$.
                Note: Two other common sums, $18$ and $21$, are also possible.
            \end{solution}
        \item Tori's mathematics test had $75$ problems: $10$ arithmetic, $30$ algebra, and $35$ geometry problems.
            Although she answered $70\%$ of the arithmetic, $40\%$ of the algebra, and $60\%$ of the geometry problems correctly, she did not pass the test because she got less than $60\%$ of the problems right.
            How many more problems would she have needed to answer correctly to earn a $60\%$ passing grade?
            \begin{solution}
                First, calculate how many of each type of problem she got right:
                \begin{displayquote}
                    Arithmetic: $70\% \cdot 10 = 0.70 \cdot 10 = 7$ \\
                    Algebra: $40\% \cdot 30 = 0.40 \cdot 30 = 12$ \\
                    Geometry: $60\% \cdot 35 = 0.60 \cdot 35 = 21$
                \end{displayquote}
                Altogether, Tori answered $7 + 12 + 21 = 40$ questions correct.
                To get $60\%$ on her test overall, she needed to get $60\% \cdot 75 = 0.60 \cdot 75 = 45$ questions right.
                Therefore, she needed to answer $45 - 40 = 5$ more questions to pass.
            \end{solution}
    \end{enumerate}
\end{multicols}
\end{document}
