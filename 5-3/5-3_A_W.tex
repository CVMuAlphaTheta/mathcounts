\documentclass{article}

\usepackage[margin=0.5in]{geometry}
\usepackage{multicol}
\usepackage{siunitx}
\usepackage{tkz-euclide}

\title{Problem-Solving Set A}
\author{}
\date{}

\begin{document}
\maketitle
\noindent Problems should be solved without a calculator unless otherwise specified.
Remember to explain how you solved a problem.
\begin{multicols}{2}
    \begin{enumerate}
        \item A rectangular garden $60$ feet long and $20$ feet wide is enclosed by a fence.
            To make the garden larger, while using the same fence, its shape is changed to a square.
            By how many square feet does this enlarge garden?
            \vspace{3cm}
        \item Bo, Coe, Flo, Joe, and Moe have different amounts of money.
            Neither Joe nor Bo has as much money as Flo.
            Both Bo and Coe have more than Moe.
            Joe has more than Moe, but less than Bo.
            Who has the least amount of money?
            \vspace{3cm}
        \item The third exit on a highway is located at milepost $40$ and the tenth exit is at milepost $160$.
            There is service center on the highway located three-fourths of the way from the third exit to the tenth exit.
            At what milepost would you expect to find this service center?
            \vspace{3cm}
        \item Each of the five numbers \numlist{1;4;7;10;13} is placed in one of the five squares so that the sum of the three numbers in the horizontal row equals the sum of the three numbers in the vertical column.
            What is the largest possible value for the horizontal of vertical sum?
            \begin{center}
                \begin{tikzpicture}
                    \tkzDefPoint(0,1){A}
                    \tkzDefPoint(3,1){B}
                    \tkzDefPoint(0,2){C}
                    \tkzDefPoint(3,2){D}

                    \tkzDefPoint(1,0){E}
                    \tkzDefPoint(1,3){F}
                    \tkzDefPoint(2,0){G}
                    \tkzDefPoint(2,3){H}
                    
                    \tkzDrawSegments(A,B C,D E,F G,H A,C B,D E,G F,H)
                \end{tikzpicture}
            \end{center}
            \vspace{3cm}
        \item Tori's mathematics test had $75$ problems: $10$ arithmetic, $30$ algebra, and $35$ geometry problems.
            Although she answered $70\%$ of the arithmetic, $40\%$ of the algebra, and $60\%$ of the geometry problems correctly, she did not pass the test because she got less than $60\%$ of the problems right.
            How many more problems would she have needed to answer correctly to earn a $60\%$ passing grade?
            \vspace{3cm}
    \end{enumerate}
\end{multicols}
\end{document}
