\documentclass{article}
\usepackage[inline]{enumitem}
\usepackage{amsthm}
\theoremstyle{definition}
\newtheorem*{solution}{Solution}
\title{Algebraic Expressions and Equations Set B}
\author{}
\date{}
\begin{document}
    \maketitle
    \noindent Problems should be solved without calculators unless otherwise
    specified. Remember to explain how you solved a problem.
    \begin{enumerate}
        \item Donny has two part-time jobs. He earns $\$12$ per hour tutoring,
        and he earns $\$20$ per hour baby-sitting. Donny would like to work a
        combination of hours at both jobs to earn exactly $\$360$. If he works
        at least one hour at each job, with no partial hours worked at either
        job, what is the difference between the maximum and minimum total hours
        that Donny can work to earn that amount?
        \begin{solution}
            Let's say that Donny works $x$ hours at $\$12$ per hour and $y$
            hours at $\$20$ per hour. Then we are looking for integer values of
            $x$ and $y$ such that $12x + 20y = 360$, which can be simplified to
            $3x + 5y = 90$ by dividing both sides by $4$. If Donny wanted to
            maximize the total number of hours, he should try to work more at
            the job that pays less since the total amount that he earns is
            constant. He needs to work at least one hour both jobs, so $x \geq
            1$ and $y \geq 1$. We want to maximize $x$ and minimize $y$, so we
            can first try setting $y$ to $1$. The problem is that we end up with
            $3x = 85$, which doesn't work because $x$ must be an integer. We can
            then try higher values of $y$, until we find that $y = 3$ results in
            $3x + 75$ so $x = 25$. Therefore, the maximum number of hours which
            Donny can work is $25 + 3 = 28$. To find the minimum number of hours
            which Donny can work, we need to minimize $x$ and maximize $y$. If
            you tried out different values of $x$, you would find that the
            minimum value of $x$ that makes $y$ an integer is $5$. If $x = 5$,
            then $y = 15$, so the total number of hours is $20$. The difference
            between $28$ and $20$ is $8$, which is the difference between the
            maximum and the minimum number of hours which Donny can work.
        \end{solution}
        \item A rectangular garden $60$ feet long and $20$ feet wide is enclosed
        by a fence. To make the garden larger, while using the same fence, its
        shape is changed to a square. By how many square feet does this enlarge
        the garden?
        \begin{solution}
            The perimeter of the rectangular garden is $2(60 + 20) = 160$ feet.
            A square with this perimeter has side length $\frac{160}{4} = 40$
            feet. The area of the rectangular garden is $60 \cdot 20 = 1200$
            square feet and the area of the square garden is $40^2 = 1600$
            square feet. Therefore the area increased by $1600 - 1200 = 400$
            square feet.
        \end{solution}
        \item The average (mean) age of the $40$ members of a computer science
        camp is $17$ years. There are $20$ girls, $15$ boys, and $5$ adults. If
        the average age of the girls is $15$ and the average age of the boys is
        $16$, what is the average age of the adults?
        \begin{solution}
            If $a$ is the average age, $s$ is the sum of the ages, and $n$ is
            the number of people, then $a = \frac{s}{n}$. This means that $an =
            s$, so the sum of the ages is the number of people multiplied by the
            average age. Since the average is $17$ and the number of people is
            $40$, the sum of the ages is $17 \cdot 40 = 680$. Similarly, the sum
            of the ages of the girls is $20 \cdot 15 = 300$ and the sum of the
            ages of the boys is $15 \cdot 16 = 240$. We subtract $240$ and $300$
            from $680$ to find the sum of the ages of the adults, which is
            $140$. Therefore, the average age of the adults is $\frac{140}{5} =
            28$.
        \end{solution}
        \item On a twenty-question test, each correct answer is worth $5$
        points, each unanswered question is worth $1$ point and each incorrect
        answer is worth $0$ points. Which of the following scores is NOT
        possible?
        \begin{enumerate*}
            \item $90$ \item $91$ \item $92$ \item $95$ \item $97$
        \end{enumerate*}
        \begin{solution}
            If you got all the questions right, you would get the highest score
            which is $20 \cdot 5 = 100$. The second highest score possible for a
            single question is $1$, which is $4$ less than the highest possible
            score for a single question. Therefore, if you didn't get all the
            questions right, you lost at least $4$ points, so a score of $97$ is
            not possible.
        \end{solution}
        \item Find two numbers which differ by $2$ such that $\frac{1}{3}$ of
        the smaller number plus twice the larger number equals $7$ more than the
        sum of the two numbers.
        \begin{solution}
            Let the smaller number be $x$, so the larger number is $x + 2$. We
            have $\frac{x}{3} + 2(x + 2) = x + x + 2 + 7$. Expanding the
            multiplication on the left side and simplifying the right side
            results in $\frac{x}{3} + 2x + 4 = 2x + 9$. Subtracting $2x + 4$
            from both sides gives $\frac{x}{3} = 5$, so $x = 15$. Therefore, the
            numbers are $15$ and $17$.
        \end{solution}
    \end{enumerate}
\end{document}