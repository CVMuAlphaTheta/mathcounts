\documentclass{article}
\usepackage{amsthm}
\theoremstyle{definition}
\newtheorem*{solution}{Solution}
\title{Algebraic Expressions and Equations Set A}
\author{}
\date{}
\begin{document}
    \maketitle
    \noindent Problems should be solved without calculators unless otherwise
    specified. Remember to explain how you solved a problem.
    \begin{enumerate}
        \item In a far-off land, three fish can be traded for two loaves of
        bread and a loaf of bread can be traded for four bags of rice. How many
        bags of rice is one fish worth?
        \begin{solution}
            Let $f$ be the value of one fish, $l$ be the value of a loaf of
            bread, and $r$ be the value of a bag of rice. Then $3f = 2l$, and $l
            = 4r$. Substituting $l$ from the second equation in to the first
            equation gives $3f = 2 \cdot 4r$, so $3f = 8r$. This means that $f =
            \frac{8}{3}$, so each fish is worth $\frac{8}{3}$ bags of rice.
        \end{solution}
        \item The number $66$ is written as the sum of two smaller numbers. One
        of these numbers is $3$ more than twice the other number. Find the
        larger of the two numbers.
        \begin{solution}
            If one of the numbers is $x$, then the other is $2x + 3$. So $x + 2x
            + 3 = 3x + 3 = 66$. Therefore, $3x = 63$ and $x = 21$. This is the
            smaller number and the larger one is $2 \cdot 21 + 3 = 45$.
        \end{solution}
        \item Cedric began peeling a pile of $44$ potatoes at a rate of $3$
        potatoes per minute. Four minutes later Daniel joined him and peeled at
        a rate of $5$ potatoes per minute. Whey they finished, how many potatoes
        had Daniel peeled?
        \begin{solution}
            After four minutes of Cedric peeling alone, he has peeled $4 \cdot 3
            = 12$ potatoes. This means that there are $44 - 12 = 32$ potatoes
            left. Once Daniel joins him, the two are peeling potatoes at a rate
            of $3 + 5 = 8$ potatoes per minute. Therefore they finish peeling
            After $\frac{32}{8} = 4$ minutes. In these $4$ minutes, Daniel
            peeled $4 \cdot 5 = 20$ potatoes.
        \end{solution}
        \item \begin{enumerate}
            \item The sum of two numbers is $7$ and their product is $12$. What
            is twice the sum of their reciprocals? Express your answer as a
            common fraction.
            \begin{solution}
                We can simply try to factor $12$ into two numbers which sum to
                $7$. $12$ can be factored into $1 \cdot 12$, $2 \cdot 6$, and $3
                \cdot 4$. The last pair of factors sum to $7$. We need twice the
                sum of the reciprocals of the two numbers, which is
                $2\left(\frac{1}{3} + \frac{1}{4}\right) = 2 \cdot \frac{1 \cdot
                4 + 1 \cdot 3}{3 \cdot 4} = 2 \cdot \frac{7}{12} = \frac{7}{6}$.

            \end{solution}
            \item The sum of two numbers is $11$ and their product is $13$. What
            is twice the sum of their reciprocals? Express your answer as a
            common fraction.
            \begin{solution}
                If you tried to factor $13$, you would have noticed that none of
                the possible pairs of factors it sum to $11$. This is because
                the two numbers which sum to $11$ and have a product of $13$
                aren't integers, or even rational. It would be very difficult to
                guess the two numbers here. Luckily, we can find the solution
                without finding the two numbers. If the two numbers are $a$ and
                $b$, we need to find $2\left(\frac{1}{a} + \frac{1}{b}\right)$.
                We can use the rule for adding fractions to rewrite this as $2
                \cdot \frac{a + b}{ab}$. We know that $a + b = 11$ and $ab =
                13$, so we can simply substitute these values into the
                expression to get $2 \cdot \frac{11}{13} = \frac{22}{13}$.

                If we needed to find the value of $a$ and $b$, we can use
                quadratics. The first equation can be rewritten as $b = 11 - a$,
                and substituting this into the second equation results in $a(11
                - a) = 13$. This equation can't be easily solved using the
                methods for linear equations that you're familiar with, but we
                might teach you how to solve this in the future. The solutions
                to this equation are $\frac{11}{2} - \frac{\sqrt{69}}{2}$ and
                $\frac{11}{2} + \frac{\sqrt{69}}{2}$, which are the two numbers
                that satisfy the properties stated in the problem.
            \end{solution}
        \end{enumerate}
        \item Cedric and Julia are having a violin practicing battle. They
        practice $5$ hours on the first day, $6$ hours on the second day, $7$
        hours on the third day, and so on. Let $x$ be the number of hours they
        practiced on the last day. Express the number of days in the battle in
        terms of $x$. In other words, find an expression containing $x$ such
        that the value of the expression is the number of days in the battle.
        \begin{solution}
            Cedric and Julia practice $n + 4$ hours on the $n$th day. If $d$ is
            the number of days, then the last day is the $d$th day so they
            practiced $d + 4$ hours on that day. We have $d + 4 = x$, so $d = x
            - 4$ meaning that the number of days is $x - 4$.
        \end{solution}
    \end{enumerate}
\end{document}