% cSpell:ignore rightanglemark, waypoint
\documentclass{article}
\usepackage{asymptote}
\usepackage{amsmath}
\usepackage{amsthm}
\theoremstyle{definition}
\newtheorem*{solution}{Solution}
\title{Distribution and Factoring Set B}
\author{}
\date{}
\begin{document}
    \maketitle
    \noindent Problems should be solved without calculators unless otherwise
    specified. Remember to explain how you solved each problem.
    \begin{enumerate}
        \item What is the sum of the distinct prime factors of $156$?
        \begin{solution}
            We can solve this by finding the \textbf{prime factorization} of
            $156$. We can do this by starting with $156$ and repeatedly
            factoring the numbers we have until only primes are left. $156 = 2
            \cdot 78$, $78 = 2 \cdot 39$, and $39 = 3 \cdot 13$, so $156 = 2
            \cdot 2 \cdot 3 \cdot 13$. The sum of the distinct prime factors is
            $2 + 3 + 13 + 18$.
        \end{solution}
        \item What is the sum of two consecutive positive integers whose squares
        differ by $45$?
        \begin{solution}
            If $a$ and $b$ are the two numbers, $b - a = 1$ because they are
            consecutive. $b^2 - a^2 = (b - a)(b + a) = 45$, so $b + a = \frac{(b
            - a)(b + a)}{b - a} = \frac{45}{1} = 45$.
        \end{solution}
        \item This problem will introduce one of the many proofs of the
        \textbf{Pythagorean theorem}. This important theorem establishes a
        relationship between the lengths of the sides of right triangles:
        \begin{center}
            \begin{asy}
                import olympiad; pair a = (0, 0), b = (0, 60), c = (80, 0);
                draw(a -- b -- c -- cycle); draw(rightanglemark(c, a, b, 160));
                label("$a$", a -- b, W); label("$b$", a -- c, S); label("$c$", b
                -- c, NE);
            \end{asy}
        \end{center}
        \[a^2 + b^2 = c^2\] $a$ and $b$ are the lengths of the legs, which are
        the sides of a right triangle that form the right angle. $c$ is the
        length of the hypotenuse, which is the side opposite to the right angle.
        This proof uses a square inscribed inside another square, forming four
        congruent right triangles:
        \begin{center}
            \begin{asy}
                import olympiad; pair a = (-50, -50), b = (-50, 50), c = (50,
                50), d = (50, -50); draw(a -- b -- c -- d -- cycle); pair p =
                waypoint(a -- b, 0.3), q = waypoint(b -- c, 0.3), r = waypoint(c
                -- d, 0.3), s = waypoint(d -- a, 0.3); draw(p -- q -- r -- s --
                cycle); draw(rightanglemark(p, a, s, 160)); label("$a$", a -- p,
                W); label("$b$", a -- s, S); label("$c$", p -- s, NE);
                label("$a$", s -- d, S);
            \end{asy}
        \end{center}
        \begin{enumerate}
            \item Write an expression for the area of the smaller square in
            terms of $c$.
            \begin{solution}
                $c$ is the side length, so the area is $c^2$.
            \end{solution}
            \item Write an expression for the area of the smaller square in
            terms of $a$ and $b$. Hint: find the area of the larger square by
            squaring its side length and then subtract the area of the four
            triangles.
            \begin{solution}
                The side length of the large square is $a + b$, therefore its
                area is $(a + b)^2 = a^2 + 2ab + b^2$. The area of one of the
                triangles is $\frac{ab}{2}$, so the area of the small square is
                $a^2 + 2ab + b^2 - 4 \cdot \frac{ab}{2} = a^2 + 2ab + b^2 - 2ab
                = a^2 + b^2$.
            \end{solution}
            \item Explain why this proves the Pythagorean theorem.
            \begin{solution}
                $c^2$ and $a^2 + b^2$ are both equal to the area of the small
                square, therefore they must be equal to each other and $a^2 +
                b^2 = c^2$.
            \end{solution}
        \end{enumerate}
        \item If $a + b = 1$ and $a^2 + b^2 = 2$, find $a^4 + b^4$.
        \begin{solution}
            If we square the first equation, we get $a^2 + 2ab + b^2 = 1$.
            Subtracting the second equation gives $2ab = -1$, so $ab =
            -\frac{1}{2}$. If we square the second equation, we get $a^4 + 2a^2
            b^2 + b^4 = 4$. Since $ab = -\frac{1}{2}$, $2a^2 b^2 = 2(ab)^2 =
            2\left(-\frac{1}{2}\right)^2 = \frac{1}{2}$. We can subtract this
            value from the square of the second equation to get $a^4 + b^4 = 4 -
            \frac{1}{2} = \frac{7}{2}$.
        \end{solution}
        \item Given that $9876^2 = 97535376$, find $9877^2$ without using
        multiplication.
        \begin{solution}
            \[\begin{split} 9877^2 & = (9876 + 1)^2 \\
                & = 9876^2 + 9876 + 9876 + 1 \\
                & = 97535376 + 19752 + 1 \\
                & = 97555129 \end{split}\] We can also visualize it:
            \begin{center}
                \begin{asy}
                    draw((0, 0) -- (0, 100) -- (100, 100) -- (100, 0) -- cycle);
                    draw((0, 80) -- (100, 80)); draw((80, 0) -- (80, 100));
                    label("$9876$", (0, 40), W); label("$1$", (0, 90), W);
                    label("$9876$", (40, 0), S); label("$1$", (90, 0), S);
                    label("$9876^2$", (40, 40)); label("$9876 \cdot 1$", (40,
                    90)); label(rotate(-90) * Label("$1 \cdot 9876$"), (90,
                    40)); label("$1^2$", (90, 90));
                \end{asy}

                (Not drawn to scale)
            \end{center}
            Writing the area of the whole square as the sum of the areas of its
            rectangular components shows that $9877^2 = 9876^2 + 9876 + 9876 +
            1$.
        \end{solution}
    \end{enumerate}
\end{document}