\documentclass{article}

\usepackage{siunitx}

\title{4.1 A Hints}
\author{}
\date{}

\begin{document}
\maketitle

\begin{enumerate}
	\item Let $\alpha$ be the measure of $\angle A$ and $\beta$ be the sum of the measures of all the other angles.
		Write two equations using the information that you're given and the fact that the sum of the angles of a quadrilateral is always $\ang{360}$.
	\item Vertical angles are always equal, and two inscribed angles which subtend the same arc are always equal.
	\item $\angle CAB$ is an inscribed angle.
		Use the inscribed angle theorem to find its measure.
	\item There's a corollary of the inscribed angle theorem which states that the angle between a chord ($\overline{BC}$) and the tangent line at one of its intersection points ($\overline{AB}$) equals half of the central angle subtended by the chord ($\angle BOC$, where $O$ is the center of the circle).
		Once you find $m\angle BOC$, use the inscribed angle theorem to find $m\angle D$.
	\item Draw lines from $T$ and $A$ to the center of the circle $O$.
		Those two lines are radii of the circle, so they are perpendicular to the tangent lines $\overline{PT}$ and $\overline{PA}$.
		Use the quadrilateral angle sum to find $m\angle TOA$, then use the inscribed angle theorem to find $\angle X$.
\end{enumerate}
\end{document}
