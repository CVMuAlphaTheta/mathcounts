\documentclass{article}
\usepackage{amsmath}
\usepackage{amsthm}
\usepackage{siunitx}
\theoremstyle{definition}
\newtheorem*{solution}{Solution}
\title{1.1 Set A Solutions}
\author{}
\date{}
\begin{document}
    \maketitle
    \begin{enumerate}
        \item What number is $\frac{1}{2}$ of $\frac{1}{4}$ of $\frac{1}{8}$ of
        $60$? Express your answer as a common fraction.
        \begin{solution}
            In mathematics, the word ``of'' generally means multiply, so
            $\frac{1}{2}$ of $\frac{1}{4}$ of $\frac{1}{8}$ of $60$ means
            $\frac{1}{2} \cdot \frac{1}{4} \cdot \frac{1}{8} \cdot 60 =
            \frac{60}{64} = \frac{15}{16}$.
        \end{solution}
        \item If $d = 5 + c(7 + c(4 + c)) - (4 + c(4 + c))$, what is the value
        of $d$ when $c = -1$?
        \begin{solution}
            We substitute $c = -1$ into the expression by replacing every
            instance of $c$ with $-1$. Therefore,
            \[\begin{split} d & = 5 + (-1)(7 + (-1)(4 + (-1))) - (4 + (-1)(4 +
                (-1))) \\
                & = 5 + (-1)(7 + (-1)3) - (4 + (-1)3) \\
                & = 5 + (-1)4 - 1 \\
                & = 1 - 1 \\
                & = 0 \end{split}\]
        \end{solution}
        \item The sum of two numbers is $20$ and their difference is $5$. What
        is the smaller of the two numbers?
        \begin{solution}
            Let $a$ be the greater number and let $b$ be the smaller number. $a
            + b = 20$ and $a - b = 5$, so adding $5$ to the second equation
            results in $a = b + 5$ Substituting this into the first equation
            results in $b + 5 + b = 20$, so $2b + 5 = 20$ and $b =
            \frac{15}{2}$.
        \end{solution}
        \item Bradley scores $34$ points in a basketball game by making $12$
        baskets. If each basket was worth either $2$ or $3$ points, how many
        three-pointers did Bradley make?
        \begin{solution}
            Begin by assuming that Bradley only made two-point shots. Twelve
            baskets would give him $24$ points. We need $10$ more points, and a
            three-point shot gives one more point than a two-point shot, so he
            must have made $10$ three point shots.

            Alternatively, we can also use a system of equations. If $a$ is the
            number of two point shots and $b$ is the number of three point
            shots, we have $a + b = 12$ and $2a + 3b = 34$. Multiplying the
            first equation by $-2$ gives $-2a - 2b = -24$, and adding this with
            the second equation gives $2a + 3b - 2a - 2b = 34 - 24$, so $b =
            10$.
        \end{solution}
        \item The length of a rectangle is three times its width. The area of
        the rectangle is \SI{75}{\cm\squared}. What is its width?
        \begin{solution}
            Since the length of the rectangle is three times its width, then we
            can visualize this as a three squares side by side. The area of the
            rectangle is \SI{75}{\cm\squared} and $75 = 3 \cdot 25$. Therefore
            the width of the rectangle is the same as the side length of a
            square of area \SI{25}{\cm\squared}, which is
            $\sqrt{\SI{25}{\cm\squared}} = \SI{5}{\cm}$.

            We can also solve this using a system of equations. If $a$ is the
            width and $b$ is the length in centimeters, we have $b = 3a$ and $ab
            = 75$. We can substitute the expression for $b$ given in the first
            equation into the second equation to get $a \cdot 3a = 75$. Since
            $3a^2 = 75$, $a^2 = 25$, therefore $a = \sqrt{25} = 5$.
        \end{solution}
    \end{enumerate}
\end{document}
