\documentclass{article}

\usepackage[margin=0.5in]{geometry}
\usepackage{multicol}
\usepackage{amsthm}
\usepackage{amsmath}
\usepackage{siunitx}

\theoremstyle{definition}
\newtheorem*{solution}{Solution}

\title{Triangles Set A}
\date{}
\author{}

\begin{document}
\maketitle

\begin{multicols}{2}
    \begin{enumerate}
        \item What is the area of an isosceles triangle that has a base length of $12$ units and base angle measuring $\ang{30}$?
            Express your answer in simplest radical form.
            \begin{solution}
               If we drop an altitude from the vertex opposite the base, which has length $12$ units, then we create two $30$-$60$-$90$ right triangles.
               Both triangles have a longer leg of length $\frac{12}{2} = 6$ units and a shorter leg of length $\frac{6}{\sqrt{3}} = \frac{6}{\sqrt{3}} \cdot \frac{\sqrt{3}}{\sqrt{3}} = \frac{6\sqrt{3}}{3} = 2\sqrt{3}$ units.
               Since the shorter leg is also the altitude of the original isosceles triangle, it follows that the triangle has area $\frac{1}{2} \cdot 12 \cdot 2\sqrt{3} = 12\sqrt{3}$ units$^{2}$. 
            \end{solution}
            \item If in $\triangle ABC$, $\angle A + \angle B = \ang{90}$, $AC = 4$, and $AB = 5$, what is $BC$?
            \begin{solution}
                Since $\angle A + \angle B = \ang{90}$, we know that $\angle C = \ang{90}$, so we can apply the Pythagorean Theorem: $4^2 + (BC)^2 = 5^2$, so $BC = 3$.
            \end{solution}
        \item A $25$-foot ladder is placed against a vertical wall.
            The foot of the ladder is $7$ feet from the base of the wall.
            If the top of the ladder slips $4$ feet, then how far will the foot slide?
            \begin{solution}
                From the Pythagorean Theorem, the top of the ladder was originally $\sqrt{25^2 - 7^2} = 24$ feet from the base.
                After sliding, it is $20$ feet from the base, so the foot of the ladder is $\sqrt{25^2 - 20^2} = 15$ feet from the base of the wall.
                Thus the foot of the ladder slid $8$ feet.
            \end{solution}
            \item Find the area of $\triangle ABC$ if $AB = AC = 50$ and $BC = 80$
            \begin{solution}
                Since the triangle is isosceles, the altitude $AX$ to side $BC$ bisects $BC$.
                Thus, from the Pythagorean Theorem on the right triangle $ABX$, we find $AX = 30$, so $ABC = \frac{BC \cdot AX}{2} = \frac{80 \cdot 30}{2} = 1200$ squared units.
            \end{solution}
            \item Find the length $AC$ in the regular hexagon $ABCDEF$ if the perimeter of the hexagon is $24$ cm.
            \begin{solution}
                If the perimeter is $24$ cm, the side length is $4$ cm.
                Let's name one angle of the hexagon $\angle ABC$.
                The sum of the angles of polygons with $n$ sides is $180(n - 2)$, thus the sum of the angles in a hexagon is $\ang{720}$.
                Because this is a regular hexagon, we can divide $720$ by the number of angles to find the degree of each interior angle.
                $\angle ABC$ is $\ang{120}$.
                Drawing a line between $A$ and $C$, we've created an isosceles triangle $\triangle ABC$.
                We can conclude that $\angle A$ and $\angle C$ are both $\ang{30}$ using the triangle angle sum theorem.
                By dropping an altitude from vertex $B$ to line $AC$, we create two $30$-$60$-$90$ triangles.
                $AB$ and $BC$ equal $4$, so $BX$ equals $2$, and $AX$ and $CX$ both equal $2\sqrt{3}$.
                The sum of $AX$ and $CX$ equals the length of $AC$, or $4\sqrt{3}$.
            \end{solution}
    \end{enumerate}
\end{multicols}
\end{document}