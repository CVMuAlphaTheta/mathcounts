\documentclass{article}
\usepackage{amsmath}
\usepackage{amsthm}
\theoremstyle{definition}
\newtheorem*{solution}{Solution}
\title{Distribution and Factoring Set A}
\author{}
\date{}
\begin{document}
    \maketitle
    \noindent Problems should be solved without calculators unless otherwise
    specified. Remember to explain how you solved each problem.
    \begin{enumerate}
        \item What is the value of $\left(\frac{1}{2}\right)^{-3}$?
        \begin{solution}
            To compute a negative exponent, we take the reciprocal of the base
            raised to the positive exponent.
            \[\left(\frac{1}{2}\right)^{-3} =
            \frac{1}{\left(\frac{1}{2}\right)^3} = \frac{1}{\frac{1}{8}} = 8\]
            We can also take the reciprocal of the base first and the raise it
            to the positive exponent.
            \[\left(\frac{1}{2}\right)^{-3} = 2^3 = 8\]
        \end{solution}
        \item Show that $a^3 - b^3 = (a - b)(a^2 + ab + b^2)$ and $a^3 + b^3 =
        (a + b)(a^2 - ab + b^2)$.
        \begin{solution}
            \[\begin{split} (a - b)(a^2 + ab + b^2) & = a(a^2 + ab + b^2) -
                b(a^2 + ab + b^2) \\
                & = a^3 + a^2 b + ab^2 - a^2 b - ab^2 - b^3 \\
                & = a^3 - b^3 \end{split}\]
            \[\begin{split} (a + b)(a^2 - ab + b^2) & = a(a^2 - ab + b^2) +
                b(a^2 - ab + b^2) \\
                & = a^3 - a^2 b + ab^2 + a^2 b - ab^2 + b^3 \\
                & = a^3 + b^3 \end{split}\]
        \end{solution}
        \item What is the largest prime factor of $8 \cdot 7 \cdot 6 \cdot 5
        \cdot 4 \cdot 3 - 6 \cdot 5 \cdot 4 \cdot 3 \cdot 2 \cdot 1$?
        \begin{solution}
            We can factor out $6 \cdot 5 \cdot 4 \cdot 3 \cdot 2$ to get $6
            \cdot 5 \cdot 4 \cdot 3 \cdot (4 \cdot 7 - 1) = 6 \cdot 5 \cdot 4
            \cdot 3 \cdot 2 \cdot 27$. If we factor each of the numbers, we get
            $2 \cdot 3 \cdot 5 \cdot 2 \cdot 2 \cdot 3 \cdot 2 \cdot 3 \cdot 3
            \cdot 3$. The largest prime factor is $5$.
        \end{solution}
        \item If $(x^2 y^3)^4 (x^4 y^5)^6 = x^a y^b$ for all real numbers $x$
        and $y$, what is the sum of $a$ and $b$?
        \begin{solution}
            Exponents distribute over multiplication, so $(x^2 y^3)^4 (x^4
            y^5)^6 = (x^2)^4 (y^3)^4 (x^4)^6 (y^5)^6$. Raising a number to two
            exponents is the same as raising it to the product of the two
            exponents, so this equal to $x^8 y^{12} x^{24} y^{30}$. The product
            of a base raised to two different exponents is the same as the base
            raised to the sum of the exponents, so this can be simplified to
            $x^{32} y^{42}$. This must be equal to $x^a y^b$ for all values of
            $x$ and $y$, therefore $a = 32$ and $b = 42$ so $a + b = 74$.
        \end{solution}
        \item This problem is an introduction to solving quadratic equations by
        factoring. \textbf{Quadratic equations} are essentially equations
        containing $x^2$. They are often written in \textbf{standard form},
        which is $ax^2 + bx + c = 0$ where $a$, $b$, and $c$ are constants. For
        example, $3x^2 + 4x - 7 = 0$ is a quadratic equation. Factoring is the
        fastest way to solve a quadratic, but this method can't be used to solve
        all quadratics.
        \begin{enumerate}
            \item Show that $(x + a)(x + b) = x^2 + (a + b)x + ab$.
            \begin{solution}
                \[\begin{split} (x + a)(x + b) & = x^2 + ax + bx + ab \\
                    & = x^2 + (a + b)x + ab \end{split}\]
            \end{solution}
            \item If $(x + a)(x + b) = x^2 + 5x + 6$, find $a$ and $b$. Hint:
            Using the result from part a, we see that
            \[\begin{alignedat}{3} (x + a)(x + b) & = x^2 &{} + (a + b)x &&{} +
                ab &\\
                & = x^2 &{} + 5x &&{} + 6 & \end{alignedat}\] Determine the
            values of $a + b$ and $ab$, then use guess-and-check to find $a$ and
            $b$.
            \begin{solution}
                In order for $(x + a)(x + b) = x^2 + (a + b)x + ab$ to be equal
                to $x^2 + 5x + 6$, $a + b$ must be $5$ and $ab$ must be $6$.
                This means that $a = 2$ and $b = 3$, or $a = 3$ and $b = 2$.
            \end{solution}
            \item To solve $x^2 + 5x + 6 = 0$, we can first \textbf{factor} the
            left side by rewriting it as $(x + a)(x + b)$ for some constants $a$
            and $b$ so that we have $(x + a)(x + b) = 0$. Now that you've found
            $a$ and $b$, what are the solutions to this equation? Note that
            ``solutions'' is plural.
            \begin{solution}
                We know that $x^2 + 5x + 6 = (x + 2)(x + 3)$. In order for $(x +
                2)(x + 3)$ to be equal to $0$, either $x + 2$ or $x + 3$ must be
                $0$. This means that $x = -2$ or $x = -3$.
            \end{solution}
        \end{enumerate}
    \end{enumerate}
\end{document}