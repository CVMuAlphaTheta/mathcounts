\documentclass{article}

\usepackage[margin=0.5in]{geometry}
\usepackage{multicol}
\usepackage{amsthm}
\usepackage{amsmath}

\theoremstyle{definition}
\newtheorem*{solution}{Solution}
\title{General Number Theory Set B}
\date{}
\author{}

\begin{document}
\maketitle
\begin{multicols}{2}
    \begin{enumerate}
        \item Syd doubles the $3$-digit number $TWO$ to get the $4$-digit number $FOUR$.
        Each letter represents a distinct digit, and $O$ represents the digit $7$.
        What is the $3$-digit number $TWO$?
        \begin{solution}
            We are told that $O$ represents the digit $7$.
            Next, since $7 \cdot 2 = 14$, which has units digit $4$,
            it follows that $R$ must represent the digit $4$ and that one $10$ will be carried over from the units place.
            Then, the greatest possible result when a $3$-digit number is doubled is $1998 = 999 \cdot 2$, so $F$ can only represent the digit $1$.
            Now we need to determine which distinct digits are represented by $T$, $W$ and $U$.
            It turns out that $T$ must represent $8$ and there must be one $1000$ carried over from the hundreds place.
            This is confirmed since $2 \cdot 8 + 1 =17$ and $1$ and $7$ are the first to digits of the product.
            The only digits to choose from for $W$ and $U$ are $3$, $5$, $6$, $8$ and $9$.
            Since we need to carry one $100$ over from the tens place, $W$ has to represent a digit greater than or equal to $5$.
            But $2 \cdot 5 +1 = 11$ results in $U$ representing $1$,
            which cannot be true since the digits are all distinct and we've already determined that $F$ represents the digit $1$.
            Likewise, if $W$ represents $7$, $8$ or $9$, we end up with digits that are not distinct.
            So, it must be the case that $W$ represents the digit $6$, in which case $U$ represents the digit $3$, since $6 \cdot 2 +1 = 13$
            Now that we know what digit each letter represents, we can rewrite the equation $TWO \cdot 2 = FOUR$ as $867 \cdot 2 = 1734$.
            Therefore, the $3$-digit number represented by $TWO$ is $867$.
        \end{solution}
        \item How many positive integers in the set of numbers from $1$ to $1000$, inclusive, multiples of $2$, $3$ and $5$ but not $8$?
        \begin{solution}
            Since $2$, $3$ and $5$ are relatively prime, the least common multiple is $2 \cdot 3 \cdot 5 = 30$.
            From $30$ up to $990$ there are $\frac{960}{30} = 32$ multiples of $30$, $33$ multiples if we include $990$.
            These are the only integers that are multiples of $4 \cdot 30 = 120$ that are divisible by $8$.
            There are $8$ of them.
            That leaves $33 - 8 = 25$ integers.
        \end{solution}
        \item If $n$ is the product of three consecutive positive integers and $n = 22 \cdot 14 \cdot k$,
        what is the least possible value of $k$?
        \begin{solution}
            The prime factors of $n$ must include the prime factors of $22$ and $14$, which include $11$ and $7$.
            These two primes are not close enough together to be among three consecutive integers.
            Since we know that $n$ is a multiple of $22$, let's consider $21$ and $22$, which are consecutive and have factors $7$ and $11$.
            We are looking for the least possible value of $k$, a factor of $n$, so let the three consecutive positive integers be $20$, $21$ and $22$.
            Then their product is $20 \cdot 21 \cdot 22 = 9240$, and $k$ would have to be $\frac{9240}{22 \cdot 14} = 30$.
        \end{solution}
        \item Rahul Ilhangovan can arrange his dad's collection of quarters as a rectangular array with $10$ equal rows,
        $12$ equal rows or $18$ equal rows, using all the quarters in each arrangement.
        What is the least possible monetary value in dollars of the quarter collection?
        \begin{solution}
            The least common multiple (LCM) of $10$, $12$ and $18$ is $180$.
            The monetary value of $180$ quarters is $\frac{180}{4} = \$45$.
        \end{solution}
        \item Maxine secretly chooses a positive integer between $1$ and $2018$, inclusive.
        Martin wants to identify her number with a series of guesses.
        Each time Martin makes a guess, Maxine tells him whether his number is correct, too high or too low.
        With an appropriate strategy, Martin can always identify Maxine's number after at most $n$ guesses.
        What is the least value of $n$ for which Martin can correctly guess Maxine's number?
        \begin{solution}
            The best strategy for Martin to follow can be found by looking at the problem backwards
            and asking, ``How wide a range of numbers can Martin handle with $g$ guesses?''
            Suppose he has just $1$ guess. Then clearly he cannot handle any range with more than $1$ number.
            If he has $2$ guesses, then he can handle the range from $1$ to $3$, inclusive,
            because he can first guess the middle number, $2$, and if that is not correct,
            then by knowing whether his guess is too high or too low, he will be able to guess correctly the second time.
            If he has $3$ guess, then he can handle the range from $1$ to $7$ by initially guessing $4$, again the middle number.
            If that is wrong, then he has narrowed the range to $3$ numbers and can apply his strategy on that range with his remaining $2$ guesses.
            The range pattern is $1$, $3$, $7$, $15$, $\dots$.
            With $g$ guesses, Martin can handle a range of $2^g - 1$ numbers.
            Because $2^{10} - 1 = 1023$ and $2^11 - 1 = 2047$, $10$ guesses will not be enough to handle the range from $1$ to $2018$, but $11$ guesses will be enough.
            The strategy is to make a guess right in the middle of the narrowed range at each stage until the range has been narrowed to one number.
            The answer is $n = 11$.
        \end{solution}
    \end{enumerate}
\end{multicols}
\end{document}