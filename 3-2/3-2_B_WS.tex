\documentclass{article}

\usepackage[margin=0.5in]{geometry}
\usepackage{multicol}
\usepackage{amsthm}
\usepackage{amsmath}

\theoremstyle{definition}
\newtheorem*{solution}{Solution}

\title{General Number Theory Set B}
\date{}
\author{}

\begin{document}
\maketitle

\begin{multicols}{2}
    \begin{enumerate}
        \item Syd doubles the $3$-digit number $TWO$ to get the $4$-digit number $FOUR$.
        Each letter represents a distinct digit, and $O$ represents the digit $7$.
        What is the $3$-digit number $TWO$?
        \begin{solution}
            We are told that $O$ represents the digit $7$.
            Next, $7 \cdot 2 = 14$, which has units digit $4$, so $R$ must represent the digit $4$ and that $1$ will be carried over to the tens place.
            Then, the greatest possible result when a $3$-digit number is doubled is $1998 = 999 \cdot 2$, so $F$ can only represent the digit $1$.
            Now we need to determine which distinct digits are represented by $T$, $W$ and $U$.
            The hundreds digit of $FOUR$ is $7$, which means $T = 8$, because the carry from the tens digit is at most $1$.
            The only digits to choose from for $W$ and $U$ are $3$, $5$, $6$, $8$ and $9$.
            Since we need to carry $1$ from the tens place to the hundreds place, $W$ has to represent a digit greater than or equal to $5$.
            But $2 \cdot 5 + 1 = 11$ results in $U$ representing $1$, which doesn't work out since the digits are all distinct and we've already determined that $F$ represents the digit $1$.
            Likewise, if $W$ represents $7$, $8$ or $9$, we end up with digits that are not distinct.
            So, it must be the case that $W$ represents the digit $6$, in which case $U$ represents the digit $3$, since $6 \cdot 2 + 1 = 13$
            Now that we know what digit each letter represents, we can rewrite the equation $TWO \cdot 2 = FOUR$ as $867 \cdot 2 = 1734$.
            Therefore, the $3$-digit number represented by $TWO$ is $\boxed{867}$.
        \end{solution}
        \item How many positive integers in the set of numbers from $1$ to $1000$ inclusive are multiples of $2$, $3$ and $5$ but not $8$?
        \begin{solution}
            The numbers which are multiples of $2$, $3$, and $5$ are the multiples of the LCM of $2$, $3$, and $5$.
            Since $2$, $3$ and $5$ are relatively prime, the least common multiple is $2 \cdot 3 \cdot 5 = 30$.
            The multiples of $30$ from $1$ to $1000$ are $1 \cdot 30 = 30$, $2 \cdot 30 = 60$, all the way to $33 \cdot 30 = 990$, so there are $33$ of them.
            We need to subtract out the numbers included here which are multiples of $8$.
            The numbers which are multiples of both $30$ and $8$ are the multiples of the LCM of $30$ and $8$, which is $120$.
            The largest multiple of $120$ less than $1000$ is $120 \cdot 8 = 960$, so we need to subtract $8$ from $33$ to get $\boxed{25}$.
        \end{solution}
        \item If $n$ is the product of three consecutive positive integers and $n = 22 \cdot 14 \cdot k$,
        what is the least possible value of $k$?
        \begin{solution}
            $k$ is proportional to $n$, so if we want to minimize $k$, we should minimize $n$.
            The prime factorization of $n$ must include the prime factorization of $22 \cdot 14$, which is $2^2 \cdot 7 \cdot 11$.
            If we had one number that is a multiple of both $11$ and $7$, that would make $n$ at least $75 \cdot 76 \cdot 77$, which is probably bigger than necessary.
            Therefore, we should have one number that's a multiple of $11$ and another number that's a multiple of $7$.
            $21$ and $22$ are smallest two such numbers that are close enough together, and $20$ is conveniently a multiple of $4$, the remaining factor of $22 \cdot 14$ besides $7$ and $11$.
            Their product is $20 \cdot 21 \cdot 22 = 9240$, so $k$ would have to be $\frac{9240}{22 \cdot 14} = \boxed{30}$.
        \end{solution}
        \item Rahul Ilhangovan can arrange his dad's collection of quarters as a rectangular array with $10$ equal rows,
        $12$ equal rows or $18$ equal rows, using all the quarters in each arrangement.
        What is the least possible monetary value in dollars of the quarter collection?
        \begin{solution}
            The number of quarters must be a multiple of $10$, $12$, and $18$.
            The least common multiple (LCM) of $10$, $12$ and $18$ is $180$.
            The monetary value of $180$ quarters is $\frac{180}{4} = \boxed{\$45}$.
        \end{solution}
        \item Maxine secretly chooses a positive integer between $1$ and $2018$, inclusive.
        Martin wants to identify her number with a series of guesses.
        Each time Martin makes a guess, Maxine tells him whether his number is correct, too high or too low.
        With an appropriate strategy, Martin can always identify Maxine's number after at most $n$ guesses.
        What is the least value of $n$ such that Martin can always correctly guess Maxine's number?
        \begin{solution}
            The best strategy for Martin to follow can be found by looking at the problem backwards
            and asking, ``How wide a range of numbers can Martin handle with $g$ guesses?''
            Suppose he has just $1$ guess. Then clearly he cannot handle any range with more than $1$ number.
            If he has $2$ guesses, then he can handle the range from $1$ to $3$, inclusive.
            He can first guess $2$, and if that's not correct, he will know if its $1$ or $3$ and be able to guess it correctly next time.
            If he has $3$ guesses, then he can handle the range from $1$ to $7$ by initially guessing $4$, again the middle number.
            If that is wrong, then he has narrowed the range to $3$ numbers and can apply his strategy on that range with his remaining $2$ guesses.
            The pattern for the maximum range that can be handled with a given number of guesses is $1$, $3$, $7$, $15$, $\dots$.
            With $g$ guesses, Martin can handle a range of $2^g - 1$ numbers.
            Because $2^{10} - 1 = 1023$ and $2^11 - 1 = 2047$, $10$ guesses will not be enough to handle the range from $1$ to $2018$, but $11$ guesses will be enough.
            The strategy is to make a guess right in the middle of the narrowed range at each stage until the range has been narrowed to one number.
            The answer is $n = \boxed{11}$.

            To be $100\%$ sure that the pattern is $2^g - 1$, we can prove it using mathematical induction.
            If the maximum range for $g$ guesses is $x$, then the maximum range for $g + 1$ guesses is $2x + 1$, since we can have double the range for $g$ guesses plus the number in the middle that we guess first.
            The pattern works for $g = 1$, since $2^1 - 1 = 1$.
            If we assume that the maximum range for some $g$ is $2^g - 1$, then the maximum range for $g + 1$ is $2(2^g - 1) + 1 = 2^{g + 1} - 1$, so the maximum range for $g$ guess is $2^g - 1$ for all $g$ by induction.
            We will probably learn more about mathematical induction when we get to the unit on proofs.
        \end{solution}
    \end{enumerate}
\end{multicols}
\end{document}
