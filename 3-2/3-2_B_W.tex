\documentclass{article}

\usepackage[margin=0.5in]{geometry}
\usepackage{multicol}

\title{General Number Theory Set B}
\date{}
\author{}

\begin{document}
\maketitle
\noindent Problems should be solved without calculators unless otherwise specified.
Remember to explain how you solved a problem.
\begin{multicols}{2}
    \begin{enumerate}
        \item Brett doubles the $3$-digit number $TWO$ to get the $4$-digit number $FOUR$.
            Each letter represents a distinct digit, and $O$ represents the digit $7$.
            What is the $3$-digit number $TWO$?
            \vspace{3cm}
        \item How many positive integers in the set of numbers from $1$ to $1000$ inclusive are multiples of $2$, $3$ and $5$ but not $8$?
            \vspace{3cm}
        \item If $n$ is the product of three consecutive positive integers and $n = 22 \cdot 14 \cdot k$, what is the least possible value of $k$?
            \vspace{3cm}
        \item Eddy can arrange his dad's collection of quarters as a rectangular array with $10$ equal rows, $12$ equal rows or $18$ equal rows, using all the quarters in each arrangement.
            What is the least possible monetary value in dollars of the quarter collection?
            \vspace{3cm}
        \item Maxine secretly chooses a positive integer between $1$ and $2018$, inclusive.
            Martin wants to identify her number with a series of guesses.
            Each time Martin makes a guess, Maxine tells him whether his number is correct, too high, or too low.
            With an appropriate strategy, Martin can always identify Maxine's number after at most $n$ guesses.
            What is the least value of $n$ for which Martin can correctly guess Maxine's number?
            \vspace{3cm}
    \end{enumerate}
\end{multicols}
\end{document}
