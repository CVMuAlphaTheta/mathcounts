\documentclass{article}

\usepackage[margin=0.5in]{geometry}
\usepackage{multicol}
\usepackage{amsthm}
\usepackage{amsmath}
\usepackage[group-separator={,}]{siunitx}

\theoremstyle{definition}
\newtheorem*{solution}{Solution}
\title{General Number Theory Set A}
\date{}
\author{}

\begin{document}
\maketitle
\begin{multicols}{2}
    \begin{enumerate}
        \item What is the remainder when the absolute difference between $4 \cdot 199$
        and $7 \cdot 219$ is divided by $3$?
        \begin{solution}
            Since we only want the remainder when this difference is divided by $3$,
            we can make our calculations easier by working with the remainders from the start.
            We can rewrite $4 \cdot 199$ as $1 \cdot 1 = 1$.
            In other words, since each of these numbers is $1$ more than a multiple of $3$,
            their product will be $1$ more than a multiple of $3$.
            Similarly, we can rewrtie $7 \cdot 219$ as $1 \cdot 0 = 0$.
            Since the second product is greater than the first, let's subtract the first product from the second product,
            so our remainder is $0 - 1 = -1$
            But a remainder of $-1$ when dividing by $3$ is equivalent to a remainder of $2$, since $-1 \equiv 2 \pmod{3}$.
            So, our remainder is $2$.
        \end{solution}
        \item What fraction of positive integer factors of $1000^3$ are perfect squares?
        Express your answer as a common fraction.
        \begin{solution}
            The prime factorization of $1000^3$, or $\num{1000000000}$, is $2^9 \cdot 5^9$.
            This number has $10 \cdot 10 = 100$ positive factors.
            To combine these prime factors in ways that produce perfect square factors,
            we calculate that each of $5$ perfect square powers of $2$ (namely $2^0$, $2^2$, $2^4$, $2^6$ and $2^8$)
            can be multiplied by each of $5$ perfect square powers of $5$ (namely $5^0$, $5^2$, $5^4$, $5^6$, and $5^8$).
            That's $5 \cdot 5 = 25$ perfect square factors, which is $\frac{25}{100} = \frac{1}{4}$ of that factors of factors of $1000^3$
        \end{solution}
        \item What is the smallest possible integer multiple of $130$ that is divisible by $365$?
        \begin{solution}
            We are looking for the least common multiple of $130$ and $365$.
            The prime factorization of $365$ is $5 \cdot 73$.
            We know that $130$ has $5$ as a factor, but not $73$.
            The LCM is $130 \cdot 73 = 9490$.
        \end{solution}
        \item When the three-digit number $ABC$ is multiplied by $9$, the result is the four-digit number $1ABC$.
        What is the three-digit number $ABC$?
        \begin{solution}
            The digit $C$ multiplied by $9$ still leaves the digit $C$ in the ones place, so it must be either $0$ or $5$.
            Zero will not carry anything to the next column, so $C$ has to be $5$ and a digit of $4$ will carry over to the tens column.
            This means the digit $B$ times $9$ plus the $4$ that was carried has to leave $B$ in the tens place.
            This works if $B$ is $2$, since $2 \cdot 9 + 4 = 22$, and now a $2$ will carry over to the hundreds column.
            The digit $A$ times $9$ plus $2$ has to leave the digit $A$ in the hundreds place and only carry a $1$ to the thousands place.
            This works if $A$ is $1$, since $1 \cdot 9 + 2 = 11$. We have $125 \cdot 9 = 1125$, so the three-digit number $ABC$ is $125$.
        \end{solution}
        \item Chase Coutinho has $p$ pennies, $n$ nickels, $d$ dimes and $q$ quarters with a total value of $\$1.08$.
        If the numbers $p$, $n$, $d$ and $q$ are distinct and positive,
        and the greatest common divisor of each pair of these numbers is $1$,
        what is the least possible value of $p + n + d + q$?
        \begin{solution}
            Since we know that the quantities are distinct and pairwise relatively prime, the least we can hope for is $1 + 2 + 3 + 5 = 11$.
            Let's see if we can find a scenario that works with those quantities and totals $\$1.08$.
            We can start by letting $p = 3$. That leaves $1.08 - 0.03 = \$1.05$ left to make with only nickels, dimes and quarters.
            If we then let $n = 1$, $d = 5$ and $q = 2$, our total is $\$1.08$.
            Thus, the least possible value of $p + n + d + q$ is $11$.
        \end{solution}
    \end{enumerate}
\end{multicols}
\end{document}