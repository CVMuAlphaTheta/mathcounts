\documentclass{article}

\usepackage[margin=0.5in]{geometry}
\usepackage{multicol}
\usepackage{amsthm}
\usepackage{amsmath}

\theoremstyle{definition}
\newtheorem*{solution}{Solution}

\title{More Combinatorics Set A}
\date{}
\author{}

\begin{document}
\maketitle

\begin{multicols}{2}
    \begin{enumerate}
        \item In how many ways can a five letter ``word'' be written using only the 
            first half of the alphabet with no repetitions such that the third and 
            fifth letters are vowels and the first letter is a consonant?
            \begin{solution}
                There are three vowels and ten consonants in the first half of the 
                alphabet. Following the strategy of counting the position with the 
                most restrictions first, we'll first count the number of ways to 
                choose the letters in the first, third, and fifth positions and then 
                count the number of ways to choose the letters in the second and 
                fourth positions. There are $10$ ways to choose the first letter and 
                $3$ ways to choose the third letter. Note that we don't have to 
                worry about the two letters being the same because one is a 
                consonant and the other is a vowel. Next, there are $2$ ways to 
                choose the fifth letter since we can't repeat the third letter. 
                There are $10$ ways to choose the second letter since there are $13$ 
                letters in the first half of the alphabet and $3$ of them are 
                already used. Similarly, there are $9$ ways to choose the fourth 
                letter. Therefore, the total is $10 \cdot 3 \cdot 2 \cdot 10 \cdot 9 
                = 5400$.
            \end{solution}
        \item In how many ways can the letters SEANCHEN be rearranged?
            \begin{solution}
                There are $8$ letters in total. There are two E's and two N's, and 
                all the other letters are unique. There are $8!$ ways to arrange $8$ 
                distinct things. If we considered the two E's and the two N's to be 
                unique, then there would be $2$ ways to order the E's and $2$ ways 
                to order the N's. Since reordering the E's and the N's doesn't 
                actually change anything, the number of arrangements where all the 
                letters are unique is $2! \cdot 2! = 4$ times the number of 
                arrangements that we want to count. Therefore, there are 
                $\frac{8!}{4} = 10080$ ways to arrange the letters SEANCHEN.
            \end{solution}
        \item At the end of a professional bowling tournament, the top $5$ bowlers 
            have a playoff. First \#$5$ bowls \#$4$. The loser receives fifth prize 
            and the winner bowls \#$3$ in another game. The loser of this game 
            receives fourth prize and the winner bowls \#$2$. The loser of this game 
            receives third prize and the winner bowls \#$1$. The winner of this game 
            gets first prize and the loser gets second prize. In how many orders can 
            bowlers \#$1$ through \#$5$ receive the prizes?
            \begin{solution}
                There are four games. We might hypothesize that every sequence of 
                outcomes for the four games result in a different prize order. To 
                prove this, imagine changing the outcome for some of the games. No 
                matter how we change the outcomes, there will always be at least one 
                bowler who gets a higher prize than before, so the order of the 
                prizes must be different. Since each sequence of outcomes result in 
                a different prize order, the number of possible prize orders is 
                equal to the number of possible outcome sequences, which is $2^4 = 
                16$.
            \end{solution}
        \item In how many ways can the letters in the word MINIMIZATION be arranged 
            such that no two vowels are consecutive?
            \begin{solution}
                Notice that exactly half of the letters are vowels. First, we'll 
                choose the positions for the vowels. It's easy to see that there are 
                $7$ ways to choose $6$ non-adjacent positions out of $12$. Now that 
                we have chosen the positions for the vowels and consonants, we can 
                count the number of ways to arrange the vowels and consonants 
                separately. For the vowels, we have one O, four I's, and one A. 
                There are $6$ ways to choose the position of the O, and $5$ ways to 
                choose the position of the A, and the rest must be I's. Therefore, 
                there are $6 \cdot 5 = 30$ ways to order the vowels. For the 
                consonants, we have two N's, two M's, one T, and one Z. There are 
                $6!$ ways to arrange six letters, and we divide by $2! \cdot 2!$ to 
                account for the two pairs of repeated letters. (See the notes for 
                more details on why we do this.) The number of ways to arrange the 
                consonants is then $\frac{6!}{2! \cdot 2!} = 180$. The total is then 
                $7 \cdot 30 \cdot 180 = 37800$.
            \end{solution}
        \item In how many ways can $4$ identical red chips and $2$ identical white chips be 
            arranged in a circle?
            \begin{solution}
                The only thing that matters is the gap between the two white chips. 
                The white chips can be adjacent, have one red chip between them, or 
                have two red chips between them. Therefore there are $3$ 
                arrangements.
            \end{solution}
    \end{enumerate}
\end{multicols}
\end{document}
