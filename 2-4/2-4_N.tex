\documentclass[twocolumn]{article}

\usepackage[margin=0.5in]{geometry}
\usepackage{flushend}
\usepackage{asymptote}
\usepackage{amsmath}
\usepackage{enumitem}

\title{2.4 Notes}
\author{}
\date{}

\begin{document}

\section*{Concepts}

\subsection*{Paths on a Grid}
Combinations sometimes appear in places where you don't expect them. One example 
of this is counting the number of paths on a grid. Let's say we have a $4$ by 
$5$ grid and we wanted to know the number of paths from the top left corner to 
the bottom right corner which only move downwards and rightwards.
\begin{center}
	\begin{asy}
		unitsize(0.5inch);
		for (int i = 0; i < 4; ++i)
		{
			draw((0, i) -- (4, i));
		}
		for (int i = 0; i < 5; ++i)
		{
			draw((i, 0) -- (i, 3));
		}
	\end{asy}
\end{center}
In order to get from the top left corner to the bottom right corner, we have to 
move downwards exactly three times and move rightwards exactly four times in any 
order. We make seven moves in total and we can choose three of them to be 
downwards moves, so there are $\binom{7}{3} = \frac{7 \cdot 6 \cdot 5}{3 \cdot 
2} = 35$ paths satisfying the restrictions.

\subsection*{Complementary Counting}
If we want to count the number of objects satisfying a certain property, it is 
often easier to count the number of objects which do not satisfy the property 
and then subtract it from the total. You should try complementary counting 
before going into tedious casework.

To demonstrate complementary counting, we can look at problem 21 from the 2006 
AMC 10A. It asks to count the number of \textbf{four-digit positive integers} 
that have \textbf{at least} one digit that is a $2$ or a $3$. First of all, 
notice that four-digit positive integers are not the same as for-digit 
sequences, because the first digit can't be $0$. If we tried to count the number 
of possibilities for each digit, we will quickly run into a problem: the 
possible values for one digit depends on the values of the other digits. If the 
first digit is $2$ or $3$, then the other digits can be anything, but if the 
first digit is not $2$ or $3$, then the other digits must have a $2$ or $3$. We 
can go through all the cases, but there's a better way. We first count the 
number of four-digit positive integers which don't have a $2$ or a $3$. This is 
easier to do because now the possible values of each digit don't depend on the 
values of other digits. We simply can't have $2$ or $3$. The first digit has $7$ 
possibilities since we exclude $0$, $2$, and $3$. The other three digits each 
have $8$ possibilities, for a total of $3584$. There are $9000$ four-digit 
positive integers, so the number of positive four-digit integers which have at 
least one $2$ or $3$ is $9000 - 3584 = 5416$.

\subsection*{Stars and Bars}
Another non-obvious application of combinations is to count the number of ways 
to distribute $n$ indistinguishable objects into $k$ distinguishable bins, such 
that no bin is empty. To solve this problem, we can think of it as inserting $k 
- 1$ dividers (represented by bars) between the $n$ objects (represented by 
stars). The stars to the left of the first bar are in the first bin, the stars 
between the first and second bars are in the second bin, and so on. We can't put 
a bar to the left of the first star, to the right of the last star, or right 
next to another bar, because that would represent an empty bin. Therefore there 
are $n - 1$ gaps between the stars where we have to place $k - 1$ bars, so the 
number of ways to distribute $n$ indistinguishable objects into $k$ 
distinguishable boxes such that no bin is empty is $\binom{n - 1}{k - 1}$.

We can also count the number of ways to distribute $n$ indistinguishable objects 
into $k$ distinguishable bins where a bin can be empty. We can reduce this to 
the previous problem simply by adding $k$ stars. There's a one-to-one 
correspondence between distributions of $n$ objects into $k$ potentially empty 
bins and distributions of $n + k$ objects into $k$ non-empty bins because for 
each distribution of the first kind, we can add an object to every bin to get a 
distribution of the second kind, and for each distribution of the second kind, 
we can remove an object from every bin to get a distribution of the first kind. 
Therefore, the number of ways to distribute $n$ indistinguishable objects into 
$k$ distinguishable bins where a bin can be empty is $\binom{n + k - 1}{k - 1}$.

\subsection*{$\binom{n}{k} = \binom{n}{n - k}$}
This is one of the most basic properties of combinations. This is true because 
choosing a set of $k$ objects out of $n$ objects is the same as choosing $n - k$ 
objects out of $n$ objects to exclude from the set. We can also prove this using 
the formula for combinations: $\binom{n}{k} = \frac{n!}{k!(n - k)!} = 
\frac{n!}{(n - (n - k))!(n - k)!} = \frac{n!}{(n - k)!(n - (n - k))!} = 
\binom{n}{n - k}$. If you wanted to compute $\binom{100}{98}$, you can instead 
compute $\binom{100}{2}$, which is a little easier to think about.

\section*{Problems}

\subsection*{Combinations}
\begin{enumerate}
	\item In how many ways can a three-person subcommittee be chosen from a 
		five-person committee?
		\vspace{3cm}
	\item In how many ways can a three-person subcommittee be chosen from a 
		five-person committee if a particular person must be on the 
		subcommittee?
		\vspace{3cm}
	\item How many ways are there to pick no objects from a set of $n$ objects?
		\vspace{3cm}
	\item Show that $\binom{n}{k} = \frac{n!}{(n - k)!k!}$.
		\vspace{3cm}
	\item Show that $\binom{n}{k} = \binom{n}{n - k}$.
		\vspace{3cm}
\end{enumerate}

\subsection*{Complementary Counting}
\begin{enumerate}[resume]
	\item Anwar flips a fair coin eight times. In how many ways can he flip at 
		least two heads?
		\vspace{3cm}
	\item Felix and Candice are on the same science bowl (that's a thing?!?) 
		team. There are eight competitors on the team. If there are five people 
		participating in the astronomy event on the science bowl team, how many 
		different combinations include Felix, Candice, or both?
		\vspace{3cm}
\end{enumerate}

\subsection*{Stars and Bars}
\begin{enumerate}[resume]
	\item Alex has ten one-dollar bills to distribute among the five members of 
		Mu Alpha Theta. How many ways are there to distribute his money?
		\vspace{3cm}
	\item How many ways are there to roll a sum of $7$ with three standard 
		six-sided dice?
		\vspace{3cm}
	\item Each of the numbers $1$ through $10$ is placed in a bag and drawn at 
		random with replacement. How many ways can three numbers be drawn whose 
		sum is $13$?
		\vspace{3cm}
\end{enumerate}
\end{document}
