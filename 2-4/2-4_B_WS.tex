\documentclass[twocolumn]{article}

\usepackage[margin=0.5in]{geometry}
\usepackage{flushend}
\usepackage{amsthm}
\usepackage{amsmath}
\usepackage{ulem}
\normalem

\theoremstyle{definition}
\newtheorem*{solution}{Solution}

\title{More Combinatorics Set B}
\author{}
\date{}

\begin{document}
\maketitle
\begin{enumerate}
    \item Less than $50$ people are at a party.
        Each person \sout{shakes hands with} elbow bumps every other person exactly once.
        If there is an odd number of total elbow bumps at the party,
        what is the largest number of people that could be at the party?
        \begin{solution}
            Let the number of people be $n$,
            so that the total number of handshakes is $\frac{n(n - 1)}{2}$.
            For this to be odd, there must be only one factor of $2$ in the product $n(n - 1)$,
            otherwise the $2$ in the denominator would cancel one $2$,
            but others would remain to make the number even.
            Since either $n$ or $n - 1$ must be even,
            whichever is even must be divisible by $2$ and not $4$.
            Thus the pairs $48, 48$ and $48, 47$ are ruled out, since $4$ divides $48$.
            The next smaller pair, $47, 46$, works fine, since $2$ only divides $46$ once.
            Therefore $n = 47$.
        \end{solution}
    \item In how many ways can Sean give one M\&M each to two children
        if he has $3$ different red, $4$ different brown, and $5$ different tan ones
        and two children insist upon having M\&Ms of different colors?
        \begin{solution}
            We can separate individual cases, find the number of ways for each,
            and then add them up.
            In this problem there are $6$ cases:
            red-brown, brown-red, red-tan, tan-red, brown-tan, tan-brown.
            For each red-brown we have $3 \cdot 4 = 12$ ways (and the same for brown-red),
            for red-tan $3 \cdot 5 = 15$ ways (same for tan-red),
            and for brown-tan $4 \cdot 5 = 20$ ways (same for tan-brown).
            The total is $2(12 + 15 + 20) = 94$.
        \end{solution}
    \item Palindromes, like $23432$, read the same forward and backward.
        Find the sum of all four digit positive integer palindromes.
        (Hint: What is the sum of the first and last four digit palindromes?)
        \begin{solution}
            The list of all the palindromes would look like
            $1001, 1111, 1221, \dots, 9889, 9999$.
            Notice that if we add the first and the last, we get $11000$.
            The same holds if we add the second and the next-to-last and so on.
            Each such pair always sums to $11000$.
            So how many pairs are there?
            There are ten palindromes starting with each digit from $1$ to $9$,
            so there are $90$ palindromes.
            Hence, there are $45$ pairs and our desired sum is $45 \cdot 11000 = 495000$.
        \end{solution}
    \item How many of the arrangements of the letters in the word COUNTING contain
        a double N?
        \begin{solution}
            The restriction requires that the N's be grouped together,
            so we do just that: treat them as a single letter.
            There are $7$ letters so we get $^7P_7 = 7! = 5040$.
        \end{solution}
    \item The $54$-member Council for Security and Cooperation in Europe wishes to 
        choose $3$ member states for different leadership positions.
        The L lobby decrees that at least one of
        Lithuania, Lichtenstein, Latvia, and Luxembourg must be chosen.
        In how many ways can the committee be selected?
        \begin{solution}
            The simplest approach is to count the number of choices of the offices
            \emph{without} choosing any of the L countries.
            Then there are $50$ states and $3$ different positions,
            making $50 \cdot 49 \cdot 48$ choices.
            We wish to subtract the number of committees
            which can be formed without restriction, or $54 \cdot 53 \cdot 52$.
            The committees left are those which satisfy the L lobby.
            The result is just $54 \cdot 53 \cdot 52 - 50 \cdot 49 \cdot 48 = 31224$.
        \end{solution}
\end{enumerate}
\end{document}
