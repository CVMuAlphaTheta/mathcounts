\documentclass{article}

\title{2.1 Solutions}
\author{}
\date{}

\begin{document}
\maketitle

\begin{enumerate}
	\item \begin{enumerate}
			\item Recall that when the outcomes of a random experiment are all 
				equally likely, the probability of an event is the number of 
				outcomes which are included in the event divided by the total 
				number of possible outcomes. The event we're interested in here 
				is the sum of the two dice being equal to $6$, so we need to 
				count the number of outcomes where the sum is $6$. If we write 
				each outcome as a pair of two numbers representing the result 
				from the two dice, we see that there are a total of $5$ ways for 
				the dice to add up to $6$: $(1, 5)$, $(2, 4)$, $(3, 3)$, $(4, 
				2)$, and $(5, 1)$. There are a total of $6^2 = 36$ possible 
				outcomes of rolling two dice, so the probability that the sum is 
				$6$ is $\frac{5}{36}$.
			\item There is one way for the sum to be $2$, two ways for the sum 
				to be $3$, three ways for the sum to be $4$, and so on, until we 
				get to $7$, which can be made in six ways. There are five ways 
				to make $8$, four ways to make $9$, and so on. Therefore, the 
				probabilities start at $\frac{1}{36}$, increase by 
				$\frac{1}{36}$ for each number until we get to $7$, and then it 
				decreases by $\frac{1}{36}$ for each number until we end up back 
				at $\frac{1}{36}$ for $12$.
		\end{enumerate}
	\item There are $7$ balls in total and Chloe draws two of them with 
		replacement, so there are a total of $7^2 = 49$ possible outcomes. To 
		count the number of outcomes where one ball is red and the other is 
		blue, we can consider two cases. If the red ball comes first, there are 
		$3$ ways to choose the red ball and $4$ ways to choose the blue ball for 
		a total of $12$. Similarly, if the blue ball comes first, there are also 
		$12$ outcomes, so we have a total of $24$ outcomes out of $49$. This 
		means the probability is $\frac{24}{49}$.
	\item To find the medians, we first need to list the numbers in order. We 
		end up with $223, 251, 317, 607, 607, 636, 766$ and $309, 341, 435, 652, 
		734, 759, 870$. The median of the first list is $607$ since that is the 
		number in the middle, and the median of the second list is $652$. The 
		difference between them is $45$.
	\item A total of $1000 - 270 = 730$ bats arrived in June, which has $30$ 
		days. Therefore, an average of $\frac{730}{30} \approx 24.3$ bats 
		arrived each day.
	\item Flipping a coin four times has $2^4 = 16$ possible outcomes. In order 
		for it to land heads up at least as many times as it lands tails up, it 
		needs to land heads up at least two times. We can count the number of 
		outcomes where it lands at heads up at least two times by counting the 
		cases where it lands heads up exactly two times, exactly three times, 
		and exactly four times. There are $6$ possible outcomes in the first 
		case, $4$ possible outcomes in the second case, and $1$ possible outcome 
		in the third case, for a total of $11$. Therefore, the probability is 
		$\frac{11}{16}$.
\end{enumerate}
\end{document}
