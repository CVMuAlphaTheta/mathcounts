\documentclass{article}

\usepackage{amsthm}

\theoremstyle{definition}
\newtheorem*{solution}{Solution}

\title{Statistics and Probability Set A}
\date{}
\author{}

\begin{document}
    \maketitle
    \noindent Problems should be solved without calculators unless otherwise
    specified. Remember to explain how you solved a problem.
    \begin{enumerate}
        \item Anwar rolls two dice and takes the sum of the results.
            \begin{enumerate}
                \item Find the probability of rolling a six if two six-sided dice are rolled.
                \begin{solution}
                    First consider all possible ways in which a $6$ cam arise:
                    $1-5, 2-4, 3-3, 4-2, 5-1$.  There are 5 different cases to
                    consider. However, for each case we are looking at a
                    configuration which can only concur in one way. Since there
                    are 36 total configurations, each configuration has
                    probability $\frac{1}{36}$, so that the $5$ configurations
                    which yield 6 add up to make probability $\frac{5}{36}$.
                \end{solution}
            \item Find the probability of all rolls, $2$ to $12$. Try to notice a pattern.
                \begin{solution}
                    $2: 1-1$, for probability $\frac{1}{36}$ \\
                    $3: 1-2, 2-1$, for proability $\frac{2}{36}$ \\
                    $4: 1-3, 2-2, 3-1$, for probaility $\frac{3}{36}$ \\
                    And so on. From $2$ to $12$, the numbers of ways to attain the rolls are
                    $1, 2, 3, 4, 5, 6, 5, 4, 3, 2, 1$, and the probabilities are just these
                    values divided by $36$.
                \end{solution}
        \end{enumerate} 
        \item If there are $3$ boys and $4$ girls in a group are two are chosen to give a report,
        what is the probability that one boy and one girl are chosen?
        \begin{solution}
            There are $3\cdot4 = 12$ different girl-boy partnerships possible. In total,
            there are $\frac{7\cdot6}{2} = 21$ different partnerships possible. The
            probaility is $\frac{12}{21}$.
        \end{solution}
        \item Alastair's favorite book series has seven books containing 223, 251, 317, 
        636, 766, 607, and 607 pages. Monatana's favorite book series has seven books 
        containing 309, 341, 435, 734, 870, 652, and 759 pages. What is the absolute
        difference between the median numbers of pages in the books of Alastair's and
        Monatana's favorite book series?
        \begin{solution}
            If we arrange Alastair's books from the fewest pages to the most pages, we
            have 223, 251, 317, 607, 607, 636, and 766. The median is the middle number of
            pages, which is $607$. Next, arranging Monatana's books in this manner, we have
            309, 341, 435, 652, 734, 759, and 870. The median of these numbers is $652$.
            The absolute difference between the two medians, then, is $652 - 607 = 45$.
        \end{solution}
        \item There are $270$ bats living in a cave on the first day of June. Every day in
        June, some number of bats arrive to join the bats already there. At the end of June
        30, there are $1000$ bats in the cave. On average, how many bats arrived each day
        in June? Express your answer as a decimal to the nearest tenth.
        \begin{solution}
            In the 30 days of June, the numb er of bats in the cave increased by
            $1000 - 270 = 730$ bats. Each day, on average, $730 \div 30 \approx 24.3$
            bats arrived.
        \end{solution}
    \end{enumerate}
\end{document}
