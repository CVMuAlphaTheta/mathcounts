\documentclass[twocolumn]{article}

\usepackage[margin=0.5in]{geometry}
\usepackage{flushend}
\usepackage{amsmath}
\usepackage{asymptote}

\title{2.2 Notes}
\author{}
\date{}

\begin{document}
\maketitle

\section*{Key Concepts}

\subsection*{Terminology}
A \textbf{set} is a collection of distinct \emph{things}. For example, we can 
have a set of integers, a set of people, a set of English words, etc. Sets can 
be written as a list of things enclosed in curly brackets, such as $\{1, 2, 3, 
4, 5\}$ or $\{\text{Alex}, \text{Chloe}, \text{Cedric}\}$. Sets usually contain 
things with some similarities, but it's possible to have a set containing 
completely unrelated things, like $\{\text{Cedric}, 42, 
\text{``sacrilegious''}\}$.

\subsection*{Venn Diagrams}
Venn diagrams are useful for visualizing sets with common elements. To draw a 
Venn diagram for the set of people who like math and the set of people who like 
to play sports, we would draw two intersecting circles. One circle represents 
the people who like math, and the other circle represents the people who like to 
play sports. The area where the circles intersect represent the people who both 
like math and like to play sports.
\begin{center}
	\begin{asy}
		draw(circle((-40, 0), 80));
		draw(circle((40, 0), 80));
		label("Math", (-40, 80), N);
		label("Sports", (40, 80), N);
		label(rotate(-90) * "Both math and sports", (0, 0));
		label("Only math", (-80, 0));
		label("Only sports", (80, 0));
		label("Neither math nor sports", (0, -120));
	\end{asy}
\end{center}
From this diagram, we can see that the sum of the number of people who only like 
math and the number of people who like sports is the number of people who like 
math or like sports. We can also see that the number of people who only like 
math is the number of people who like math minus the number of people who like 
both math and sports. When you solve a problem involving intersecting sets like 
this, you can use a Venn diagram to help visualize the problem. Write in numbers 
which you know and try to find the numbers which aren't given.

\subsection*{Inclusion-Exclusion Principle}
The \textbf{inclusion-exclusion principle} states that the number of things 
which are in either of two sets is the sum of the sizes of the two sets minus 
the number of things which are in two sets. For example, if $3$ people like 
math, $5$ people like sports, and $2$ people like both math and sports, then $3 
+ 5 - 2 = 6$ people like either math or sports (or both). We can use the Venn 
diagram to see why this works. If we added size of the math circle and the size 
of the sports circle together, we would count the region where the circles 
intersect twice. Therefore, we subtract the size of that region to get the size 
of the region formed by combining the two circles.

Note that the inclusion-exclusion principle can also be used in reverse. We can 
find the number of people who like both math and sports by adding the number of 
people who like math to the number of people who like sports and subtracting the 
number of people who like either math or sports. When we add the size of the 
math circle with the size of the sports circle, we count the left region once, 
the middle region twice, and the right region once. When we subtract the size of 
the region formed by combing the circles, we will end up with just the middle 
region.

\subsection*{Two-Way Tables}
\textbf{Two-way tables} are another way to visualize intersecting sets. A 
two-way table looks like this:
\begin{center}
	\begin{tabular}{| c | c | c | c |}
		\hline
		& Likes math & Doesn't like math & Total \\
		\hline
		Likes sports & $2$ & $3$ & $5$ \\
		\hline
		Doesn't like sports & $1$ & $4$ & $5$ \\
		\hline
		Total & $3$ & $7$ & $10$ \\
		\hline
	\end{tabular}
\end{center}

Notice that each of the cells in the total row is the sum of the numbers in the 
corresponding column, and each of the cells in the total column is the sum of 
the numbers in the corresponding row. For example, the total number of people 
who like math is the sum of the people who like math and like sports, and the 
people who like math and don't like sports.
\end{document}
