\documentclass{article}

\usepackage[margin=0.5in]{geometry}
\usepackage{multicol}
\usepackage{hyperref}
\usepackage{amsmath}
\usepackage{enumitem}

\DeclareMathOperator{\lcm}{lcm}

\title{3.1 Notes}
\author{}
\date{}

\begin{document}
\maketitle

\section*{Concepts}
\begin{multicols}{2}
	
	\subsection*{Divisibility Rules}
	We often need to quickly test for divisibility by small numbers.
	For example, you might want to check if $28376$ is divisible by $3$, $6$, and $8$ quickly.
	Here is a list of common divisibility rules.
	
	\begin{tabular}{|c|l|}
		\hline
		\# & Rule \\
		\hline
		$2$    & Last digit is even \\
		$3$    & Sum of the digits is a multiple of $3$ \\
		$4$    & Last two digits form a multiple of $4$ \\
		$5$    & Last digit is $0$ or $5$ \\
		$6$    & Divisible by $2$ and $3$ \\
		$8$    & Last three digits form a multiple of $8$ \\
		$9$    & Sum of the digits is a multiple of $9$ \\
		$10$   & Last digit is $0$ \\
		$11$   & Alternating digit sum is a multiple of $11$ \\
		$12$   & Divisible by both $3$ and $4$ \\
		$14$   & Divisible by both $2$ and $7$ \\
		$15$   & Divisible by both $3$ and $5$ \\
		$16$   & Last four digits form a multiple of $16$ \\
		$18$   & Divisible by both $2$ and $9$ \\
		$20$   & Last digit is $0$ and second-to-last digit is even \\
		\hline
	\end{tabular}
	
	For $11$, the alternating digit sum is the first digit minus the second digit plus the
	third digit minus the fourth digit and so on.
	To check if a number $n$ is divisible by $k$, we just check if $n$ satisfies the condition
	for $k$.
	For example, to check if $12345$ is a multiple of $3$,
	we check if the sum of the digits of $12345$ is a multiple of $3$.
	
	You can find more rules at \url{https://en.wikipedia.org/wiki/Divisibility_rule}.
	
	\subsection*{Testing for Primes}
	Here's how to quickly test if a number is prime.
	
	First, it's useful to have all the primes less than $100$ memorized.
	You can find a list of these online.
	Next, check if the last digit of the number is $1$, $3$, $7$, or $9$.
	If it's not one of these, then the number is a multiple of $2$ or $5$.
	Then we can use trial division, which means we check if the number is divisible
	by potential factors.
	We only have to test if the number is divisible by primes,
	because if a number $n$ is divisible by a composite number $k$,
	then it is also divisible by any prime factor of $k$.
	Also, we only have to test potential factors up to the square root of the number
	that we want to check.
	This is because if a number $n$ is divisible by some factor $k$ that is greater than $\sqrt{n}$,
	then it is also divisible by $\frac{n}{k}$, which is less than $\sqrt{n}$.
	
	For example, to check if $131$ is a prime, we can check if its divisible by
	$2$, $3$, $5$, $7$, or $11$.
	The next prime, $13$, is bigger than $\sqrt{131}$, so we can stop here.
	
	\subsection*{The Fundamental Theorem of Arithmetic}
	The \textbf{fundamental theorem of arithmetic} states that every positive integer
	has a unique prime factorization.
	
	\subsection*{Prime Factorizations of Perfect Squares}
	A number is a perfect square if and only if its prime factorization contains only
	even exponents.
	To see why, lets say we had a number with a prime factorization of
	$p_1^{k_1} p_2^{k_2} \dots p_n^{k_n}$, where $p_1, p_2, \dots, p_n$ are primes.
	If we took the square root, it would look like
	$\sqrt{p_1^{k_1}} \sqrt{p_2^{k_2}} \dots \sqrt{p_n^{k_n}}$.
	Taking the square root of a power is the same as halving the exponent,
	and we can only do this if the exponent is even.
	Therefore, to tell if the number is a perfect square, we simply check if all the exponents
	in the prime factorization are even.
	
	\subsection*{Counting Factors}
	A number $m$ is a factor of $n$ if the prime factorization of $m$ contains only the primes
	in the prime factorization of $n$, and each of the exponents in $m$ is less than or equal to
	the corresponding exponent in $n$.
	If the prime factorization of $n$ is $p_1^{k_1} p_2^{k_2} \dots p_n^{k_n}$,
	then each factor of $n$ is of the form $p_1^{j_1} p_2^{j_2} \dots p_n^{j_n}$,
	where $j_1 \leq k_1$, $j_2 \leq k_2$, and so on.
	For example, the prime factorization of $12$ is $2^2 \cdot 3$, so its factors are
	$2^0 \cdot 3^0 = 1$, $2^0 \cdot 3^1 = 3$, $2^1 \cdot 3^0 = 2$, $2^1 \cdot 3^1 = 6$,
	$2^2 \cdot 3^0 = 4$, and $2^2 \cdot 3^1 = 12$.
	
	We can use this fact to quickly count the number of factors without having to list all of them.
	Each factor is of the form $p_1^{j_1} p_2^{j_2} \dots p_n^{j_n}$, and each $j$ value
	can be anything from $0$ to its corresponding $k$ value.
	Therefore, there are $k_1 + 1$ possible values for $j_1$, $k_2 + 1$ possible values for $j_2$,
	and so on.
	Each combinations of $j$ values result in a different number because prime factorizations
	are unique, so the number of factors is the number of combinations of $j$ values,
	which is $(k_1 + 1)(k_2 + 1) \dots (k_n + 1)$.
	In other words, to find the number of factors, we add one to all the exponents in the
	prime factorization and the multiply them together.
	
	\subsection*{Greatest Common Divisor}
	The \textbf{greatest common divisor} of two or more numbers is the greatest integer that
	divides all of those numbers.
	For example, the GCD of $15$ and $25$ is $5$ because $5$ divides both $15$ and $25$,
	and no number bigger than $5$ divides both $15$ and $25$.
	The GCD of $a$ and $b$ is written as $\gcd(a, b)$.
	To find the GCD of two numbers, we can first find their prime factorizations.
	For $15$ and $25$, that would be $3 \cdot 5$ and $5^2$.
	Then we simply take what's in common.
	Both prime factorizations have a $5$, so this is the GCD.
	If we had $36$ and $48$, the prime factorizations would be $2^2 \cdot 3^2$ and $2^4 \cdot 3$.
	Both prime factorizations have two $2$s and one $3$, so the GCD is $2^2 \cdot 3 = 12$.
	
	Those of you feeling adventurous can try using the \textbf{Euclidean algorithm},
	which is a super duper fast way of finding the GCD.
	First, we right the two numbers next to each other with the bigger one on the left.
	Then, we take the \emph{remainder} when dividing the first number by the second,
	and write it at the end.
	We repeat the process of appending the remainder when dividing the last two numbers
	until we get a zero.
	Then the second-to-last number is a GCD.
	
	For example, to find the GCD of $97238478$ and $2938792$,
	we first write them together on a line:
	\[97238478 \quad 2938792\]
	Then, we take the remainder when dividing the two numbers, and write it at the end.
	\[97238478 \quad 2938792 \quad 258342\]
	Then we take the remainder when dividing $2938792$ by $258342$, and write it at the end.
	\[97238478 \quad 2938792 \quad 258342 \quad 97030\]
	Now take the remainder when dividing $258342$ by $97030$, and write it at the end.
	\[97238478 \quad 2938792 \quad 258342 \quad 97030 \quad 64282\]
	We repeat this until we get a zero.
	(I'm omitting some numbers at the beginning to save space.)
	\[97030 \quad 64282 \quad 32748 \quad 31534 \quad 1214 \quad 1184 \quad 30 \quad 14 \quad 2
	\quad 0\]
	Therefore, the GCD is $2$.
	Don't be intimidated by how long that was.
	It would have taken longer to factor those two numbers,
	and for smaller numbers it is much faster.
	
	\subsection*{Least Common Multiple}
	The \textbf{least common multiple} of two or more numbers is the least integer that
	is a multiple of all of those numbers.
	The LCM of $a$ and $b$ is written as $\lcm(a, b)$.
	To find the LCM of two numbers, we can find their prime factorizations, and then take everything
	that is in at least one of the numbers.
	For example, to find the LCM of $36$ and $48$ which have $2^2 \cdot 3^2$ and $2^4 \cdot 3$
	as their prime factorizations, we can take $2^4$ and $3^2$ to get $2^4 \cdot 3^2 = 144$.
	
	The LCM is related to the GCD in that the LCM of two numbers is their product divided by
	their GCD.
	For example, the LCM of $36$ and $48$ is $\frac{36 \cdot 48}{12} = 144$.
	
	\subsection*{Applications of GCD and LCM}
	Lets say we want to know what numbers are factors of both $42$ and $105$.
	We can do this by listing all the factors of $42$ and $105$, but there is an easier way using
	the properties of the GCD.
	It turns out that the factors of both $42$ and $105$ are just the factors of $\gcd(42, 105)$.
	To see why, recall that a number $a$ is a factor of $b$ if and only if the
	prime factorization of $a$ is a subset of the prime factorization of $b$.
	In other words, the prime factorization of $b$ includes everything in the prime factorization of
	$a$.
	Therefore, the numbers which are a factor of both $42$ and $105$ are the numbers which have
	prime factorizations included in both the prime factorization of $42$ and the
	prime factorization of $105$.
	We want to know what's in common between the prime factorization of $42$ and the prime
	factorization of $105$, which is just the GCD of $42$ and $105$.
	To find the numbers which are factors of both $42$ and $105$, we find the GCD of $42$ and $105$,
	which is $21$, and we just look at all the factors of $21$.

	To find the numbers that are a multiple of both $42$ and $105$, we just use the LCM of $42$
	and $105$.
	The LCM of $42$ and $105$ is $210$, so the multiples of both $42$ and $105$ are just
	the multiples of $210$.
	This works because the multiples of both $42$ and $105$ have prime factorizations which
	include both the prime factorization of $42$ and the prime factorization of $105$, and the
	LCM of $42$ and $105$ contains everything that is either in the prime factorization of $42$
	or the prime factorization of $105$.
	
	\subsection*{Sum of Factors}
	Let's say we wanted to find the sum of the factors of $144$.
	We can just list all of them, but that's annoying.
	A better way is to find the prime factorization of $144$, which is $2^4 \cdot 3^2$,
	and then do $(2^0 + 2^1 + 2^2 + 2^3 + 2^4)(3^0 + 3^1 + 3^2)
	= (1 + 2 + 4 + 8 + 16)(1 + 3 + 9) = 403$.
	To see why this works, we can expand the product using the distributive property,
	an area model, or whatever else you prefer.
	This will result in $2^0 \cdot 3^0 + 2^0 \cdot 3^1 + 2^1 \cdot 3^0 + 2^1 \cdot 3^1
	+ 2^2 \cdot 3^0 + 2^2 \cdot 3^1 + 2^3 \cdot 3^0 + 2^3 \cdot 3^1$,
	which are the factors of $144$.
	
	\subsection*{Product of Factors}
	To find the product of the factors of a number, we just raise the number to half the
	number of factors.
	For example, the number of factors of $144$ is $5 \cdot 3 = 15$, so the product of the factors
	is $144^{15 / 2} = 15407021574586368$.
	To see why this works, imagine writing out the factors of $144$ twice,
	once in increasing order and once in decreasing order:
	\[
		\begin{array}{cccccccc}
			1   & 2  & 3  & 4  & 6  & \dots & 72 & 144 \\
			144 & 72 & 48 & 36 & 24 & \dots & 2  & 1
		\end{array}
	\]
	If we multiplied all of these numbers together, we would get the square of the
	product of the factors, since each factor is in there twice.
	If you multiplied each pair of vertically adjacent numbers, you would see that the product
	is always $144$.
	The number of such pairs is the number of factors, so the product of all of these numbers
	is $144$ raised to the number of factors, which is $144^{15}$.
	This is the square of the product of the factors, so we take the square root,
	which is the same as halving the exponent.
\end{multicols}

\section*{Problems}

\begin{multicols}{2}
	\subsection*{Divisibility Rules}
	\begin{enumerate}
		\item Find the prime factorization of $123420$.
			\vspace{3cm}
	\end{enumerate}
	
	\subsection*{GCD and LCM}
	\begin{enumerate}[resume]
		\item Find the prime factorization of $97$.
			\vspace{3cm}
		\item Find the GCD of $100$ and $1000$.
			\vspace{3cm}
		\item Find the LCM of $100$ and $1000$.
			\vspace{3cm}
		\item Show that the product of the LCM and the GCF of two integers
			is equal to the product of those two integers.
			\vspace{3cm}
		\item If the GCD of two numbers is $8$ and the product of the numbers is $2880$,
			what is the LCM of the numbers?
			\vspace{3cm}
	\end{enumerate}
	
	\subsection*{Factors}
	\begin{enumerate}[resume]
		\item Find the prime factorization of $240$.
			\vspace{3cm}
		\item How many factors does $240$ have?
			\vspace{3cm}
		\item What is the product of the factors of $240$?
			\vspace{3cm}
		\item What is the sum of the factors of $240$?
			\vspace{3cm}
	\end{enumerate}
\end{multicols}
\end{document}
