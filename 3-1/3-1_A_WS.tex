\documentclass{article}

\usepackage[margin=0.5in]{geometry}
\usepackage{multicol}
\usepackage{amsthm}
\usepackage{amsmath}

\theoremstyle{definition}
\newtheorem*{solution}{Solution}

\title{Prime and Composite Numbers Set A}
\date{}
\author{}

\begin{document}
\maketitle
\begin{multicols}{2}
    \begin{enumerate}
        \item The greatest common factor of $42$ and $24$ is $m$.
            What is the least common multiple of $m$ and $15$?
            \begin{solution}
                The prime factorization of $42$ is $2 \cdot 3 \cdot 7$ and the prime factorization
                of $24$ is $2 \cdot 3 \cdot 4$.
                So, for $42$ and $24$, the greatest common factor (GCF) is $2 \cdot 3 = 6$.
                The prime factorization of $15$ is $3 \cdot 5$.
                So, the least common multiple (LCM) of $6$ and $15$ is $2 \cdot 3 \cdot 5 =
                \boxed{30}$.
            \end{solution}
        \item How many even numbers are factors of $240$?
            \begin{solution}
                The prime factorization of $240$ is $2^4 \cdot 3 \cdot 5$.
                Therefore, any factor of $240$ is made of zero to four $2$'s, zero or one $3$, and
                zero or one $5$.
                In other words, the prime factorization of any factor of $240$ consists of
                $2$, $3$, and $5$, where the exponent of $2$ is $0$ to $4$, the exponent for $3$
                is $0$ to $1$, and the exponent for $5$ is $0$ to $1$.
                The even factors are the ones with at least one $2$ in their prime factorization,
                so there are $4$ ways to choose the exponent for $2$ since it can be $1$ to $4$,
                $2$ ways to choose the exponent for $3$,
                and $2$ ways to choose the exponent for $5$.
                It follows that the answer is $2 \cdot 2 \cdot 4 = \boxed{16}$.
            \end{solution}
        \item Let $A = 1$, $B = 2$, $C = 3$, and so on.
            Call the product value of a word the product of its letter values.
            For example, the product value of CAT is $3 \cdot 1 \cdot 20 = 60$.
            The name of which state in the US has a product value of $3105$?
            \begin{solution}
                The prime factorization of $3105$ is $3 \cdot 3 \cdot 3 \cdot 5 \cdot 23$.
                W must be one of the letters of this state, since it's the $23$rd letter of the
                alphabet and no multiple of $23$ is a letter.
                Clearly, the other letters can't be C, C, C, and E.
                So, we have to multiply some factors and see what letters we get.
                If we use $3 \cdot 3 = 9$ and $3 \cdot 5 = 15$, we get the letters I and O.
                Now we have three letters: W, I, and O.
                Since $A = 1$, we have $9 \cdot 15 \cdot 23 \cdot 1 = 3105$.
                These letters can be arranged to spell the name of the state in the US
                that has a product value of $3105$, which is \fbox{IOWA}.
            \end{solution}
        \item How many divisors of $3240$ are perfect squares?
            \begin{solution}
                The prime factorization of $3240$ is $2^3 \cdot 3^4 \cdot 5$.
                The numbers which we want to count must have the primes $2$, $3$, and $5$ in their
                prime factorizations, and the exponent of each prime must be from $0$ to the
                corresponding exponent in the prime factorization of $3240$.
                Each exponent must also be even, since this makes the number a perfect square.
                Therefore, there are $2$ ways to choose the exponent for $2$, $3$ ways to choose
                the exponent for $3$, and $1$ way to choose the exponent for $6$.
                This means the answer is $2 \cdot 3 \cdot 1 = \boxed{6}$.
            \end{solution}
        \item How many positive divisors of $30!$ are prime?
            \begin{solution}
                All primes less than or equal to $30$ are factors of $30!$.
                No prime greater than $30$ is a factor of $30!$,
                since $30!$ can be completely factored into numbers less than or equal to $30$.
                Therefore, the prime divisors of $30!$ are just the primes less than
                or equal to $30$.
                There are $\boxed{10}$ such primes:
                $2$, $3$, $5$, $7$, $11$, $13$, $17$, $19$, $23$, and $29$.
            \end{solution}
    \end{enumerate}
\end{multicols}
\end{document}
