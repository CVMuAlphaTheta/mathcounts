\documentclass[twocolumn]{article}

\usepackage[margin=0.5in]{geometry}
\usepackage{flushend}
\usepackage{amsthm}
\usepackage{amsmath}
\usepackage{forest}

\theoremstyle{definition}
\newtheorem*{solution}{Solution}

\title{Prime and Composite Numbers Set A}
\date{}
\author{}

\begin{document}
\maketitle
\begin{enumerate}
    \item The greatest common factor of $42$ and $24$ is $m$. What is the least
        common multiple of $m$ and $15$?
        \begin{solution}
            The prime factorization of $42$ is $2 \cdot 3 \cdot 7$ and the prime 
            factorization of $24$ is $2 \cdot 3 \cdot 4$. So, for $42$ and $24$, 
            the greatest common factor (GCF) is $2 \cdot 3 = 6$. The prime
            factorization of $15$ is $3 \cdot 5$. So, the least common multiple
            (LCM) of $6$ and $15$ is $2 \cdot 3 \cdot 5 = 30$.
        \end{solution}
    \item How many odd numbers are factors of $240$?
        \begin{solution}
            Begin by looking at the prime factorization of $240$: $240 = 2^4 \cdot
            3^1 \cdot 5^1$. Any number multiplied by $2$ is even, so the odd factors
            are the ones that have no $2$'s in their prime factorization. We can only
            use the $3$'s and $5$'s as factors for a total of $2 \cdot 2 = 4$ factors:
            $1, 3, 5, \textrm{ and } 15$. 
        \end{solution}
    \item How many even numbers are factors of $240$?
        \begin{solution}
            Use complementary counting. Count the total number of factors $(20)$ and
            subtract the number of odd factors $(4)$ from above for a total of $16$
            even factors. 
        \end{solution}
    \item What is the sum of the distinct prime factors of $156$?
        \begin{solution}
            One way to arrive at the prime factorization of a number is to repeatedly
            divide the number and subsequent quotients by the primes in order $2, 3, 5$
            and so on, until the result is a prime. In this case, we start with $156 \div 2
            = 78$. This number is even and, therefore, not prime, so we divide $78 \div 2
            = 39$. Since the result is odd, it is not evenly divisible by $2$, so we divide
            by the next prime and get $39 \div 3 = 13$, which is prime. This process is
            depicted in the factor tree shown. Thus, the prime factorization of $156$ is
            $2 \cdot 2 \cdot 3 \cdot 13$. The sum of the distinct prime factors is $2 + 3 +
            13 = 18$.
            \begin{center}
                \begin{forest}
                    [156
                        [2]
                        [78 [2][39
                            [3][13]]]
                    ]     
                \end{forest}
            \end{center}
        \end{solution}
    \item Let $A = 1$, $B = 2$, $C = 3$, and so on. Call the product value of a word
        the product of its letter values. For example, the product value of CAT is
        $3 \cdot 1 \cdot 20 = 60$. The name of which state in the U.S. has a product
        value of $3105$?
        \begin{solution}
            The prime factorization of $3105$ is $3 \cdot 3 \cdot 3 \cdot 5 \cdot 23$. W
            must be one of the letters of this state, since it's the $23$rd letter of the
            alphabet and no multiple of $23$ is a letter. Clearly, the other letters can't
            be C, C, C, and E. So, we have to multiply some factors and see what letters we
            get. If we use $3 \cdot 3 = 9$ and $3 \cdot 5 = 15$, we get the letters I and O.
            Now we have three letters: W, I, and O. Since $A = 1$, we have $9 \cdot 15 \cdot
            23 \cdot 1 = 3105$. These letters can be arranged to spell the name of the state
            in the U.S. that has a product value of $3105$, which is IOWA.
        \end{solution}
\end{enumerate}
\end{document}