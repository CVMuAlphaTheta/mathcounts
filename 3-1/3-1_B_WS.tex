\documentclass{article}

\usepackage[margin=0.5in]{geometry}
\usepackage{multicol}
\usepackage{amsthm}
\usepackage{amsmath}
\usepackage[group-separator={,}]{siunitx}

\DeclareMathOperator{\lcm}{lcm}

\theoremstyle{definition}
\newtheorem*{solution}{Solution}

\title{Prime and Composite Numbers Set B}
\date{}
\author{}

\begin{document}
\maketitle

\begin{multicols}{2}
    \begin{enumerate}
        \item What is the smallest four-digit number that is divisible by $2$, $3$, $4$, $5$, $6$,
            $8$, $9$, and $10$?
            \begin{solution}
                The number must end in a zero.
                Let's assume that to be the smallest, it should start with a $1$.
                Since the digits must add up to $9$, the last three digits must add up to $8$.
                This leads us to $1080$, and we can test to see that it is a multiple of $2$, $3$,
                $4$, $5$, $6$, $8$, $9$, and $10$.

                Consider the prime factorization of the result.
                You need a $2$ for divisibility by $2$, a $3$ for divisibility by $3$,
                another $2$ for divisibility by $4$, and a $5$ for divisibility by $5$.
                The number is already divisible by $6$.
                Another $2$ must be included for divisibility by $8$, and another $3$ for
                divisibility by $9$.
                The resulting number is already divisible by $10$.
                That gives us $2^3 \cdot 3^2 \cdot 5 = 360$.
                We were asked for a four-digit integer,
                so we need to multiply $360$ by $3$ to get $1080$.

                The simplest solution is to use the fact that the multiples of all the listed
                numbers are just the multiples of the LCM of all of the numbers.
                To find $\lcm(2, 3, 4, 5, 6, 8, 9, 10)$, we can first ignore $2$, $3$, $4$, and $5$,
                since any multiple of $6$, $8$, $9$, and $10$ are multiples of $2$, $3$, $4$,
                and $5$.
                If we prime factorize $6$, $8$, $9$, and $10$, we see that a number that is
                a multiple of all of these must contain $2^3$, $3^2$, and $5$, so the LCM
                is $2^3 \cdot 3^2 \cdot 5 = 360$.
                Our answer is the smallest four-digit multiple of $360$, which is $\boxed{1080}$.
            \end{solution}
        \item Using only $1$'s and $2$'s, what is the smallest integer you can create which is
            divisible by both $3$ and $8$?
            \begin{solution}
                The units digit must be $2$.
                Since $12$ and $22$ don't work, we need a $3$-digit number that is divisible by $8$.
                $\frac{112}{8} = 14$, but it's not divisible by $3$.
                Unfortunately, $122$, $212$, and $222$ are not divisible by $8$.
                We must go to a $4$-digit number that ends in $112$ and whose digit sum is a
                multiple of $3$.
                The only $4$-digit number that works is $\boxed{2112}$.
            \end{solution}
        \item Each of the digits $0$ through $9$ is used exactly once to create a ten-digit integer.
            Find the greatest ten-digit number which uses each digit once and is divisible by $8$,
            $9$, $10$, and $11$.
            \begin{solution}
                The sum of the digits is a multiple of $9$ and this does not depend on the order
                of the digits, so we can ignore the divisible by $9$ requirement.
                The last digit must be $0$, since the number is a multiple of $10$.
                The last three digits must form a multiple of $8$, and the alternating digit sum
                must be a multiple of $11$.

                We can start with the largest number using the digits $0$ through $9$
                $(\num{9876543210})$ and modify it as necessary.
                Since we want the result to be as large as possible, we should keep as
                many of the leading digits as possible and try to only change the last few digits.
                Currently, the alternating digit sum is $9 - 8 + 7 - 6 + 5 - 4 + 3 - 2 + 1 - 0
                = 5$, so we have to increase or decrease it to a multiple of $11$.
                We can't do this by only changing the last four digits, since this can only
                change the alternating digit sum by $2$.
                Therefore, we look at the last five digits.
                If we want to increase the alternating sum, we should increase the digits
                $3$ and $1$, since those are the digits that are being added to the digit sum
                as opposed to being subtracted.
                If we put $3$ and $4$ in those positions, this would make the alternating digit
                sum $11$.
                Note that there is no other way to make the alternating digit sum a multiple of
                $11$ while only changing the last five digits.
                We must put $4$ in the second-to-last position and $2$ in the third-to-last
                position, since this is the only way to make the last three digits a multiple of
                $8$.
                The leftover digit must then be $1$, so the number is $\boxed{\num{9876513240}}$.
            \end{solution}
    \item If a $5$ digit number $5DDDD$ is divisible by $6$, then find the digit $D$.
        \begin{solution}
            First of all, $D$ must be even.
            Next, in order for the number to be divisible by $6$, it must be a multiple of $3$,
            so the sum of the digits must be a multiple of $3$.
            The sum of the digits is $5 + 4D$.
            If $D = 1$, then this is a multiple of $3$, but $D$ must be even.
            Every time we increase $D$ by $1$, the remainder when dividing $5 + 4D$ by $3$
            increases by $1$, so we must increase $D$ by $3$ in order for $5 + 4D$ to still be a
            multiple of $3$.
            Therefore, $D = \boxed{4}$
        \end{solution}
    \item Find the product of the GCD and LCM of $100$ and $120$.
        \begin{solution}
            The LCM of two numbers is the product divided by the GCD,
            Therefore the product of the GCD and the LCM is the same as the product of the two
            numbers.
            So the answer is $100 \cdot 120 = \boxed{12000}$.
        \end{solution}
    \end{enumerate}
\end{multicols}
\end{document}
