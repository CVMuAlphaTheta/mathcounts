\documentclass[twocolumn]{article}

\usepackage[margin=0.5in]{geometry}
\usepackage{flushend}
\usepackage{amsthm}
\usepackage{amsmath}
\usepackage[group-separator={,}]{siunitx}

\theoremstyle{definition}
\newtheorem*{solution}{Solution}

\title{Prime and Composite Numbers Set B}
\date{}
\author{}

\begin{document}
\maketitle
\begin{enumerate}
    \item What is the smallest four-digit number that is divisible by $2, 3, 4, 5,
        6, 8, 9, \textrm{and} 10$?
        \begin{solution}
            The number must end in a zero. Let's assume that to be the smallest, it should
            start with a $1$. Since the digits must add up to $9$, the last three digits
            must add up to $9$. $1080$ is the smallest four-digit integer divisible
            by $2, 3, 4, 5, 6, 8, 9, \text{ and } 10$. \\
            Consider the prime factorization of the result. You need a $2$ for divisibility
            by $2$, a $3$ for divisibility by $3$, another $2$ for divisibility by $4$,
            and a $5$ for divisibility by $5$. The number is already divisible by $6$.
            Another $2$ must be included for divisibility by $8$, and another $3$ for
            divisibilty by $9$. The resulting number is already divisible by $10$.
            That gives us $2^3 \cdot 3^2 \cdot 5 = 360$. We were asked for a four-digit
            integer, so we need to multiply $360$ by $3$ to get $1080$.
        \end{solution}
    \item Using only $1$'s and $2$'s, what is the smallest integer you can create
        which is divisible by both $3$ and $8$?
        \begin{solution}
            The units digit must be $2$. Since $12$ and $22$ don't work, we need a
            $3$-digit number that is divisible by $8$. $\frac{112}{8} = 14$, but it's
            not divisible by $3$. Unfortunately, $122$, $212$, and $222$ are not divisible
            by $8$. We must go to a $4$-digit number that ends in $112$ and whose digit
            sum is a multiple of $3$. The only $4$-digit number that works is $2112$.
        \end{solution}
    \item Each of the digits $0$ through $9$ is used exactly once to create a ten-digit
        integer. Find the greatest ten-digit number which uses each digit once and is
        divisible by $8, 9, 10, \textrm{and} 11$.
        \begin{solution}
            Start with the largest problem possible \\
            $(\num{9876543210})$ and modify it as necessary. It is already divisible by
            $9$ and $10$, but not by $8$ and $11$. For $11$, the digit sum is $45$, so
            the sum of alternating digits cannot be equal. The difference must be $11$
            with alternating digit sums of $17$ and $28$. $\num{9876543210}$ has alternating
            digit sums of $25$ and $20$, so we must increase the digit sum of $9 + 7 + 5
            + 3 + 1$ to $28$. Trading the $1$ for a $4$ is the best way to do this. Leave
            the largest five digits in order of the left and the zero on the right. Look
            for arrangements of the underlined digits which make the number divisible by
            $8$ while maintaining the alternating digit sums. $98765\underline{24310}$
            is not divisible by $8$, neither is $98765\underline{23140}$, but
            $\num{9876513240}$ is.
        \end{solution}
    \item What is the sum of the unique prime factors of $9991$?
        \begin{solution}
            To find the prime factorization of a number, we can systematically try to divide
            by primes in the order of the primes. If we get to the square root of a number
            without finding a prime factor, then we can be sure that the number is prime and
            we can discontinue the search. The square root of $9991$ is about $99.95$, so we
            need to check all the $25$ primes less than $99$. They are $2$, $3$, $5$, $7$, $11$,
            $13$, $17$, $19$, $23$, $29$, $31$, $37$, $41$, $43$, $47$, $53$, $59$, $61$, $67$,
            $71$, $73$, $79$, $83$, $89$ and $97$. For some of these primes, we have divisibility
            rules, so we can quickly eliminate $2$, $3$, and $5$ as factors. For most primes,
            we simply divide by that number and see if the quotient is a whole number. It turns
            out that $97$ is a divisor of $9991$ and the quotient is $103$, which is also prime.
            Thereforece, the sum of the unique prime factors of $9991$ is $97 + 103 = 200$.
        \end{solution}
    \item What is the largest prime factor $8 \cdot 7 \cdot 6 \cdot 5 \cdot 4 \cdot 3 -
        6 \cdot 5 \cdot 4 \cdot 3 \cdot 2 \cdot 1$?
        \begin{solution}
            The expression is the difference of two products. We can factor $6 \cdot 5 \cdot 4
            \cdot 3 \cdot 2$ from each product to get $6 \cdot 5 \cdot 4 \cdot 3 \cdot 2 \cdot
            (4 \cdot 7 - 1) = 6 \cdot 5 \cdot 4 \cdot 3 \cdot 2 \cdot (28 - 1) = 6 \cdot 5 \cdot 4
            \cdot 3 \cdot 2 \cdot 27$. Since $27 = 3 \cdot 3 \cdot 3$ and $6 = 3 \cdot 2$, the
            largest prime factor of this product is $5$. Alternatively, calculating the products
            and finding the diffrence, we get $\num{20160} - 720 = \num{19440} = 24 \cdot 35 \cdot 5$.
            So, the greatest prime factor is $5$.
        \end{solution}
\end{enumerate}
\end{document}
