\documentclass{article}

\usepackage[margin=0.5in]{geometry}
\usepackage{multicol}
\usepackage{amsthm}

\theoremstyle{definition}
\newtheorem*{solution}{Solution}
\title{Problem-Solving Set A}
\author{}
\date{}

\begin{document}
\maketitle
\begin{multicols}{2}
    \begin{enumerate}
        \item Happy Trails Cycles have a collection of unicycles, bicycles, and tricycles in stock, with at least one of each item in inventory.
            If there are a total of $73$ wheels on the collection of unicycles, bicycles, and tricycles, what is the maximum possible number of bicycles in the shop's inventory?
            \begin{solution}
                We know there are $73$ tires on the unicycles, bicycles, and tricycles in inventory.
                To maximize the number of bicycles, we need to minimize the number of unicycles and tricycles.
                Assuming there is only one of each accounts for $1 + 3 = 4$ tires and leaves $73 - 4 = 69$ tires.
                This could be from $1$ unicycle and the maximum of $\frac{69 - 1}{2} = 34$ bicycles.
            \end{solution}
        \item Ali's middle school has a total of $300$ students on Team A and Team B.
            After $30$ students are moved from Team A to Team B, there are twice as many students on Team A as there are on Team B.
            How many students were originally on Team B?
            \begin{solution}
                After $30$ students move from Team A to Team B, twice as many students are on Team A as on Team B.
                So, Team A has $\frac{2}{3} \cdot 300 = 200$ students, and Team B has $\frac{1}{3} \cdot 300 = 100$ students.
                That means the number of students originally on Team B was $100 - 30 = 70$ students.
            \end{solution}
        \item Marco's age is between $20$ and $60$.
            The square of his age is a four-digit number, and the sum of the digits of his age is $8$.
            How old is Marco if his age has two different digits and is not prime?
            \begin{solution}
                We are initially told that Marco is older than $20$ but younger than $60$.
                Reading on, we see that Marco must be at least $32$ years old in order for the square of his age to be a four-digit number.
                But the sum of the digits of his age is $8$, resulting in three possibilities for Marco's age: $35$, $44$, and $53$.
                Since we are told that the digits of his age are different and his age is not prime, that rules out $44$ and $53$ as possible ages.
                So, we conclude that Marco is $35$ years old.
            \end{solution}
        \item Nya starts with an integer $N$ and repeatedly subtracts $6$.
            Mya starts with an integer $M$ and repeatedly adds $8$.
            When Nya and Mya have each performed their respective operation $13$ times, both have a resulting value of $25$.
            What is the value of $N + M$?
            \begin{solution}
                For Nya, $N - 6 \cdot 13 = 25$, so $N = 25 + 78 = 103$.
                For Mya, $M + 8 \cdot 13 = 25$, so $M = 25 - 104 = -79$.
                Thus, $N + M = 103 + (-79) = 24$.
                Alternatively, the sum of Mya and Nya's numbers increases by $8 - 6$ = 2 at each step for a total increase of $13 \cdot 2 = 26$.
                The sum of their numbers ends at $25 + 25 = 50$, so the sum would have started at $50 - 26 = 24$.
            \end{solution}
        \item Andrew mowed one-half of a lawn, and Ben mowed one-third of the same lawn, each at a constant rate.
            If Andrew, continuing at the same rate, finished mowing the rest of the lawn in $12$ minutes, how many minutes would it have taken to mow the entire lawn by himself?
            \begin{solution}
                Andrew and Ben mowed $\frac{1}{2} + \frac{1}{3} = \frac{3}{6} + \frac{2}{6} = \frac{5}{6}$ of the lawn, and then Andrew mowed the remaining $1 - \frac{5}{6} = \frac{1}{6}$ of the lawn in $12$ minutes.
                To mow the entire lawn by himself, at that rate, would have taken Andrew $12 \cdot 6 = 72$ minutes.
            \end{solution}
    \end{enumerate}
\end{multicols}
\end{document}