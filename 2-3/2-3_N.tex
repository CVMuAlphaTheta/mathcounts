\documentclass[twocolumn]{article}

\usepackage{amsmath}
\usepackage[margin=0.5in]{geometry}
\usepackage{flushend}
\usepackage{enumitem}
\usepackage[group-separator={,}]{siunitx}

\title{Combinatorics Notes}
\author{}
\date{}

\begin{document}
\maketitle

\section*{Concepts}

\subsection*{Factorials}
Let's say we wanted to count the number of ways to order $7$ letters. We can 
count the number of ways to choose the letter in each position. There are $7$ 
ways to choose the first letter, $6$ ways to choose the second letter (since we 
can't repeat the first letter), $5$ ways to choose the third letter, and so on. 
Therefore, the number of ways to order $7$ objects is $7 \cdot 6 \cdot 5 \cdot 4 
\cdot 3 \cdot 2 \cdot 1$. In general, the number of ways to order $n$ objects is 
$n \cdot (n - 1) \cdot (n - 2) \cdot \dots \cdot 2 \cdot 1$. It turns out that 
this is very useful in combinatorics, so there's a special notation for it. The 
\textbf{factorial} of a non-negative integer $n$ is written as $n!$ and it is 
equal to $n \cdot (n - 1) \cdot (n - 2) \cdot \dots \cdot 2 \cdot 1$.

The factorial of $0$ is defined as $1$. To see why, notice that to get from $3!  
= 6$ to $2! = 2$, we divide by $3$, and to get from $2!$ to $1!$, we divide by 
$2$. Therefore, to get from $1!$ to $0!$, we divide by $1$, so $0! = 
\frac{1!}{1} = 1$.

\subsection*{Permutations}
Let's say instead of counting the number of ways to order $7$ letters, we want 
to count the number of ways to pic $3$ letters out of $7$ where the order in 
which we pick the letters matters. In other words, we want to count the number 
of ways to pick $3$ letters out of $7$ and then arrange them in some order. 
There are $7$ ways to choose the first letter, $6$ ways to choose the second 
letter, and $5$ ways to choose the third letter, so the total is $7 \cdot 6 
\cdot 5$. In general, the number of ways to choose $k$ objects out of a total of 
$n$ objects where order matters is $n \cdot (n - 1) \cdot (n - 2) \cdot \dots 
\cdot (n - k + 1)$. We can use factorials to write this in a more simple way. If 
we have $\frac{n!}{(n - k)!}$, then the $(n - k)!$ cancels out with the last $n 
- k$ terms of $n!$, so we are left with the first $k$ terms of $n!$. For 
example, $\frac{7!}{(7 - 3)!} = \frac{7 \cdot 6 \cdot 5 \cdot 4 \cdot 3 \cdot 2 
\cdot 1}{4 \cdot 3 \cdot 2 \cdot 1} = 7 \cdot 6 \cdot 5$. There are several 
notations for permutations. Some of them are $P^n_k$, $_nP_k$, $^nP_k$, $P_{n, 
k}$, and $P(n, k)$. All of these denote the number of ways to choose $k$ objects 
out of $n$ where order matters, which is $\frac{n!}{(n - k)!}$. $_nP_k$ and 
$P(n, k)$ tends to be more commonly used. Also, each way of choosing $k$ objects 
out of $n$ where order matters is called a \textbf{$k$-permutation of $n$}.

\subsection*{Combinations}
Suppose we want to count the number of ways to pick $3$ letters out of $7$ 
letters, where order doesn't matter. This is the number of $3$-combinations of 
$7$. We can start with $P(7, 3)$, which counts the number of ways to pick $3$ 
objects out of $7$ where order matters. For any group of $3$ different objects, 
we can order them in $3!$ different ways, so $P(7, 3)$ counts each combination 
$3!$ times. Therefore, we divide $P(7, 3)$ by $3!$ to get the number of 
combinations, which is $\frac{7!}{3!(7 - 3)!}$. In general, the number of ways 
to choose $k$ objects out of $n$ where order doesn't matter is $\frac{n!}{k!(n - 
k)}!$, which is just the number of permutations divided by $k!$. The notation 
for combinations are $C^n_k$, $_nC_k$, $^nC_k$, $C_{n, k}$, $C(n, k)$, and 
$\binom{n}{k}$. These can be read as ``$n$ choose $k$'', and they're also called 
\textbf{binomial coefficients} for reasons that we'll discuss later.

\subsection*{Computing Permutations and Combinations}
The formulas for permutations and combinations ($P(n, k) = \frac{n!}{(n - k)!}$ 
and $C(n, k) = \frac{n!}{k!(n - k)!}$) look simple and are useful for proving 
theorems, but you shouldn't use them to actually compute the combinations and 
permutations. The reason is that you can have extremely large numbers which end 
up canceling out. For example, if you tried to compute $C(100, 2)$ using the 
formula, you would have to evaluate $100!$, which is a number with $158$ digits. 
Instead, notice that $\frac{100!}{98!}$ is just $100 \cdot 99$, so 
$\frac{100!}{2!98!} = \frac{100 \cdot 99}{2} = 4950$. You could also think of it 
as having $100$ ways to choose the first object and $99$ ways to choose the 
second object, and dividing by $2$ to account for the $2$ possible orderings of 
each combination. Additionally, instead of multiplying all the numbers in the 
numerator and denominator before dividing them, it's easier to cancel out common 
factors first. For example, if we wanted to compute $C(13, 5)$, we would have 
$\frac{13 \cdot 12 \cdot 11 \cdot 10 \cdot 9}{5 \cdot 4 \cdot 3 \cdot 2}$. $3 
\cdot 4 = 12$ and $5 \cdot 2 = 10$, so those cancel out and we're just left with 
$12 \cdot 11 \cdot 9$.

\subsection*{Letter Arrangement Problems}
How many arrangements of the letters $COUNTING$ are possible? Your first thought 
might be that there are $8$ letters, so it's just $8!$, but notice that two of 
the letters are the same. $8!$ is the number of ways to arrange $8$ different 
letters, such as $COUN_1 TIN_2 G$. Each arrangement that we want is counted 
twice here, since there are two ways to arrange $N_1$ and $N_2$. For example, 
$CN_1 TIN_2 GUO$ and $CN_2 TIN_1 GUO$ are actually the same. Therefore, we 
divide $8!$ by $2! = 2$ to get the final answer. If $3$ of the letters are the 
same, we would divide by $3!$. If we had $AOOOEE$, we would take $6!$, divide it 
by $3!$ to account for the three $O$s, and divide it by $2!$ to account for the 
two $E$s.

\subsection*{Permutations With Restrictions}
How many \emph{even} five-digit numbers contain each of the digits from $1$ to 
$5$? You might try $\frac{5!}{2}$, but this assumes that half of the numbers are 
even, which isn't true because there are more odd digits from $1$ to $5$ than 
even digits. First of all, note that the five-digit numbers containing each of 
the digits from $1$ to $5$ are the same as the five digit numbers made of of the 
digits from $1$ to $5$ without repeats. We might try to count the number of ways 
to choose each of the digits. There are $5$ ways to choose the first digit, $4$ 
ways to choose the second digit, and so on. But when we get to the last digit, 
there will only be one choice left, and it could be odd. The solution to this is 
to start counting the digit with the most restrictions first. If we started from 
the last digit, we see that there are $2$ ways to choose the last digit, $4$ 
ways to choose the second-to last digit, $3$ ways to choose the middle digit, 
$2$ ways to choose the second digit, and $1$ way to choose the first digit, so 
the total is $2 \cdot 4 \cdot 3 \cdot 2 = 48$. \textbf{Counting the thing with 
the most restrictions first is good general strategy.}

\subsection*{Some Number Theory Stuff}
I know there's already a lot this week but for some reason Chloe or Anwar put a 
whole bunch of number theory problems in set B so here are some things which 
might be useful.

\subsubsection*{Factors and Divisibility}
An integer $a$ is a \textbf{factor} of another integer $b$ if $a$ divides $b$ 
evenly. That is, $\frac{b}{a}$ is an integer, and when dividing $b$ by $a$ the 
remainder is zero. For example, the factors of $12$ are $1$, $2$, $3$, $4$, $6$, 
and $12$. If $a$ is a factor of $b$, $b$ is said to be \textbf{divisible} by $a$ 
and $b$ is a \textbf{multiple} of $a$.

\subsubsection*{Primes}
A \textbf{prime} is a number that has exactly two factors: $1$ and itself. It 
cannot be broken down into the product of two smaller numbers. For example, $7$ 
is a prime because there are no two integers which multiply to $7$ other than 
$1$ and $7$. A \textbf{composite} number has more than two factors. $4$ is 
composite because it has three factors: $1$, $2$, and $4$. $1$ is neither prime 
nor composite because it has one factor.

\subsubsection*{Prime Factorization}
The \textbf{prime factorization} of a number is that number expressed as the 
product of primes. For example, the prime factorization of $12$ is $2^2 \cdot 
3$. The \textbf{fundamental theorem of arithmetic} states that each number has a 
unique prime factorization. A number $a$ is a multiple of $b$ if the prime 
factorization of $b$ contains only the primes found in $a$ and the exponent of 
each prime in $b$ is less than or equal to the corresponding exponent in $a$. 
Following this, the factors of a number $a$ with a prime factorization 
$p_1^{k_1} p_2^{k_2} \dots p_n^{k_n}$ are the numbers $p_1^{j_1} p_2^{j_1} \dots 
p_n^{j_n}$ where $0 \leq j_i \leq k_i$ for all $i$. For example, the prime 
factors of $24 = 2^3 \cdot 3$ are $2^0 \cdot 3^0 = 1$, $2^0 \cdot 3^1 = 3$, $2^1 
\cdot 3^0 = 2$, $2^1 \cdot 3^1 = 6$, $2^2 \cdot 3^0 = 4$, $2^2 \cdot 3^1 = 12$, 
$2^3 \cdot 3^0 = 8$, and $2^3 \cdot 3^1 = 24$.
\section*{Problems}

\subsection*{Letter and Number Arrangement Problems}
\begin{enumerate}
	\item How many four letter ``words'' are there if we ignore spelling?
		\vspace{3cm}
	\item How many four-letter ``words'' are there with vowels (not including Y) 
		in the middle two places and a consonant at the end?
		\vspace{3cm}
	\item How many four letter ``words'' are there with vowels in the middle two 
		places and consonants in the other two, with no letter repeated?
		\vspace{3cm}
	\item How many odd numbers with third digit $5$ are there between $20000$ 
		and $69999$ inclusive?
		\vspace{3cm}
	\item How many five-digit numbers ending with $1$, $2$, or $4$ are there 
		with no digit repeated?
		\vspace{3cm}
\end{enumerate}

\subsection*{Counting Divisors}
\begin{enumerate}[resume]
	\item How many divisors does $540$ have?
		\vspace{3cm}
	\item How many divisors does $\num{1000000}$ have?
		\vspace{3cm}
\end{enumerate}

\subsection*{Permutations}
\begin{enumerate}[resume]
	\item In how many ways can $n$ people be seated in a row of $n$ chairs?
		\vspace{3cm}
	\item In how many ways can a row of $k$ seats be filled from a set of $n$ 
		people?
		\vspace{3cm}
	\item In how many ways can $6$ different keys be arranged on a key chain?
		\vspace{3cm}
\end{enumerate}
\end{document}
