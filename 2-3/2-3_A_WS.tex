\documentclass[twocolumn]{article}

\usepackage[margin=0.5in]{geometry}
\usepackage{flushend}
\usepackage{amsmath}
\usepackage{amsthm}

\theoremstyle{definition}
\newtheorem*{solution}{Solution}

\title{Combinatorics Set A}
\author{}
\date{}

\begin{document}
\maketitle
\begin{enumerate}
    \item In how many different ways can a student guess a complete set of 
        answers to a five-item true/false quiz?
        \begin{solution}
            For each question there are $2$ choices, so for all five there are 
            $2^5 = 32$ choices by the Fundamental Counting Principle.
        \end{solution}
    \item How many ways can $5$ books be arranged on a shelf if $2$ of the books 
        must remain together?
        \begin{solution}
            The key here is to realize that the books which must be together may 
            be treated as a single unit. Within that unit, the books may be in 
            $2$ different orders (AB or BA) but we can account for that by 
            simply multiplying by $2$ at the end. We thus have $4$ books (one is 
            our two-book unit) and these can be arranged in $4! = 24$ ways.  
            Multiplying by $2$ then yields $48$. (Make sure in your mind that 
            all configurations are counted, and that none are counted twice!)
        \end{solution}
    \item If $n = 100$, find the value of $\frac{(n+1)!}{(n-1)!}$.
        \begin{solution}
            We are asked to evaluate $\frac{101!}{99!}$. Remember that $101! = 
            101 \cdot 100 \cdot 99!$, so $\frac{101!}{99!} = 101 \cdot 100 = 
            10100$.
        \end{solution}
    \item Andrew and Lindsey are seated at a round table with four other people.  
        If everyone is randomly seated, what is the probability that Andrew and 
        Lindsey are seated next to each other? Express your answer as a common 
        fraction.
        \begin{solution}
            If these six people were standing in a line, then there would be $6!  
            = 6 \cdot 5 \cdot 4 \cdot 3 \cdot 2 \cdot 1 = 720$ orders. Since 
            this is a round table, these same arrangements of people can be 
            rotated in $6$ different places, so there would seem to be only 
            $\frac{720}{6} = 120$ seating arrangements. Since we are interested 
            in seating arrangements with Andrew and Lindsey seated together, 
            let's imagine that Andrew and Lindsey are stuck together and act as 
            a single person. Then we have the same situation with five people, 
            so there are $\frac{5!}{5} = \frac{120}{5}  = 24$ seating 
            arrangements.  However, we can glue Andrew and Lindsey together in 
            $2$ different ways, so there are actually $2 \cdot 24 = 48$ seating 
            arrangements with Andrew and Lindsey seated together.  Therefore, 
            the probability that they are seated together is $\frac{48}{120} = 
            \frac{2}{5}$.

            Another way to do this problem would be to imagine randomly 
            assigning a seat to each person one-by-one. If we assign a seat to 
            Andrew first, there will be $5$ seats left over, $2$ of which are 
            next to him. Therefore, if we randomly pick a seat for Lindsey, the 
            probability which they will be together is $\frac{2}{5}$.
        \end{solution}
    \item What is the probability of getting exactly $6$ correctly answers on a 
        $10$-question true-false test by randomly guessing? Express your answer 
        as a common fraction.
        \begin{solution}
            Each of the $10$ answers has an equal probability of being true and 
            false, so there are $2^{10} = 1024$ possible outcomes in total.  
            There are ``$10$ choose $6$'' or $\binom{10}{6} = \frac{10!}{4!  
            \cdot 6!} = \frac{10 \cdot 9 \cdot 8 \cdot 7}{4 \cdot 3 \cdot 2 
            \cdot 1} = 10 \cdot 3 \cdot 7 = 210$ ways to get exactly $6$ correct 
            answers.  So, the probability is $\frac{210}{1024} = 
            \frac{105}{512}$.
        \end{solution}
\end{enumerate}
\end{document}
