\documentclass{article}

\title{2.1 B Solutions}
\author{}
\date{}

\begin{document}
\maketitle

\begin{enumerate}
    \item Since the average is the sum divided by the number of students, the 
        sum is the average multiplied by the number of students. Therefore, the 
        sum of the scores of the first group of students is $25 \cdot 84 = 
        2100$, and the sum of the scores of the second group of students is $20 
        \cdot 66 = 1320$. The sum of all of the scores is then $2100 + 1320 = 
        3420$, and the average is $\frac{3420}{25 + 20} = 76$.
    \item Following the hint, we'll count the probability that none of the dice 
        comes up with a $5$ or a $6$. In other words, we're finding the 
        probability that all of the dice comes up with a number from $1$ to $4$.  
        The probability of one dice coming up with a number from $1$ to $4$ is 
        $\frac{4}{6} = \frac{2}{3}$, since there are $6$ total outcomes and $4$ 
        of them are from $1$ to $4$. Since the dice are separate and have no 
        influence on each other, the result of each of the dice are independent.  
        Therefore, the probability that all of them result in a number from $1$ 
        to $4$ is equal to the product of the probabilities that each of the 
        dice individually produce a number from $1$ to $4$. It follows that the 
        probability that none of the dice produce $5$ or $6$ is 
        $\left(\frac{2}{3}\right)^3 = \frac{8}{27}$. The event that at least one 
        of the dice comes up with $5$ or $6$ is the complement of the event that 
        none of the dice comes up with $5$ or $6$, so the answer is $1 - 
        \frac{8}{27} = \frac{19}{27}$.
    \item We consider two cases: the case where it rains and we can go outside, 
        and the case where it doesn't rain and we can go outside. These two 
        events are mutually exclusive because it either rains or doesn't rain, 
        so we can't have both at the same time. Therefore, the probability that 
        either of those cases occur is the sum of the probability of each of 
        those events. The probability that it rains and we can go outside is the 
        product of the probability that it rains and the probability that we can 
        go outside given that it rains, which is $0.2 \cdot 0.1 = 0.02$. The 
        probability that it doesn't rain and we can go outside is the product of 
        the probability that it doesn't rain and the probability that we can go 
        outside given that it doesn't rain, which is $0.8 \cdot 0.8 = 0.64$.  
        Adding these up, we get $0.66$ as the probability that we can go 
        outside.
    \item There are $10$ ways of getting exactly $9$ heads, since there are $10$ 
        ways to choose the flip that results in tails. There is $1$ way of 
        getting $10$ heads. Therefore there are $11$ ways of getting at least 
        $9$ heads, out of a total of $2^{10}$ outcomes, so the probability is 
        $\frac{11}{2^{10}} = \frac{11}{1024}$.
    \item In this stem-and-leaf plot, the digit on the left of the bar is the 
        first digit of all of the numbers represented by that row, and each 
        digit on the right of the bar is the second digit of a number.  
        Therefore, the heights of the players are $68$, $68$, $69$, $69$, $71$, 
        $72$, $77$, $77$, $78$, $78$, $80$, and $81$. The average height is 
        therefore $74$ inches.
\end{enumerate}
\end{document}
