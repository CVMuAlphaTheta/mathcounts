\documentclass[twocolumn]{article}

\usepackage[margin=0.5in]{geometry}
\usepackage{flushend}
\usepackage{enumitem}
\usepackage{amsmath}

\newcommand{\pand}{\mathop\text{and}}
\newcommand{\por}{\mathop\text{or}}

\title{2.1 Notes}
\author{}
\date{}

\begin{document}
\maketitle

\section*{Key Concepts}

\subsection*{Statistics}

\subsubsection*{Mean}
The \textbf{mean} of a set of numbers is their sum divided by the number of
numbers. For example, the mean of $1$, $2$, and $3$ is $\frac{1 + 2 + 3}{3}
= 2$. A useful property of the mean is that the product of the mean and the
number of numbers is the sum.

\subsubsection*{Median}
To find the \textbf{median} of a set of numbers, first list them in
non-decreasing order. If there are an odd number of numbers, the median is
simply the number in the middle. If there are an even number of numbers, the
median is the average (mean) of the two numbers in the middle. For example,
the median of $1$, $3$, $8$, and $9$ is $\frac{3 + 8}{2} = \frac{11}{2}$ and
the median of $5$, $8$, $1$, $2$, and $7$ is $5$.

\subsubsection*{Mode}
The \textbf{mode} of a set of numbers is the most frequent number, if this
number is unique. The mode of $1$, $3$, $3$, $4$, $5$, $5$, $5$, and $6$, is
$5$.

\subsubsection*{Range}
The \textbf{range} is simply the absolute difference between the minimum
number and the maximum number.

\subsection*{Probability}
In the context of probability, an \textbf{experiment} is a process that
produces a random outcome, such as a coin toss or a dice roll. The
\textbf{sample space} is the set of possible outcomes, such as $\{\, 1, 2,
3, 4, 5, 6 \,\}$ in the case of a dice roll. An \textbf{event} is a property
of a subset of the \textbf{sample space}. An event for a dice roll could be
that the result is even, or that the result is less than $3$ for example. An
event is said to occur if the outcome has that property.

The \textbf{probability} of an event or outcome is a real number between $0$
and $1$ which describes how likely it is to occur. The greater the
probability, the more likely it is for an event to occur. An event with a
probability of $0$ never occurs and an evert with a probability of $1$
always occurs. You can think of the probability as the average frequency of
an event. An event with a probability of $\frac{1}{2}$ occurs one in two
times on average, and an event with a probability of $\frac{3}{4}$ occurs
three in four times on average. The probability of an event $E$ is written
as $P(E)$. The probability of all the outcomes in a sample space always add
up to $1$. The probability of an event is the sum of the probabilities of
the outcomes which it includes. When the outcomes of an experiment are all
equally likely, this is equal to the number of outcomes in that event
divided by the total number of possible outcomes. The probability of a dice
roll resulting in a number less than $5$ is $\frac{2}{3}$, because the
outcomes are equally likely, there are $4$ outcomes less than $5$, and there
are $6$ total possible outcomes.

\subsubsection*{Compound Events}
The \textbf{complement} of an event is the opposite of it. The complement of
an event $E$ occurs when $E$ doesn't occur. Going back to the dice roll
example, the complement of ``less than $5$'' is ``greater than or equal to
$5$'' and the complement of ``odd'' is ``even''. The complement of $E$ is
written as $\overline{E}$. The probability of the complement of $E$ is one
minus the probability of $E$, so $P(\overline{E}) = 1 - P(E)$.

Given two events $E_1$ and $E_2$, their \textbf{union} is defined as ``$E_1$
or $E_2$''. That is, the union of $E_1$ and $E_2$ occurs if either $E_1$,
$E_2$, or both, occur. This definition extends to more than two events.  In
a dice roll, the union of the events ``even'' and ``less than $5$'' is
``even or less than $5$'', which includes anything other than $5$.

The union of $E_1$ and $E_2$ is written as $E_1 \por E_2$, or $E_1 \cup E_2$.
The \textbf{intersection} of events $E_1$ and $E_2$ is the event that both $E_1$
and $E_2$ occur. The intersection of $E_1$ and $E_2$ is written as $E_1 \pand
E_2$, or $E_1 \cap E_2$.

\subsubsection*{Independent Events}
Two events are said to be \textbf{independent} if the outcome of one of the
events does not affect the probability of the other event. If I first flip a
coin and then roll a die, knowing that the coin flip resulted in heads doesn't
change the probability that the dice roll will result in an odd number, so these
events are independent. The event of a dice roll resulting in a number greater
than or equal to $3$ is not independent from the event of the same dice roll
resulting in a number greater than or equal to $4$, because if we know that it
is greater than or equal to $3$, it's more likely to be greater than or equal to
$4$. Note that events from separate processes are always independent, but
independent events are not necessarily from separate processes. For example, the
event that a dice roll results in an even number is independent from the event
that the same dice roll results in a number less than $4$.

When two events are independent, the probability of their intersection is the
product of their probabilities. That is, if $E_1$ and $E_2$ are independent,
$P(E_1 \pand E_2) = P(E_1)P(E_2)$. The probability of the union of two events is
the sum of their probabilities minus the probability of their intersection. In
other words, $P(E_1 \por E_2) = P(E_1) + P(E_2) - P(E_1 \pand E_2)$.

\subsubsection*{Mutually Exclusive Events}
Two events are \textbf{mutually exclusive} if they never both occur. In other
words, $E_1$ and $E_2$ are mutually exclusive if $P(E_1 \pand E_2) = 0$. Recall
that $P(E_1 \por E_2) = P(E_1) + P(E_2) - P(E_1 \pand E_2)$. If $E_1$ and $E_2$
are mutually exclusive, then this equation becomes $P(E_1 \por E_2) = P(E_1) +
P(E_2)$, so the probability of the union of two mutually exclusive events is
simply the sum of the probabilities of the two events.

\subsubsection*{Conditional Probability}
When two events are not independent, the outcome of one event influences the 
probability of the other event. \textbf{Conditional probability} measures the 
probability of one event given that another event has occurred. We write the 
probability of an event $B$ given that $A$ has occurred as $P(B \mid A)$. For 
example, let $A$ be the event that a dice roll resulted in a number greater than 
$1$, and $B$ be the event that the same dice roll resulted in a number greater 
than $2$. Let's say we rolled a dice and didn't look at the result. Someone else 
told us that it's greater than $1$. Then the probability that it's greater than 
$2$ is $P(B \mid A)$.

When two events $A$ and $B$ aren't necessarily independent, $P(A \pand B) = 
P(A)P(B \mid A)$. Using the previous example, $P(A) = \frac{5}{6}$ since there 
are five outcomes greater than $1$. $P(B \mid A) = \frac{4}{5}$ since four of 
the five outcomes greater than $1$ are greater than $2$. $A \pand B$ is the 
event that the number is both greater than $1$ and greater than $2$, so it is 
either $3$, $4$, $5$, or $6$. There are four outcomes included in this event, so 
its probability should be $\frac{4}{6} = \frac{2}{3}$. If we use the equation 
above, we also get $\frac{5}{6} \cdot \frac{4}{5} = \frac{4}{6} = \frac{2}{3}$.

\section*{Problems}

\subsection*{Statistics}
\begin{enumerate}
	\item Find the mode, median, arithmetic mean, range, and geometric mean
		of the following numbers: $2, 4, 4, 5, 6, 10$.
		\vspace{3cm}
	\item Two numbers $x$ and $y$ have a geometric mean of $12$ and an
		arithmetic mean of $12.5$. Find $x^2 + y^2$.
		\vspace{3cm}
	\item A student has an average score of $80$ on her first four tests.
		What must she score on the next test to raise her average to $82$?
		\vspace{3cm}
	\item The arithmetic mean of $12$ scores is $82$. When the highest and
		lowest scores are removed, the new mean becomes $84$. If the highest
		of the $12$ scores is $98$, what is the lowest score?
		\vspace{3cm}
	\item What is the average of seven numbers if the average of the first
		two is $9$ and the average of the last $5$ is $16$?
		\vspace{3cm}
\end{enumerate}

\subsection*{Probability}
\begin{enumerate}[resume]
	\item A box contains three red balls and three green balls. Two balls
		are chosen.
		\begin{enumerate}
			\item What is the probability that a red ball is chosen first,
				then a green ball?
				\vspace{3cm}
			\item What is the probability that one red ball and one green
				ball are chosen?
				\vspace{3cm}
		\end{enumerate}
	\item John and Jayne each choose an integer from $1$ to $10$ inclusive.
		What is the probability that they each pick a number greater than
		$7$?
		\vspace{3cm}
	\item If Saphira randomly chooses a $4$-digit number, what is the
		probability that all four digits will be distinct?
		\vspace{3cm}
	\item A Flo Hyman spike puts the ball away $60\%$ of the time. What is
		the probability that the first spike on a given point is returned
		and the second is not?
		\vspace{3cm}
\end{enumerate}
\end{document}
