\documentclass{article}

\usepackage{amsmath}

\title{1.5 Notes}
\author{}
\date{}

\begin{document}
    \maketitle

    \section*{Key Concepts}
    \subsection*{Arithmetic Sequences}
    An \textbf{arithmetic sequence}, also known as an \textbf{arithmetic
    progression}, is a sequence of numbers where the difference between
    consecutive terms are equal. For example, $2, 5, 8, 11, \dots$ is an
    arithmetic sequence because each term is $3$ more than the term before it.
    The difference between consecutive terms in an arithmetic sequence is called
    the \textbf{common difference}. If this difference is negative, than the
    sequence is decreasing. To get from one term to the next, we add the common
    difference. To get from one term to the previous term, we subtract the
    common difference.

    To find the terms of an arithmetic sequence, we can use the formula
    \[a_n = a_m + (n - m)d\] $a_n$ is the $n$th term, $a_m$ is the $m$th term,
    and $d$ is the common difference. You should be able to see how this formula
    corresponds to the definition above. To get from the $m$th term to the $n$th
    term, we need to add the common difference $n - m$ times. We can use this
    formula so solve problems involving the terms of arithmetic sequences. For
    example, if we know that the second term of a sequence is $3$ and the common
    difference is $2$, the fifth term would be $3 + (5 - 2)2 = 9$. We can also
    solve for the value of $d$ if we know two of the terms.

    \subsection*{Arithmetic Series}
    A \textbf{series} is just the sum of the terms of a sequence. $2, 5, 8, 11$
    is an arithmetic sequence, so $2 + 5 + 8 + 11$ is an arithmetic series. We
    can find the sum of an arithmetic series by computing each term of the
    sequence and adding them all up, but that can take a long time when there
    are a large number of terms. It turns out that there is a simple formula
    that can be used to find the sum of an arithmetic series:
    \[S = \frac{n(a_1 + a_n)}{2}\] $S$ is the sum of the arithmetic series
    starting with $a_1$ and ending with $a_n$, or $a_1 + a_2 + \dots + a_n$. $n$
    is the number of terms. We can prove this formula by first writing the
    arithmetic series in two different ways:
    \begin{align*}
        S &= a_1 + (a_1 + d) + (a_1 + 2d) + \dots + (a_1 + (n - 2)d) + a_1 + (n - 1)d) \\
        S &= a_n + (a_n - d) + (a_n - 2d) + \dots + (a_n - (n - 2)d) + (a_n - (n - 1)d)
    \end{align*}
    In the first way, we start with the first term and then repeatedly add $d$
    to obtain successive terms until we reach last term. In the second way, we
    start with the last term and then repeatedly subtract $d$ to obtain
    successive terms until we reach the first term. If we add these two
    together, all the $d$ values cancel out and we get $2S = n(a_1 + a_n)$,
    which means $S = \frac{n(a_1 + a_n)}{2}$.

    We can also substitute $a_n = a_1 + (n - 1)d$ to get a formula for the sum
    in terms of $a_1$, $n$, and $d$:
    \[S = na_1 + \frac{n(n - 1)}{2}d\]

    The formula for the sum of an arithmetic series leads us to some useful
    properties of arithmetic sequences.  The mean of an arithmetic sequence is
    its sum divided by the number of terms, which is $\frac{S}{n}$. Looking at
    the first formula for the sum of an arithmetic series, we see that this is
    equal to $\frac{a_1 + a_n}{2}$. Therefore, the mean of an arithmetic
    sequence is equal to the mean of the first and last terms. If we substitute
    $a_n = a_1 + (n - 1)d$ into $\frac{a_1 + a_n}{2}$, we get $\frac{2a_1 + (n -
    1)d}{2} = a_2 + \frac{n - 1}{2}d$. If there are an odd number of terms, this
    is equal to the term in the middle of the sequence. If there are an even
    number of terms, this is equal to the mean of the two terms in the middle.
    Therefore, the mean of an arithmetic sequence is equal to its median.

    \subsection*{Geometric Sequences}
    A \textbf{geometric sequence} is like an arithmetic sequence, except that
    the ratio between consecutive terms is constant instead of the difference.
    This ratio is called the \textbf{common ratio}. For example, $3, 6, 12, 24,
    \dots$ is a geometric sequence. To get from one term to the next, we
    multiply by the common ratio. To get from one term to the previous term, we
    divide by the common ratio.

    The formula for the terms of a geometric sequence is
    \[a_n = a_mr^{n - m}\] $a_n$ is the $n$th term, $a_m$ is the $m$th term, and
    $r$ is the common ratio.
    \subsection*{Geometric Series}
    To find the sum of a geometric series, we can use the formula
    \[S = a_1\frac{1 - r^n}{1 - r}\]
    The proof of is formula
\end{document}