\documentclass{article}

\usepackage{amsmath}
\usepackage{commath}
\usepackage{graphicx}
\usepackage{hyperref}
\usepackage{enumitem}

\title{1.5 Notes}
\author{}
\date{}

\begin{document}
    \maketitle

    \section*{Key Concepts}
    \subsection*{Arithmetic Sequences}
    An \textbf{arithmetic sequence}, also known as an \textbf{arithmetic
    progression}, is a sequence of numbers where the difference between
    consecutive terms are equal. For example, $2, 5, 8, 11, \dots$ is an
    arithmetic sequence because each term is $3$ more than the term before it.
    The difference between consecutive terms in an arithmetic sequence is called
    the \textbf{common difference}. If this difference is negative, than the
    sequence is decreasing. To get from one term to the next, we add the common
    difference. To get from one term to the previous term, we subtract the
    common difference.

    To find the terms of an arithmetic sequence, we can use the formula
    \[a_n = a_m + (n - m)d\] $a_n$ is the $n$th term, $a_m$ is the $m$th term,
    and $d$ is the common difference. You should be able to see how this formula
    corresponds to the definition above. To get from the $m$th term to the $n$th
    term, we need to add the common difference $n - m$ times. We can use this
    formula so solve problems involving the terms of arithmetic sequences. For
    example, if we know that the second term of a sequence is $3$ and the common
    difference is $2$, the fifth term would be $3 + (5 - 2)2 = 9$. We can also
    solve for the value of $d$ if we know two of the terms.

    \subsection*{Arithmetic Series}
    A \textbf{series} is just the sum of the terms of a sequence. $2, 5, 8, 11$
    is an arithmetic sequence, so $2 + 5 + 8 + 11$ is an arithmetic series. We
    can find the sum of an arithmetic series by computing each term of the
    sequence and adding them all up, but that can take a long time when there
    are a large number of terms. It turns out that there is a simple formula
    that can be used to find the sum of an arithmetic series:
    \[S = \frac{n(a_1 + a_n)}{2}\] $S$ is the sum of the arithmetic series
    starting with $a_1$ and ending with $a_n$, or $a_1 + a_2 + \dots + a_n$. $n$
    is the number of terms. We can prove this formula by first writing the
    arithmetic series in two different ways:
    \begin{align*}
        S &= a_1 + (a_1 + d) + (a_1 + 2d) + \dots + (a_1 + (n - 2)d) + a_1 + (n - 1)d) \\
        S &= a_n + (a_n - d) + (a_n - 2d) + \dots + (a_n - (n - 2)d) + (a_n - (n - 1)d)
    \end{align*}
    In the first way, we start with the first term and then repeatedly add $d$
    to obtain successive terms until we reach last term. In the second way, we
    start with the last term and then repeatedly subtract $d$ to obtain
    successive terms until we reach the first term. If we add these two
    together, all the $d$ values cancel out and we get $2S = n(a_1 + a_n)$,
    which means $S = \frac{n(a_1 + a_n)}{2}$.

    We can also substitute $a_n = a_1 + (n - 1)d$ to get a formula for the sum
    in terms of $a_1$, $n$, and $d$:
    \[S = na_1 + \frac{n(n - 1)}{2}d\]

    The formula for the sum of an arithmetic series leads us to some useful
    properties of arithmetic sequences.  The mean of an arithmetic sequence is
    its sum divided by the number of terms, which is $\frac{S}{n}$. Looking at
    the first formula for the sum of an arithmetic series, we see that this is
    equal to $\frac{a_1 + a_n}{2}$. Therefore, the mean of an arithmetic
    sequence is equal to the mean of the first and last terms. If we substitute
    $a_n = a_1 + (n - 1)d$ into $\frac{a_1 + a_n}{2}$, we get $\frac{2a_1 + (n -
    1)d}{2} = a_2 + \frac{n - 1}{2}d$. If there are an odd number of terms, this
    is equal to the term in the middle of the sequence. If there are an even
    number of terms, this is equal to the mean of the two terms in the middle.
    Therefore, the mean of an arithmetic sequence is equal to its median.

    The sums of a arithmetic series which start with $1$ and have common
    difference $1$ are called \textbf{triangular numbers}. Some examples are $1
    + 2 + 3 = 6$ and $1 + 2 + 3 + 4 + 5 = 15$. These numbers are called
    triangular numbers because the $n$th triangular number is the number of dots
    needed to form an triangle pattern of size $n$:
    \begin{center}
        \includegraphics[width=6cm]{triangular_numbers}

        By Melchoir - Own work, CC BY-SA 3.0,
        \url{https://commons.wikimedia.org/w/index.php?curid=16942149}
    \end{center}
    We can also define the $n$th triangular number as the sum of the first $n$
    positive integers. The formula for triangular numbers is
    \[T_n = \frac{n(n + 1)}{2}\] which is just a special case of the arithmetic
    series sum formula.

    \subsection*{Geometric Sequences}
    A \textbf{geometric sequence} is like an arithmetic sequence, except that
    the ratio between consecutive terms is constant instead of the difference.
    This ratio is called the \textbf{common ratio}. For example, $3, 6, 12, 24,
    \dots$ is a geometric sequence. To get from one term to the next, we
    multiply by the common ratio. To get from one term to the previous term, we
    divide by the common ratio.

    The formula for the terms of a geometric sequence is
    \[a_n = a_mr^{n - m}\] $a_n$ is the $n$th term, $a_m$ is the $m$th term, and
    $r$ is the common ratio.

    \subsection*{Geometric Series}
    To find the sum of a geometric series, we can use the formula
    \[S = a_1 \frac{1 - r^n}{1 - r}\] Here's the proof of the above formula:
    \begin{align*}
        S &= a_1 + a_2 + \dots + a_n \\
        Sr &= a_1 r + a_2 r + \dots + a_n r \\
        Sr &= a_2 + a_3 + \dots + a_{n + 1} \\
        S - Sr &= a_1 - a_{n + 1} \\
        S(1 - r) &= a_1 - a_{n + 1} \\
        S &= \frac{a_1 - a_{n + 1}}{1 - r} \\
        S &= \frac{a_1 - a_1 r^n}{1 - r} \\
        S &= a_1 \frac{1 - r^n}{1 - r}
    \end{align*}

    So far, we've only looked at series with finite numbers of terms. But it
    turns out that a geometric series which goes on forever can actually have a
    finite sum. For example, consider $1 + \frac{1}{2} + \frac{1}{4} +
    \frac{1}{8} + \dots$. This is a geometric series with common ratio
    $\frac{1}{2}$. If you computed $1 + \frac{1}{2}$, $1 + \frac{1}{2} +
    \frac{1}{4}$, $1 + \frac{1}{2} + \frac{1}{4} + \frac{1}{8}$, and so on, you
    will notice that these sums get closer and closer to $2$. We say that the
    sum of this infinite geometric series is equal to $2$ since as we add on
    more terms the sum gets infinitely closer to $2$. The formula for the sum of
    an infinite geometric series is quite simple:
    \[S = \frac{a_1}{1 - r}\] Here's the proof, which is similar to proof for
    the finite geometric series sum formula:
    \begin{align*}
        S &= a_1 + a_2 + a_3 + \dots \\
        Sr &= a_1 r + a_2 r + a_3 r + \dots \\
        Sr &= a_2 + a_3 + a_4 + \dots \\
        S - Sr &= a_1 \\
        S(1 - r) &= a_1 \\
        S &= \frac{a_1}{1 - r}
    \end{align*}
    Not all infinite geometric series have finite sums. The series $1 + 2 + 4 +
    8 + \dots$ clearly grows infinitely. We say that the series
    \textbf{converges} if it has a finite sum and we say that it
    \textbf{diverges} otherwise. An infinite geometric series with common ratio
    $r$ has a finite sum if and only if $\abs{r} < 1$. To see why, consider what
    happens to the finite geometric series sum as we increase $n$. If $\abs{r} <
    1$, then $r^n$ will approach $0$ as $n$ gets larger and larger, so we end up
    with the infinite geometric series sum formula.

    \subsection*{Geometric Mean}
    The word ``mean'' usually refers to the \textbf{arithmetic mean}, but
    there's also something called the \textbf{geometric mean}. The geometric
    mean of the $n$ numbers $x_1, x_2, \dots, x_n$ is $\sqrt[n]{x_1 x_2 \dots
    x_n}$.

    \subsection*{Inserting Means}
    ``Insert $n$ arithmetic means between $a$ and $b$'' means we need to insert $n$
    numbers $x_1, x_2, \dots, x_n$ between $a$ and $b$ such that $a, x_1, x_2,
    \dots, x_n, b$ forms an arithmetic sequence. Similarly, ``insert $n$
    geometric means between $a$ and $b$'' means to insert $n$ numbers $x_1, x_2,
    \dots, x_n$ between $a$ and $b$ such that $a, x_1, x_2, \dots, x_n, b$ forms
    an geometric series. For example, inserting three arithmetic means between
    $1$ and $9$ results in $1, 3, 5, 7, 9$, and inserting two geometric
    means between $1$ and $8$ results in $1, 2, 4, 8$. We can find the terms
    that we need to insert by using the arithmetic or geometric sequence
    formulas to find the common ratio or difference.

    \section*{Examples}
    \subsection*{Arithmetic Sequences}
    \begin{enumerate}
        \item A child’s mother gives her $10$ cents one day. Every day
        thereafter her mother gives her $3$ more cents than the previous day.
        After $20$ days, how much does she have?
        \vspace{3cm}
        \item Prove that
        \[a + (a + d) + (a + 2d) + \dots + (a + (n - 1)d = \frac{n}{2}(2a + (n -
        1)d)\]
        \vspace{3cm}
        \item If the sum of the first ten terms of an arithmetic progression is
        four times the sum of the first five terms, find the ratio of the first
        term to the common difference.
        \vspace{3cm}
        \item What is the sixth term of the arithmetic sequence whose $31$st and
        $73$rd terms are $18$ and $46$, respectively?
        \vspace{3cm}
        \item Insert three arithmetic means between $3$ and $4$.
        \vspace{3cm}
    \end{enumerate}
    \subsection*{Geometric Sequences}
    \begin{enumerate}[resume]
        \item Prove that
        \[a + ar + \dots + ar^{n - 1} = \frac{a - ar^n}{1 - r}\]
        \vspace{3cm}
        \item Evaluate $2 - \frac{2}{3} + \frac{2}{9} - \frac{2}{27} +
        \frac{2}{81}$ using the finite geometric series sum formula.
        \vspace{3cm}
        \item For what value of $x$ does $1 + x + x^2 + x^3 + x^4 + \dots = 4$?
        \vspace{3cm}
        \item Insert two geometric means between $3$ and $4$.
        \vspace{3cm}
    \end{enumerate}
\end{document}