\documentclass{article}

\usepackage{amsmath}

\title{1.5 Notes}
\author{}
\date{}

\begin{document}
    \maketitle

    \section*{Key Concepts}
    \subsection*{Arithmetic Sequences}
    An \textbf{arithmetic sequence}, also known as an \textbf{arithmetic progression}, is a sequence of numbers where the difference between consecutive terms are equal. For example, $2, 5, 8, 11, \dots$ is an arithmetic sequence because each term is $3$ more than the term before it. The difference between consecutive terms in an arithmetic sequence is called the \textbf{common difference}. If this difference is negative, than the sequence is decreasing.

    To find the terms of an arithmetic sequence, we can use the formula
    \[a_n = a_m + (n - m)d\]
    $a_n$ is the $n$th term, $a_m$ is the $m$th term, and $d$ is the common difference. You should be able to see how this formula corresponds to the definition above. To get from the $m$th term to the $n$th term, we need to add the common difference $n - m$ times. We can use this formula so solve problems involving the terms of arithmetic sequences. For example, if we know that the second term of a sequence is $3$ and the common difference is $2$, the fifth term would be $3 + (5 - 2)2 = 9$. We can also solve for the value of $d$ if we know two of the terms.

    \subsection*{Arithmetic Series}
    A \textbf{series} is just the sum of the terms of a sequence. $2, 5, 8, 11$ is an arithmetic sequence, so $2 + 5 + 8 + 11$ is an arithmetic series. We can find the sum of an arithmetic series by computing each term of the sequence and adding them all up, but that can take a long time when there are a large number of terms. It turns out that there is a simple formula that can be used to find the sum of an arithmetic series:
    \[S = \frac{n(a_1 + a_n)}{2}\]
    $S$ is the sum of the arithmetic series starting with $a_1$ and ending with $a_n$, or $a_1 + a_2 + \dots + a_n$. $n$ is the number of terms. We can prove this formula by first writing the arithmetic series in two different ways:
    \begin{align*}
        S &= a_1 + (a_1 + d) + (a_1 + 2d) + \dots + (a_1 + (n - 2)d) + a_1 + (n - 1)d) \\
        S &= a_n + (a_n - d) + (a_n - 2d) + \dots + (a_n - (n - 2)d) + (a_n - (n - 1)d)
    \end{align*}
    In the first way, we start with the first term and then repeatedly add $d$ to obtain successive terms until we reach last term. In the second way, we start with the last term and then repeatedly subtract $d$ to obtain successive terms until we reach the first term. If we add these two together, all the $d$ values cancel out and we get $2S = n(a_1 + a_n)$, which means $S = \frac{n(a_1 + a_n)}{2}$.
\end{document}