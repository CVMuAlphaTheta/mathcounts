\documentclass{article}
\usepackage{amsthm}
\theoremstyle{definition}
\newtheorem*{solution}{Solution}
\title{Fractions, Ratios, and Proportional Reasoning Set B}
\author{}
\date{}
\begin{document}
    \maketitle
    \noindent Problems should be solved without calculators unless otherwise
    specified. Remember to explain how you solved a problem.
    \begin{enumerate}
        \item A MATHCOUNTS club has a total of $27$ sixth- and seventh-grade
        members. If there are $25\%$ more seventh graders than sixth graders,
        how many sixth graders are in the club?
        \begin{solution}
            Let $s$ represent the number of sixth graders. Then we have the
            following equation: $1.25s + s = 27$. Solving for $s$, we get $2.25s
            = 27$ and $s = 12$. So there are $12$ sixth graders.
        \end{solution}
        \item Ms. A owns a house worth $\$10000$. She sells it to Mr. B at
        $10\%$ profit. Mr. B sells the house back to Ms. A at a $10\%$ loss. How
        much money does Ms. A make?
        \begin{solution}
            Since A sees a $10\%$ profit on the first deal, she sells the house
            for $\$10000 \cdot 1.1 = \$11000$. If B then sells his $\$11000$ at
            a loss of $10\%$, the selling price must be $\$11000 \cdot 0.9 =
            \$9900$. Thus, A sells a house for $\$11000$ and buys it back for
            $\$9900$, for a profit of $\$1100$.
        \end{solution}
        \item Before a plum is dried to become a prune, it is $92\%$ water. A
        prune is just $20\%$ water. If only water is evaporated in the drying
        process, how many pounds of prunes can be made with $100$ pounds of
        plums?
        \begin{solution}
            $100$ pounds of plums amounts to $92$ pounds of water and $8$ pounds
            of pulp. When the plums become prunes, the $8$ pounds of pulp stays
            constant but becomes $80\%$ of the total. The other $20\%$ consists
            of $2$ pounds of water. $10$ pounds of prunes can be made with $100$
            pounds of plums.
        \end{solution}
        \item Given $\frac{x}{y} = \frac{2}{3}$ and $\frac{y}{z} = \frac{3}{2}$,
        find $\frac{x}{z}$.
        \begin{solution}
            Multiply the ratios together to get $\frac{x}{y} \cdot \frac{y}{z} =
            \frac{x}{z} = \frac{2}{3} = \frac{3}{2} = 1$.
        \end{solution}
        \item New packaging for fruit snacks contains $10\%$ less weight than
        the original packaging. If the new package costs $15\%$ more than the
        original package, by what fraction did the unit price increase? Express
        your answer as a common fraction.
        \begin{solution}
            The unit price is the cost divided by the weight. Let the original
            unit price be $\frac{c}{w}$. With a $10\%$ reduction in weight and a
            $15\%$ increase in cost, the new unit price is $\frac{1.15c}{0.9w} =
            \frac{1.15}{0.9} \cdot \frac{c}{w} = \frac{23}{18} \cdot
            \frac{c}{w}$. The increase in unit price is $\frac{23}{18} - 1 =
            \frac{23}{18} - \frac{18}{19} = \frac{5}{18}$.
        \end{solution}
    \end{enumerate}
\end{document}