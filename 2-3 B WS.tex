\documentclass{article}
\usepackage{amsthm}
\usepackage{amsmath}

\theoremstyle{definition}
\newtheorem*{solution}{Solution}

\title{Permutations and Combinations Set A}
\author{}
\date{}

\begin{document}
    \maketitle
    \noindent Problems should be solved without calculators unless otherwise specified.
    Remember to explain how you solved a problem.
    \begin{enumerate}
        \item How many factors of $2^{95}$ are there which are greater than $1000000$?
        \begin{solution}
            This is most easily done by counting the wrong thing and subtracting. The number $2^{95}$
            has $95 + 1 = 96$ factors, all of which are powers of $2$. The number of factors
            which are smaller than $1000000$ may be found by finding the largest one which is
            smaller. We have $2^{10} = 1024 > 10^3$, so $2^{20} > 10^6 = 1000000$. But
            $2^{19} = 1024 \cdot 512$ is clearly smaller than $1000000$, so this is the largest.
            The number of factors smaller than a million is thus $19 + 1 = 20$, and there are
            $96 - 20 = 76$ factors greater than a million.
        \end{solution}
        \item What is the units digit of the sum $1! + 2! + 3! + \dots + 14! + 15!$?
        \begin{solution}
            Observe: $1! = 1$, $2! = 2$, $3! = 6$, $4! = 4$, $5! = 120$, $6! = 720 \dots$
            Since $5!$ ends in a zero, all those following it muse also end in 0. Thus the units
            digit of $1 + 2 + 6 + 4 = 13$, or $3$.
        \end{solution}
        \item Numbers that read the same forward and backward are called palindromes. How many
        three-digit numbers are palindromes?
        \begin{solution}
            The first digit can be chosen in $9$ ways, since it can't be $0$. The last digit
            must also be the same as the first. The middle digit can be chosen in $10$ ways.
            The total number is thus $10 \cdot 9 = 90$.
        \end{solution}
        \item What is the value of the expression $\frac{8! - 7 \cdot 7! + 6 \cdot 6!}
        {7! - 6 \cdot 6! + 5 \cdot 5!}$? Express your answer as a common fraction.
        \begin{solution}
            With factorials, we can often do some factoring to simplify things and avoid large
            numbers. We can simplify this particular problem as shown. \\
            Numerator:
            \begin{align*}
                8! \hspace{0.525cm} - 7 \cdot \hspace{0.1cm} 7! \hspace{0.07cm} + 6 \cdot 6! & \\
                \overbrace{8 \cdot 7 \cdot 6!} - 7 \cdot \overbrace{7 \cdot 6!} + 6 \cdot 6! &= 
                \hspace{0.25cm} 6! \hspace{0.25cm} \cdot \hspace{0.25cm} \underbrace{8 \cdot 7 - 7 \cdot 7 + 6} \\
                &= \overbrace{6 \cdot 5!} \hspace{0.675cm} \cdot \hspace{0.675cm} 13
            \end{align*}
            Denominator:
            \begin{align*}
                7! \hspace{0.525cm} - 6 \cdot \hspace{0.1cm} 6! \hspace{0.07cm} + 5 \cdot 5! & \\
                \overbrace{7 \cdot 6 \cdot 5!} - 6 \cdot \overbrace{6 \cdot 5!} + 5 \cdot 5! &= 
                5! \hspace{0.25cm} \cdot \hspace{0.25cm} \underbrace{7 \cdot 6 - 6 \cdot 6 + 5} \\
                &= 5! \hspace{0.725cm} \cdot \hspace{0.725cm} 11
            \end{align*}
            So, the original expression equals $\frac{6 \cdot 5! \cdot 13}{5! \cdot 11} =
            6 \cdot \frac{13}{11} = \frac{78}{11}$.
        \end{solution}
        \item How many odd positive integers are factors of $480$?
        \begin{solution}
            We first factor $480$ into $2^5 \cdot 3 \cdot 5$. The total number of factors would be
            $(5 + 1)(1 + 1)(1 + 1) = 24$, but the odd factors may not have any powers of $2$, so of
            these there are only $(1 + 1)(1 + 1) = 4$.
        \end{solution}
    \end{enumerate}
\end{document}