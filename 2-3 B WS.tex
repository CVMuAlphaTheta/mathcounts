\documentclass{article}

\usepackage{amsthm}
\usepackage{amsmath}
\usepackage[group-separator={,}]{siunitx}

\theoremstyle{definition}
\newtheorem*{solution}{Solution}

\title{Combinatorics Set B}
\author{}
\date{}

\begin{document}
\maketitle
\begin{enumerate}
    \item How many factors of $2^{95}$ are there which are greater than 
        $\num{1000000}$?
        \begin{solution}
            Since $2$ is the only prime factor of $2^{95}$, the factors of 
            $2^{95}$ must be powers of two. The largest power of two we can have 
            is $2^{95}$, and we need to find the smallest power of two which is 
            greater than $\num{1000000}$. You can just use some trial and error, 
            or if you're a computer nerd you'll know that $2^{20}$ is the 
            smallest power of two greater than $10^6$. There numbers that we 
            want to count are $2^{20}, 2^{21}, \dots, 2^{95}$, so there are $95 
            - 20 + 1 = 76$. You have to remember to add $1$ to the difference 
            there since the bounds are inclusive.
        \end{solution}
    \item What is the units digit of the sum $1! + 2! + 3! + \dots + 14! + 15!$?
        \begin{solution}
            The units digit of the sum of multiple numbers is only dependent on 
            the units digit of the numbers. Observe: $1! = 1$, $2! = 2$, $3! = 
            6$, $4! = 4$, $5! = 120$, $6! = 720 \dots$. Any factorial after $5!$ 
            is a multiple of $5!$, and since $5!$ is a multiple of $10$, any 
            factorial after $5!$ must be a multiple of $10$ and end with $0$. 
            Therefore, only the first $4$ factorials will contribute to the 
            units digit of the sum. The sum of the units digits of the first $4$ 
            factorials is $1 + 2 + 6 + 4 = 13$, so the answer is $3$.
        \end{solution}
    \item Numbers that read the same forward and backward are called 
        palindromes. How many three-digit numbers are palindromes?
        \begin{solution}
            The first digit can be chosen in $9$ ways, since it can't be $0$. 
            The last digit must also be the same as the first. The middle digit 
            can be chosen in $10$ ways. The total number is thus $10 \cdot 9 = 
            90$.
        \end{solution}
    \item What is the value of the expression $\frac{8! - 7 \cdot 7! + 6 \cdot 
        6!} {7! - 6 \cdot 6! + 5 \cdot 5!}$? Express your answer as a common 
        fraction.
        \begin{solution}
            With factorials, we can often do some factoring to simplify things 
            and avoid large numbers. First, the numerator and denominator have a 
            common factor of $5!$, so we can factor that out first: $\frac{8! - 
            7 \cdot 7! + 6 \cdot 6!}{7! - 6 \cdot 6! + 5 \cdot 5!} = \frac{5!(8 
            \cdot 7 \cdot 6 - 7 \cdot 7 \cdot 6 + 6 \cdot 6)}{5!(7 \cdot 6 - 6 
            \cdot 6 + 5)} = \frac{8 \cdot 7 \cdot 6 - 7^2 \cdot 6 + 6^2}{7 \cdot 
            6 - 6^2 + 5}$. We can factor a $6$ out of the top to get $6(8 \cdot 
            7 - 7^2 + 6) = 6 \cdot 13$, and the bottom evaluates to $11$, so the 
            answer is $\frac{6 \cdot 13}{11} = \frac{78}{11}$.
        \end{solution}
    \item How many odd positive integers are factors of $480$?
        \begin{solution}
            We first factor $480$ into $2^5 \cdot 3 \cdot 5$. Odd integers must 
            not have a factor of $2$, so the prime factorization of any odd 
            factor of $480$ can only contain $3$ and $5$. There are $4$ such 
            factors: $1$, $3$, $5$, and $3 \cdot 5 = 15$.
        \end{solution}
\end{enumerate}
\end{document}
