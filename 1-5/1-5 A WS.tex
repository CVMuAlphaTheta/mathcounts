\documentclass{article}

\usepackage{amsthm}

\theoremstyle{definition}
\newtheorem*{solution}{Solution}

\title{Sequences, Series, and Patterns Set A}
\date{}
\author{}

\begin{document}
    \maketitle
    \noindent Problems should be solved without calculators unless otherwise
    specified. Remember to explain how you solved a problem.
    \begin{enumerate}
        \item What is the value of $k$ in the arithmetic sequence $-13, -7, -1,
        k, 11$?
        \begin{solution}
            An arithmetic sequence has a constant difference between consecutive
            terms. We can see that the difference between each number is $6$ so
            the value of $k$ must be $-1 + 6 = 5$
        \end{solution}
        \item Insert two geometric means between $2$ and $16$.
        \begin{solution}
            We wish to find $x$ and $y$ such that $2, x, y, 16$ is a geometric
            sequence. Thus, $x = 2r$, $y = xr$, and $16 = yr$ for some r, so $16
            = yr = xr^2 = 2r^3$ and $r = 2$. Thus, the two numbers are $x = 2r =
            4$ and $y = 2x = 2 \cdot 4 = 8$.
        \end{solution}
        \item The students from Regent Middle School sat $20$ rows in such a way
        that each row after the first had $3$ more students sitting in it than
        the previous row. If $16$ students sat in the fifth row, how many
        students are there in total?
        \begin{solution}
            The number of students in each row form an arithmetic series, and we
            need to find its sum. We know that there are $16$ students in the
            fifth row, so to find the number of students in the first row we can
            use the formula for the terms of an arithmetic sequence to get $a_1
            = a_5 + (1 - 5)3 = 16 - 12 = 4$. Similarly, the $20$th term is $a_5
            + (20 - 5)3 = 16 + 45 = 61$. Using the arithmetic series sum
            formula, we see that there are a total of $\frac{20(4 + 61)}{2} =
            650$ students.
        \end{solution}
        \item A rectangular lawn is mowed every five days. On each of these
        days, the lawn is mowed in one direction, but the direction varies from
        one mowing to the next in the following order: East-West (E-W),
        Northeast-Southwest (NE-SW), North-South (N-S), and Northwest-Southeast
        (NW-SE). If the lawn is first mowed E-W on Day $1$, on Day $201$ in
        which direction is the lawn mowed, E-W, NE-SW, N-S, or NW-SE?
        \begin{solution}
            The lawn is mowed on days $1, 6, 11, 16, \dot, 191, 196, 201$. This
            is an arithmetic sequence and the common difference is $5$, since
            the lawn is mowed every $5$ days. To get from $1$ to $201$, the
            common difference will have been added $\frac{201 - 1}{5} =
            \frac{200}{5} = 40$ times. This means day $201$ will be the $41$st
            mowing. For the first $40$ mowing days, the lawn will have been
            mowed $\frac{40}{4} = 10$ times in each of the directions E-W,
            NE-SW, N-S, and NW-SE. Therefore, for the $41$st mowing, on Day
            $201$, the sequence of directions will begin again with a mowing in
            the direction of E-W or East-West.
        \end{solution}
        \item Cara received a rose that has $n$ petals, $\frac{n + 5}{3}$
        leaves, and $\frac{n-2}{4}$ thorns. The number of petals, leaves, and
        thorns are all numbers in the Fibonacci sequence $(1, 1, 2, 3, 5, 8,
        \dots)$, and $n < 100$. How many leaves does Cara's rose have?
        \begin{solution}
            Since $n < 100$, the number of petals on Cara's rose has to be one
            of the following Fibonacci numbers: $1$, $1$, $2$, $3$, $5$, $8$,
            $13$, $21$, $34$, $55$ and $89$. The numbers of thorns is $\frac{n -
            2}{4}$, so $n$ must be $2$ less than a multiple of $4$, which
            narrows it down to $2$ and $34$. The number of leaves is $\frac{n +
            5}{3}$, so $n$ must also be $5$ less than a multiple of $3$, which
            means $n$ must be $34$. Cara's rose has $34$ petals, $\frac{34 +
            5}{3} = 13$ leaves, and $\frac{34 - 2}{4} = 8$ thorns, and all of
            these are indeed Fibonacci numbers.
        \end{solution}
    \end{enumerate}
\end{document}