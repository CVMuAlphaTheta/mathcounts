\documentclass{article}

\usepackage[margin=0.5in]{geometry}
\usepackage{multicol}
\usepackage{siunitx}

\title{4.3 Notes}
\author{}
\date{}

\begin{document}
\maketitle

\begin{multicols}{2}
	\section*{Polygons}
	\subsection*{Classification}
	A polygon is \textbf{convex} if all of its interior angles are less than $\ang{180}$.
	Otherwise, it's \textbf{concave}.
	If all the edges have the same length, the polygon is \textbf{equilateral}, and if all the angles are the same, the polygon is \textbf{equiangular}.
	Note that with the exception of triangles, equilateral polygons are not necessarily equiangular and equiangular polygons are not necessarily equilateral.
	A \textbf{regular} polygon is both equilateral and equiangular.

	\subsection*{Interior Angle Sums}
	Recall that the sum of the interior angles of a triangle is always $\ang{180}$.
	A polygon with $n$ sides can always be divided into $n - 2$ triangles by drawing lines between vertices, so the sum of the interior angles is $180(n - 2)$.
	The sum of the exterior angles of a polygon is always $\ang{360}$.
	To see why, imagine walking along the edges of a polygon.
	When you encounter a vertex, you turn, and the amount by which you turn is the exterior angle at that vertex.
	If you walk all the way around a polygon, you would have turned $\ang{360}$ in total, so the sum of the exterior angles is $\ang{360}$.
	Note that if the interior angle is greater than $\ang{180}$, the corresponding exterior angle is negative.

	\subsection*{Diagonals}
	A \textbf{diagonal} of a polygon is a line segment connecting two vertices that is not one of the edges.
	A polygon with $n$ sides has $\frac{n(n - 3)}{2}$ diagonals, since each of the $n$ vertices has a diagonal to each of the $n - 3$ other non-adjacent vertices, and each diagonal connects two vertices.

	\section*{Trapezoids}
	\subsection*{Trapezoid Midsegment Theorem}
	A midsegment of a trapezoid is a segment connecting the midpoints of the two legs (the opposite sides that aren't parallel).
	The \textbf{trapezoid midsegment theorem} states that the midsegment of the trapezoid is parallel to the bases of the trapezoid, and its length is the average of the lengths of the bases.

	\subsection*{Area}
	The area of a trapezoid with height $h$ and with base lengths $b_1$ and $b_2$ is $\frac{h(b_1 + b_2)}{2}$.
	You can prove this by constructing a parallelogram using two copies of a trapezoid.

	\section*{Parallelograms}
	\subsection*{Rhombuses}
	A \textbf{rhombus} is an equilateral quadrilateral.
	All rhombuses are parallelograms, and all squares are rhombuses.
	The two diagonals of a rhombus are perpendicular bisectors of each other.

	\subsection*{Rectangles}
	The diagonals of rectangles bisect each other and have the same length.

	\subsection*{Squares}
	Squares are both rhombuses and rectangles, so the two diagonals of a square have the same length and are perpendicular bisectors of each other.

	\section*{Kites}
	A \textbf{kite} is a quadrilateral with two pairs of adjacent, equal-length sides.
	The area of a kite is half the product of the length of the diagonals.
\end{multicols}
\end{document}
