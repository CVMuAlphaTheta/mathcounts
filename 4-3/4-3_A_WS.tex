\documentclass{article}

\usepackage[margin=0.5in]{geometry}
\usepackage{multicol}
\usepackage{siunitx}
\usepackage{amsthm}

\theoremstyle{definition}
\newtheorem*{solution}{Solution}
\title{Polygons Set A}
\author{}
\date{}

\begin{document}
\maketitle

\begin{multicols}{2}
    \begin{enumerate}
        \item In parallelogram $ABCD$, diagonals $AC$ and $BD$ intersect at $E$.
            If $AE = 2x + 9$ and $CE = 5x$, what is the length of diagonal $AC$?
            \begin{solution}
                The diagonals of a parallelogram bisect each other, so the length of $AE$ and $CE$ must be equal.
                Solving for $2x + 9 = 5x$, we get $x = 3$, so $AE = CE = 15$ and $AC = 30$.
            \end{solution}
        \item What is the interior angle measure of a regular $15$-gon?
            Solve using two methods.
            \begin{solution}
                The exterior angle measure is easiest to find: $\frac{360}{15} = 24$, so the interior angle is its supplement $\ang{156}$.
                Alternatively, apply the equation $\frac{180(n - 2)}{n}$ to get the same result.
            \end{solution}
        \item What is the area of a triangle with integral side lengths and a perimeter of $8$ cm?
            \begin{solution}
                Consider the combinations of the integers that add up to $8$: $1 + 1 + 6$, $1 + 2 + 5$, $1 + 3 + 4$, $2 + 2+ 4$, $2 + 3 + 3$.
                According to the Triangle Inequality, only one of these sets of integers can be the lengths of the sides of a triangle: $2, 3$, and $3$.
                The altitude of this isosceles triangle meets the base at a right angle, so we can use the Pythagorean theorem to find the height $3^2 -1^2 = h^2$, so $h = \sqrt{8} = 2\sqrt{2}$ and the area is $1 \cdot 2\sqrt{2} = 2\sqrt{2}$.
            \end{solution}
        \item What is the smallest angle which can be created by connecting three vertices of a regular $36$-gon?
            \begin{solution}
                Connecting two adjacent points to any other vertex will create a $\ang{5}$ angle.
            \end{solution}
        \item In isosceles triangle $ABC$, $m\angle A = \ang{96}$.
            What is $m\angle B$?
            \begin{solution}
                At $\ang{96}$, $\angle A$ cannot be a base angle of an isosceles triangle.
                Thus $\angle B$ and $\angle C$ are base angles, each measuring $\ang{42}$.
            \end{solution}
    \end{enumerate}
\end{multicols}
\end{document}