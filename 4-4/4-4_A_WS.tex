\documentclass{article}

\usepackage[margin=0.5in]{geometry}
\usepackage{multicol}
\usepackage{siunitx}
\usepackage{amsthm}

\theoremstyle{definition}
\newtheorem*{solution}{Solution}
\title{Solid Geometry Set A}
\author{}
\date{}

\begin{document}
\maketitle
\begin{multicols}{2}
    \begin{enumerate}
        \item A right circular cone with base radius $\SI{4}{cm}$ has surface area $\SI{56\pi}{cm\squared}$.
            What is the height of the cone?
            Express your answer in simplest radical form.
            \begin{solution}
                Recall the formula for the surface area of a right circular cone $\text{SA} = \pi \cdot r^2 + \pi \cdot r \cdot \sqrt{r^2 + h^2}$.
                For this cone, $r = \SI{4}{cm}$ and $\text{SA} = \SI{56\pi}{cm\cubed}$.
                Substituting the known values into the formula gives $56\pi = \pi \cdot 4^2 + \pi \cdot 4 \cdot \sqrt{4^2 + h^2}$.
                Simplifying and combining like terms yields $56 = 16 + 4 \cdot \sqrt{16 + h^2} \rightarrow 40 = 4 \cdot \sqrt{16 + h^2} \rightarrow 10 = \sqrt{16 + h^2}$.
                Squaring both sides, we get $100 = 16 + h^2$, so $h^2 = 84$, and $h = \sqrt{84} = 2\sqrt{21}$ centimeters.
            \end{solution}
        \item What is the greatest distance between any two vertices of a rectangular prism with dimensions $\SI{8}{cm}$ by $\SI{9}{cm}$ by $\SI{12}{cm}$?
            \begin{solution}
                The greatest distance between any two vertices of a rectangular prism is called a ``space diagonal''.
                A space diagonal is an imagined line that connects two vertices and passes through the interior space of the rectangular prism.
                The length of a space diagonal can be calculated directly by using this three-dimensional version of the Pythagorean Theorem: $d^2 = a^2 + b^2 + c^2$.
                We have $d^2 = 8^2 + 9^2 + 12^2$, so $d = \sqrt{64 + 81 + 144} = \sqrt{289} = \SI{17}{cm}$.
            \end{solution}
        \item What is the number of solid $3$-inch by $2$-inch by $2$-inch rectangular prisms that can be arranged to form a solid cube of edge length $1$ foot?
            \begin{solution}
                Each of the rectangular prisms has a volume of $3 \cdot 2\cdot 2 = \SI{12}{in\cubed}$.
                A cube of edge length $1$ foot has volume $12 \cdot 12 \cdot 12 = \SI{1728}{in\cubed}$.
                It will take $1728 \div 12 = 144$ prisms to fill the cube.
            \end{solution}
        \item A cylindrical container holds $20$ fluid ounces.
            It has a radius of $3$ inches and a height of $12$ inches.
            How many fluid ounces will a similar container with a radius of $4.5$ inches hold?
            Express your answer as a decimal to the nearest tenth.
            \begin{solution}
                The ratio of the larger radius to the smaller radius is $\frac{4.5}{3} = \frac{3}{2}$.
                This is the scale factor between the two similar containers.
                The volume of the larger container will be greater by a factor that is the cube of this scale factor.
                We are told that the volume of the smaller container is $20$ fluid ounces, so the volume of the larger container will be $20 \cdot \left( \frac{3}{2} \right)^2 = 20 \cdot \frac{27}{8} = \frac{135}{2} = 67.5$ fluid ounces.
            \end{solution}
        \item Keaton wants to build a rectangular prism with volume $\SI{2016}{in\cubed}$ so that the length of each edge is a whole number of inches.
            What is the least possible sum of the three dimensions of the prism he builds?
            \begin{solution}
                The rectangular prism should be as close to a cube as possible, given the available factors.
                The prime factorization of $2016$ is $2^5 \cdot 3^2 \cdot 7$.
                We want to assign these prime factors to three groups that have as close to the same product as possible.
                One triple with a product of $2016$ is $8 \cdot 14 \cdot 18$, which makes the sum of the dimensions $8 + 14 + 18 = 40$.
                Another triple is $9 \cdot 14 \cdot 16$, for which $9 + 14 + 16 = 39$.
                THe triple with the lowest sum, however, is $12 \cdot 12 \cdot 14$, with a sum of $12 + 12 + 14 =38$.
            \end{solution}
    \end{enumerate}
\end{multicols}
\end{document}
