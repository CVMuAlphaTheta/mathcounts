\documentclass{article}
\usepackage{multicol}
\usepackage{siunitx}
\usepackage{enumitem}
\title{1.3 Notes}
\author{}
\date{}
\begin{document}
    \maketitle
    \section*{Key Concepts}
    \subsection*{The Distributive Property}
    \[a(b + c) = ab + ac\]
    \subsection*{Product of Sums}
    We can expand $(a + b)(c + d)$ by using the distributive property twice.
    First, we distribute $(a + b)$ to $c$ and $d$ to get $(a + b)c + (a + b)d$.
    Then, we distribute $c$ and $d$ to get $ac + bc + ad + bd$.
    \subsection*{Difference of Squares}
    A special case of the product of sums occurs when we expand $(a + b)(a -
    b)$. Using the technique from the previous section, we see that this expands
    to $a^2 + ab - ab - b^2 = a^2 - b^2$. This leads to the identity $(a + b)(a
    - b) = a^2 - b^2$. Recognizing the patterns of this identity is crucial to
    solving many problems.
    \subsection*{Squares of Sums}
    Another special case occurs when we expand $(a + b)^2$. We end up with $a^2
    + ab + ab + b^2 = a^2 + 2ab + b^2$.
    \subsection*{Exponent and Radical Rules}
    \[x^a \cdot x^b = x^{(a + b)}\]
    \[(xy)^a = x^a + y^a\]
    \[(x^a)^b = x^{ab}\]
    \[x^{\frac{a}{b}} = \sqrt[b]{x^a} = \left(\sqrt[b]{x}\right)^a\]
    \[x^{-a} = \frac{1}{x^a}\]
    \[\sqrt{ab} = \sqrt{a}\sqrt{b}\]
    \[(x + y)^a \neq x^a + y^a\]
    \[\sqrt{a + b} \neq \sqrt{a} + \sqrt{b}\]
    \textbf{Pay attention to the last two and don't make those mistakes!!!}
    \subsection*{Simplifying Radicals}
    Let's say we wanted to simplify $\sqrt{250} + \sqrt{360}$. There's no
    obvious way to do this, but we can first \textbf{simplify} the two square
    roots. To simplify a square root, we first factor out the largest perfect
    square from the number inside it. We can factor $250$ into $25 \cdot 10$.
    Then, we use the rule of distributing exponents to get $\sqrt{250} =
    \sqrt{25 \cdot 10} = \sqrt{25}\sqrt{10} = 5\sqrt{10}$. Similarly,
    $\sqrt{360} = \sqrt{36 \cdot 10} = \sqrt{36}\sqrt{10} = 6\sqrt{10}$. Now we
    can add $5\sqrt{10}$ and $6\sqrt{10}$ to get $11\sqrt{10}$. When your
    answers contain square roots, you will generally be required to simplify
    them.
    \section*{Distribution and Factoring Practice}
    \begin{multicols}{2}
        \subsection*{The Distributive Property}
        \begin{enumerate}
            \item $4x - 3(x - 2)$
            \vspace{0.5cm}
            \item $-(3b - 4a)$
            \vspace{0.5cm}
            \item $a(b + c)$
            \vspace{0.5cm}
            \item $(18a + 3)$
            \vspace{0.5cm}
            \item $8x^2 y + 24xy$
            \vspace{0.5cm}
        \end{enumerate}
        \subsection*{Polynomials}
        \begin{enumerate}
            \item $(3x - 3)(4x - 1)$
            \vspace{0.5cm}
            \item $(8a - 2b - 3)(a + 2b)$
            \vspace{0.5cm}
            \item $(a + b)(c + d)$
            \vspace{0.5cm}
            \item $6x^2 - 23x - 18$
            \vspace{0.5cm}
            \item $16x^2 - 38x - 5$
            \vspace{0.5cm}
        \end{enumerate}
        \subsection*{Difference of Squares}
        \begin{enumerate}
            \item $(a - 5)(a + 5)$
            \vspace{0.5cm}
            \item $(6x - 7)(6x + 7)$
            \vspace{0.5cm}
            \item $(a + b)(a - b)$
            \vspace{0.5cm}
            \item $x^2 - 36x$
            \vspace{0.5cm}
            \item $4x^4 - 36x^2$
            \vspace{0.5cm}
        \end{enumerate}
        \subsection*{Perfect Squares}
        \begin{enumerate}
            \item $(x - 5)^2$
            \vspace{0.5cm}
            \item $(3x + 1)^2$
            \vspace{0.5cm}
            \item $(a + b)^2$
            \vspace{0.5cm}
            \item $x^2 - 10x + 25$
            \vspace{0.5cm}
            \item $36x^2 - 60xy + 25y^2$
            \vspace{0.5cm}
        \end{enumerate}
    \end{multicols}
    \section*{Example Problems}
    \subsection*{Distribution}
    \begin{enumerate}
        \item A square has an area of \SI{50}{\centi\meter\squared}. The length
        is increased by \SI{4}{\centi\meter} and the width is decreased by
        \SI{4}{\centi\meter}. What is the area of the resulting rectangle?
        \vspace{3cm}
        \item The sum of two numbers is five and their product is two. What is
        the sum of the squares of the two numbers?
        \vspace{3cm}
        \item If $ab = 7$, $bc = 5$ and $a + c = 4$, what is $a + b + c$?
        \vspace{3cm}
        \item The length of a rectangle is $6$ units greater than its width. If
        the area of the rectangle is $216$ square units, what is the width?
        \vspace{3cm}
        \item What is the smallest possible sum of two positive integers whose
        product is $999996$?
        \vspace{3cm}
    \end{enumerate}
    \subsection*{Factoring}
    \begin{enumerate}[resume]
        \item The perimeter of a rectangle is \SI{16}{\centi\meter}, and the
        area of the same rectangle is \SI{8}{\centi\meter\squared}. What is the
        diagonal length of the rectangle?
        \vspace{3cm}
        \item The difference between the squares of two numbers is $80$. If the
        sum of the two numbers is $16$, what is their positive difference?
        \vspace{3cm}
        \item Solve for $x$: $\frac{2x^2 - 5x - 12}{2x + 3} = 9$
        \vspace{3cm}
        \item The sum of two positive numbers is $6$, and the sum of their
        squares is $22$. What is the sum of their reciprocals? Express your
        answer as a common fraction.
        \vspace{3cm}
        \item Evaluate $100002^2 - 99998^2$.
        \vspace{3cm}
    \end{enumerate}
\end{document}