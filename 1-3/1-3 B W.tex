\documentclass{article}
\usepackage{asymptote}
\title{Distribution and Factoring Set B}
\author{}
\date{}
\begin{document}
    \maketitle
    \noindent Problems should be solved without calculators unless otherwise
    specified. Remember to explain how you solved each problem.
    \begin{enumerate}
        \item What is the sum of the distinct prime factors of $156$?
        \vspace{3cm}
        \item What is the sum of two consecutive positive integers whose squares
        differ by $45$?
        \vspace{3cm}
        \item This problem will introduce one of the many proofs of the
        \textbf{Pythagorean theorem}. This important theorem establishes a
        relationship between the lengths of the sides of right triangles:
        \begin{center}
            \begin{asy}
                import olympiad;

                pair a = (0, 0), b = (0, 60), c = (80, 0);
                draw(a -- b -- c -- cycle);
                draw(rightanglemark(c, a, b, 160));
                label("$a$", a -- b, W);
                label("$b$", a -- c, S);
                label("$c$", b -- c, NE);
            \end{asy}
        \end{center}
        \[a^2 + b^2 = c^2\] $a$ and $b$ are the lengths of the legs, which are
        the sides of a right triangle that form the right angle. $c$ is the
        length of the hypotenuse, which is the side opposite to the right angle.
        This proof uses a square inscribed inside another square, forming four
        congruent right triangles:

        \begin{center}
            \begin{asy}
                import olympiad;

                pair a = (-50, -50), b = (-50, 50), c = (50, 50), d = (50, -50);
                draw(a -- b -- c -- d -- cycle);
                pair p = waypoint(a -- b, 0.3), q = waypoint(b -- c, 0.3), r = waypoint(c -- d, 0.3), s = waypoint(d -- a, 0.3);
                draw(p -- q -- r -- s -- cycle);
                draw(rightanglemark(p, a, s, 160));
                label("$a$", a -- p, W);
                label("$b$", a -- s, S);
                label("$c$", p -- s, NE);
                label("$a$", s -- d, S);
            \end{asy}
        \end{center}
        \begin{enumerate}
            \item Write an expression for the area of the smaller square in
            terms of $c$.
            \vspace{3cm}
            \item Write an expression for the area of the smaller square in
            terms of $a$ and $b$. Hint: find the area of the larger square by
            squaring its side length and then subtract the area of the four
            triangles.
            \vspace{3cm}
            \item Explain why this proves the Pythagorean theorem.
            \vspace{3cm}
        \end{enumerate}
        \item If $a + b = 1$ and $a^2 + b^2 = 2$, find $a^4 + b^4$.
        \vspace{3cm}
        \item Given that $9876^2 = 97535376$, find $9877^2$ without using
        multiplication.
        \vspace{3cm}
    \end{enumerate}
\end{document}