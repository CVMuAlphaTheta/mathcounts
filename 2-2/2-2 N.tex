\documentclass[twocolumn]{article}

\usepackage[margin=0.5in]{geometry}
\usepackage{flushend}
\usepackage{amsmath}
\usepackage{asymptote}
\usepackage{enumitem}

\title{2.2 Notes}
\author{}
\date{}

\begin{document}
\maketitle

\section*{Key Concepts}

\subsection*{Terminology}
A \textbf{set} is a collection of distinct \emph{things}. For example, we can 
have a set of integers, a set of people, a set of English words, etc. Sets can 
be written as a list of things enclosed in curly brackets, such as $\{1, 2, 3, 
4, 5\}$ or $\{\text{Alex}, \text{Chloe}, \text{Cedric}\}$. Sets usually contain 
things with some similarities, but it's possible to have a set containing 
completely unrelated things, like $\{\text{Cedric}, 42, 
\text{``sacrilegious''}\}$.

\subsection*{Venn Diagrams}
Venn diagrams are useful for visualizing sets with common elements. To draw a 
Venn diagram for the set of people who like math and the set of people who like 
to play sports, we would draw two intersecting circles. One circle represents 
the people who like math, and the other circle represents the people who like to 
play sports. The area where the circles intersect represent the people who both 
like math and like to play sports.
\begin{center}
	\begin{asy}
		draw(circle((-40, 0), 80));
		draw(circle((40, 0), 80));
		label("Math", (-40, 80), N);
		label("Sports", (40, 80), N);
		label(rotate(-90) * "Both math and sports", (0, 0));
		label("Only math", (-80, 0));
		label("Only sports", (80, 0));
		label("Neither math nor sports", (0, -120));
	\end{asy}
\end{center}
From this diagram, we can see that the sum of the number of people who only like 
math and the number of people who like sports is the number of people who like 
math or like sports. We can also see that the number of people who only like 
math is the number of people who like math minus the number of people who like 
both math and sports. When you solve a problem involving intersecting sets like 
this, you can use a Venn diagram to help visualize the problem. Write in numbers 
which you know and try to find the numbers which aren't given.

\subsection*{Inclusion-Exclusion Principle}
The \textbf{inclusion-exclusion principle} states that the number of things 
which are in either of two sets is the sum of the sizes of the two sets minus 
the number of things which are in two sets. For example, if $3$ people like 
math, $5$ people like sports, and $2$ people like both math and sports, then $3 
+ 5 - 2 = 6$ people like either math or sports (or both). We can use the Venn 
diagram to see why this works. If we added size of the math circle and the size 
of the sports circle together, we would count the region where the circles 
intersect twice. Therefore, we subtract the size of that region to get the size 
of the region formed by combining the two circles.

Note that the inclusion-exclusion principle can also be used in reverse. We can 
find the number of people who like both math and sports by adding the number of 
people who like math to the number of people who like sports and subtracting the 
number of people who like either math or sports. When we add the size of the 
math circle with the size of the sports circle, we count the left region once, 
the middle region twice, and the right region once. When we subtract the size of 
the region formed by combing the circles, we will end up with just the middle 
region.

\subsection*{Two-Way Tables}
\textbf{Two-way tables} are another way to visualize intersecting sets. A 
two-way table looks like this:
\begin{center}
	\begin{tabular}{| c | c | c | c |}
		\hline
                        & Likes math & Dislikes math & Total \\
		\hline
		Likes sports    & $2$        & $3$           & $5$   \\
		\hline
		Dislikes sports & $1$        & $4$           & $5$   \\
		\hline
		Total           & $3$        & $7$           & $10$  \\
		\hline
	\end{tabular}
\end{center}

Notice that each of the cells in the total row is the sum of the numbers in the 
corresponding column, and each of the cells in the total column is the sum of 
the numbers in the corresponding row. For example, the total number of people 
who like math is the sum of the people who like math and like sports, and the 
people who like math and don't like sports.

We can solve problems using two way tables by filling in cells which we know and 
then using the above property to fill in the rest of the cells. For example, 
lets say out of the $15$ people in a class, $4$ people like dogs, $6$ people 
like cats, and $2$ people like dogs but don't like cats. We would have the 
following two-way table, where the letters represent unknown values:
\begin{center}
	\begin{tabular}{| c | c | c | c |}
		\hline
                      & Likes dogs & Dislike dogs & Total \\
		\hline
		 Likes cats   & $A$        & $B$          & $6$   \\
		 \hline
		Dislikes cats & $2$        & $C$          & $D$   \\
		\hline
		Total         & $4$        & $E$          & $15$  \\
		\hline
	\end{tabular}
\end{center}
In the first column, we have $A + 2 = 4$, so $A = 2$. In the third column, we 
see that $6 + D = 15$, therefore $D = 9$. In the last row, $4 + E = 15$, 
therefore $E = 11$. Now our table looks like this:
\begin{center}
	\begin{tabular}{| c | c | c | c |}
		\hline
                      & Likes dogs & Dislike dogs & Total \\
		\hline
		 Likes cats   & $2$        & $B$          & $6$   \\
		 \hline
		Dislikes cats & $2$        & $C$          & $9$   \\
		\hline
		Total         & $4$        & $11$         & $15$  \\
		\hline
	\end{tabular}
\end{center}
In the second row, we see that $2 + C = 9$, therefore $C = 7$. Then in the 
second column we have $B + 7 = 11$, so $B = 4$. Here's the finished table:
\begin{center}
	\begin{tabular}{| c | c | c | c |}
		\hline
                      & Likes dogs & Dislike dogs & Total \\
		\hline
		 Likes cats   & $2$        & $4$          & $6$   \\
		 \hline
		Dislikes cats & $2$        & $7$          & $9$   \\
		\hline
		Total         & $4$        & $11$         & $15$  \\
		\hline
	\end{tabular}
\end{center}
You can verify that everything adds up. Now we know the number of people in any 
of the categories.

\subsection*{The Fundamental Counting Principle}
\textbf{The Fundamental Counting Principle} simply states that if there are 
always $n$ ways to do one thing and $m$ ways to do another thing, there are $nm$ 
ways to do both things. For example, if I have $6$ shirts and $8$ pairs of 
pants, then there are $6 \cdot 8 = 48$ possible outfits. Note that the 
fundamental counting principle only applies if the two things are independent.  
We can't use it if certain combinations of shirts and pants aren't allowed for 
example, because then the number of ways to choose the pants would depend on 
which shirt I chose and not be constant.

\section*{Problems}

\subsection*{Counting Basics and Venn Diagrams}
\begin{enumerate}
	\item You are assigned even problems from $10$ through $40$ for tonight's 
		homework. How many problems is this?
		\vspace{3cm}
	\item The circumference of a circular table is $30$ feet. If a set of 
		silverware is placed every three feet around the circumference of the 
		table, how many place-settings are there?
		\vspace{3cm}
	\item In a science classroom: $19$ students have a brother, $15$ students 
		have a sister, $7$ students have both a brother and a sister, and $6$ 
		students don't have any siblings at all. How many students are in the 
		classroom?
		\vspace{3cm}
	\item At the pound there are $40$ dogs. If $22$ dogs have spots and $30$ 
		dogs have short hair, what is the feweset number of dogs that can have 
		short hair and spots?
		\vspace{3cm}
\end{enumerate}

\subsection*{Handshake Problems}
\begin{enumerate}[resume]
	\item $12$ strangers go bowling. If everyone shakes everyone else's hand 
		exactly once, how many handshakes have occurred?
		\vspace{3cm}
	\item What is the general formula for the number of hand-shakes that occur 
		when $n$ people shake hands?
		\vspace{3cm}
	\item Several couples arrive at a dinner party. Each person at the party 
		shakes the hand of every other person, not including his or her spouse.  
		If there were a total of $112$ handshakes, how many couples attended the 
		party?
		\vspace{3cm}
	\item If $99$ people stand in a circle, and each skaes the hand of the 
		person on each side, how many handshakes are there all together?
		\vspace{3cm}
\end{enumerate}

\subsection*{Intermediate Counting Problems}
\begin{enumerate}[resume]
	\item How many even perfect squares are there from $100$ to $10000$ 
		inclusive?
		\vspace{3cm}
	\item How many whole numbers less than $100$ are multiples of $3$ but not 
		multiples of $5$?
		\vspace{3cm}
\end{enumerate}
\end{document}
