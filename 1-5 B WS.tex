\documentclass{article}

\usepackage{amsthm}
\usepackage{amsmath}

\theoremstyle{definition}
\newtheorem*{solution}{Solution}

\title{Sequences, Series, and Patterns Set B}
\date{}
\author{}

\begin{document}
    \maketitle
    \noindent Problems should be solved without calculators unless otherwise
    specified. Remember to explain how you solved a problem.
    \begin{enumerate}
        \item The fourth term of an arithmetic sequence is $3$, and the ninth
        term is $23$. What is the first term of this sequence?
        \begin{solution}
            In an arithmetic sequence there is a constant difference between
            consecutive terms. To get from the fourth term to the ninth term,
            which is $23$, the constant difference $d$ must be added $5$ times.
            We can write $3 + 5d = 23$, so $5d = 20$ and $d = 4$. Now to go
            backward from the fourth term to this first term, we need to
            subtract the common difference $3$ times. So, the first term of this
            sequence is $3 - 3 \cdot 4 = 3 - 12 = -9$.
        \end{solution}
        \item Once upon a time, Carl and Steve played a game of ping pong. Carl
        lost, and he was very salty. He wanted to be better than Steve at
        something, so he challenged Steve to solve a math problem. Steve was too
        lazy, so Carl said that he would give Steve $\$25$ if he could compute
        the total distance which a ping pong ball dropped from a height of $10$
        meters will travel, assuming that ping pong balls always bounce up to
        $\frac{2}{3}$ of the height from which they fell. Steve is very bad at
        math but he really wants $\$25$ to buy Minecraft, so he asked you to
        help him. Help Steve buy Minecraft by solving this problem for him.
        Hint: Count the initial fall separately. After the first bounce, the
        ball will rise to $10\left(\frac{2}{3}\right)$ meters before falling
        back down. After the second bounce, the ball will rise up to
        $10\left(\frac{2}{3}\right)^2$ meters before falling back down. This
        continues forever. How far does the ball travel between the first bounce
        and the second bounce? How far does it travel between the second bounce
        and the third bounce?
        \begin{solution}
            The ball first falls $10$ meters. Then it bounces up
            $10\left(\frac{2}{3}\right)$ meters and falls back down, so it
            traveled $20\left(\frac{2}{3}\right)$ between the first and the
            second bounce. The maximum height after the second bounce is
            $10\left(\frac{2}{3}\right)^2$ meters, so the ball traveled
            $20\left(\frac{2}{3}\right)^2$ meters between the second and third
            bounce. Notice that we have an infinite geometric series here, with
            first term $20\left(\frac{2}{3}\right) = \frac{40}{3}$ and common
            ratio $\frac{2}{3}$. Its sum is $\frac{\frac{40}{3}}{1 -
            \frac{2}{3}} = 40$. Adding the initial $10$ meter fall, we see that
            the ball will travel a total of $40 + 10 = 50$ meters.
        \end{solution}
        \item Evaluate
        \begin{multline*}
            (751 - 745) + (748 - 742) + (745 - 739) + (742 - 736) + \dots \\
            + (499 - 493) + (496 - 490)
        \end{multline*}
        \begin{solution}
            Each pair evaluates to $6$, so we could just count how many pairs
            there are. We can do this by looking at the arithmetic sequence
            formed by the first number in each pair. It starts at $751$, ends at
            $496$, and has a common difference of $-3$, so there are $\frac{496
            - 751}{-3} + 1 = 86$ terms, and the sum is $86 \cdot 6 = 516$.
            Another way to solve this would be to notice that the parenthesis
            don't matter. If we ignore them, we can see that $-745$ and $745$
            cancel out, $-742$ and $742$ cancel out, and so on. In the end,
            we're left with $751 + 748 - 493 - 490 = 516$.
        \end{solution}
        \item The movies in the Revengers series have been coming out every
        three years. The sum of the years in which the first six movies came out
        is $12057$. In what year will the seventh movie come out if the pattern
        continues?
        \begin{solution}
            Let's say that the first movie in the series came out in the year
            $Y$. Then the second movie came out in year $Y + 3$, the third in
            year $Y + 6$, and so on. The sum of the release years of the first
            six movies is $Y + (Y + 3) + (Y + 6) + (Y + 9) + (Y + 12) + (Y + 15)
            = 6Y + 45$. We are told this sum equals $12057$, so we can solve
            $6Y + 45 = 12057$ for $Y$. Doing so, we get $6Y = 12012$ and $Y =
            2002$. If the movie came out in $2002$, then the seventh movie is
            expected to come out in $2002 + 3 \cdot 6 = 2002 + 18 = 2020$.
        \end{solution}
        \item A bucket initially has $5$ milliliters of water. Every day, Emily
        removes half of the water, and then Katie adds $5$ milliliters of water.
        Write a formula for the number of milliliters of water in the bucket
        after $x$ days. Hint: After the first day, the amount of water in the
        bucket is $5 \cdot \frac{1}{2} + 5$. After the second day, the amount of
        water in the bucket is $\left(5 \cdot \frac{1}{2} + 5\right)\frac{1}{2}
        + 5 = 5\left(\frac{1}{2}\right)^2 + 5 \cdot \frac{1}{2} + 5$
        milliliters. After the third day, the amount of water in the bucket is
        $\left(\left(5 \cdot \frac{1}{2} + 5\right)\frac{1}{2} +
        5\right)\frac{1}{2} + 5 = 5\left(\frac{1}{2}\right)^3 +
        5\left(\frac{1}{2}\right)^2 + 5 \cdot \frac{1}{2} + 5$.
        \begin{solution}
            The first term of the geometric series is $5$ and the common ratio
            is $\frac{1}{2}$. The number of terms in the series is one more than
            the number of days. Therefore, there are $x + 1$ terms, so the sum
            is $5 \cdot \frac{1 - \left(\frac{1}{2}\right)^{x + 1}}{1 -
            \frac{1}{2}} = 10 - \frac{5}{2^x}$.
        \end{solution}
    \end{enumerate}
\end{document}
