\documentclass{article}
\usepackage{amsmath}
\title{Distribution and Factoring Set A}
\author{}
\date{}
\begin{document}
    \maketitle
    \noindent Problems should be solved without calculators unless otherwise specified. Remember to explain how you solved each problem.
    \begin{enumerate}
        \item What is the value of $\left(\frac{1}{2}\right)^{-3}$?
        \vspace{3cm}
        \item Show that $a^3 - b^3 = (a - b)(a^2 + ab + b^2)$ and $a^3 + b^3 = (a + b)(a^2 - ab + b^2)$.
        \vspace{3cm}
        \item What is the largest prime factor of $8 \cdot 7 \cdot 6 \cdot 5 \cdot 4 \cdot 3 - 6 \cdot 5 \cdot 4 \cdot 3 \cdot 2 \cdot 1$?
        \vspace{3cm}
        \item If $(x^2 y^3)^4 (x^4 y^5)^6 = x^a y^b$ for all real numbers $x$ and $y$, what is the sum of $a$ and $b$?
        \vspace{3cm}
        \item This problem is an introduction to solving quadratic equations by factoring. \textbf{Quadratic equations} are essentially equations containing $x^2$. They are often written in \textbf{standard form}, which is $ax^2 + bx + c = 0$ where $a$, $b$, and $c$ are constants. For example, $3x^2 + 4x - 7 = 0$ is a quadratic equation. Factoring is the fastest way to solve a quadratic, but this method can't be used to solve all quadratics.
        \begin{enumerate}
            \item Show that $(x + a)(x + b) = x^2 + (a + b)x + ab$.
            \vspace{3cm}
            \item If $(x + a)(x + b) = x^2 + 5x + 6$, find $a$ and $b$. Hint: Using the result from part a, we see that
            \[\begin{alignedat}{3}
                (x + a)(x + b) & = x^2 &{} + (a + b)x &&{} + ab &\\
                & = x^2 &{} + 5x &&{} + 6 &
            \end{alignedat}\]
            Determine the values of $a + b$ and $ab$, then use guess-and-check to find $a$ and $b$.
            \vspace{3cm}
            \item To solve $x^2 + 5x + 6 = 0$, we can first \textbf{factor} the left side by rewriting it as $(x + a)(x + b)$ for some constants $a$ and $b$ so that we have $(x + a)(x + b) = 0$. Now that you've found $a$ and $b$, what are the solutions to this equation? Note that ``solutions'' is plural.
            \vspace{3cm}
        \end{enumerate}
    \end{enumerate}
\end{document}